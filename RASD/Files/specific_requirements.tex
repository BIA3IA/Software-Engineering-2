% --------------------------------------------------------------------------
%  Specific requirements % 
% --------------------------------------------------------------------------
\section{External Interface Requirements}
\label{sec:external_interface_requirements}

\subsection{User Interfaces}
\label{subsec:user_interfaces}

\subsection{Hardware Interfaces}
\label{subsec:hardware_interfaces}

\subsection{Software Interfaces}
\label{subsec:software_interfaces}

\subsection{Communication Interfaces}
\label{subsec:communication_interfaces}

% --------------------------------------------------------------------------
%  Functional requirements % 
% --------------------------------------------------------------------------
\section{Functional Requirements}
\label{sec:functional_requirements}

\subsection{Requirements}
\label{subsec:requirements}

\subsection{Use Case Diagrams}
\label{subsec:use_case_diagrams}

\subsubsection{Use Cases}
\label{subsubsec:use_cases}

\textbf{[UC1]} - User Registration
\begin{table}[H]
	\centering
	\label{tab:UC1}
	\begin{tabularx}{\textwidth}{|p{0.3\textwidth}|X|}
		\hline
		\textbf{Actor(s)}        & Guest User                                                                                               \\ \hline
		\textbf{Entry Condition} & The actor does not have an account, wants to register and is on the app’s welcome page.                  \\ \hline
		\textbf{Event Flow}      & 1. The actor clicks on the “Sign Up” button.                                                             \\
		                         & 2. The system displays the registration form.                                                            \\
		                         & 3. The actor fills in the required personal information (e.g., name, email, password).                   \\
		                         & 4. The system validates the provided data.                                                               \\
		                         & 5. If validation succeeds, the system creates a new account and send a success message.                  \\ \hline
		\textbf{Exit Condition}  & A new user account is created.                                                                           \\ \hline
		\textbf{Exceptions}      & If the data are invalid or incomplete, the system displays an error message and requests correction.     \\
		                         & If the email is already registered, the system displays an error message and prompts the user to log in. \\
		                         & If a temporary connection or server error occurs, the system displays an appropriate error message.      \\ \hline
		\textbf{Notes}           & The entered password is stored in a hashed form for security purposes.                                   \\ \hline
	\end{tabularx}
	\caption{User Registration Process Detail}
\end{table}

% image
\begin{figure}[H]
	\centering
	\includegraphics[width=0.7\textwidth]{Images/SequenceDiagrams/user_registration.png}
	\caption{User Registration Sequence Diagram}
	\label{fig:use_case_user_registration}
\end{figure}


\textbf{[UC2]} - User Log In
\begin{table}[H]
	\centering
	\label{tab:UC2}
	\begin{tabularx}{\textwidth}{|p{0.3\textwidth}|X|}
		\hline
		\textbf{Actor(s)}        & Guest User                                                                                                                       \\ \hline
		\textbf{Entry Condition} & The actor has an account, wants to access it and is on the application's welcome page.                                           \\ \hline
		\textbf{Event Flow}      & 1. The actor clicks the “Log In” button.                                                                                         \\
		                         & 2. The system displays the login form.                                                                                           \\
		                         & 3. The actor enters valid credentials (email and password).                                                                      \\
		                         & 4. The system validates the credentials against the stored data.                                                                 \\
		                         & 5. If validation succeeds, the system authenticates the user, initiates a session, and sends a success message.                  \\ \hline
		\textbf{Exit Condition}  & The actor is successfully authenticated.                                                                                         \\ \hline
		\textbf{Exceptions}      & If the credentials are invalid or incomplete, the system displays an error message and prevents access until corrected.          \\
		                         & If the credentials are unrecognized,  the system displays an error message and allows another login attempt.                     \\
		                         & If there is a temporary connection or server error,  the system displays an error message and allows retry.                      \\ \hline
		\textbf{Notes}           & After login, a valid session or authentication token is created; the session expires automatically after a period of inactivity. \\ \hline
	\end{tabularx}
	\caption{User Log In Process Detail}
\end{table}

\begin{figure}[H]
	\centering
	\includegraphics[width=0.7\textwidth]{Images/SequenceDiagrams/user_login.png}
	\caption{User Log In Sequence Diagram}
	\label{fig:use_case_user_login}
\end{figure}

\textbf{[UC3]} - User Log Out
\begin{table}[H]
	\centering
	\label{tab:UC3}
	\begin{tabularx}{\textwidth}{|p{0.3\textwidth}|X|}
		\hline
		\textbf{Actor(s)}        & Logged-in User                                                                                                                                                          \\ \hline
		\textbf{Entry Condition} & The actor is authenticated and currently using the application (any page).                                                                                              \\ \hline
		\textbf{Event Flow}      & 1. The actor clicks on the hamburger menu.                                                                                                                              \\
		                         & 2. The system displays the menu with the “Log out” option.                                                                                                              \\
		                         & 3. The actor selects the “Log out” option.                                                                                                                              \\
		                         & 4. The system prompts the actor to confirm the logout action.                                                                                                           \\
		                         & 5. The system invalidates the current session.                                                                                                                          \\
		                         & 6. The system clears user-specific local data and cached preferences as required.                                                                                       \\
		                         & 7. The system redirects the actor to the welcome page and displays a confirmation message.                                                                              \\ \hline
		\textbf{Exit Condition}  & The actor is successfully logged out.                                                                                                                                   \\ \hline
		\textbf{Exceptions}      & If the server cannot invalidate the session or the network is unavailable (temporary error), the system displays an error message and prompts the actor to retry later. \\ \hline
		\textbf{Notes}           & Access tokens are revoked; next access requires re-authentication.                                                                                                      \\ \hline
	\end{tabularx}
	\caption{User Log Out Process Detail}
\end{table}

\begin{figure}[H]
	\centering
	\includegraphics[width=0.7\textwidth]{Images/SequenceDiagrams/user_logout.png}
	\caption{User Log Out Sequence Diagram}
	\label{fig:use_case_user_logout}
\end{figure}

\textbf{[UC4]} - Search for a Path
\begin{table}[H]
	\centering
	\label{tab:UC4}
	\begin{tabularx}{\textwidth}{|p{0.3\textwidth}|X|}
		\hline
		\textbf{Actor(s)}        & Guest User, Logged-in User                                                                                          \\ \hline
		\textbf{Entry Condition} & The actor is on the home page and wants to find a bike path.                                                        \\ \hline
		\textbf{Event Flow}      & 1. The actor selects the “Search” option.                                                                           \\
		                         & 2. The system displays the search panel.                                                                            \\
		                         & 3. The actor enters the start and destination points.                                                               \\
		                         & 4. The system validates the provided inputs.                                                                        \\
		                         & 5. The system activates the Path Suggestion process to find possible routes.                                        \\
		                         & 6. The system displays the list of suggested routes to the actor.                                                   \\ \hline
		\textbf{Exit Condition}  & The suggested routes are displayed to the actor.                                                                    \\ \hline                                                        \\ \hline
		\textbf{Exceptions}      & If the inputs are invalid or incomplete,  the system displays an error message and prevents search until corrected. \\
		                         & If no route is found, the system displays an error message and suggests adjusting inputs/filters.                   \\
		                         & If there is a temporary connection or server error, the system displays an error message and allows retry.          \\ \hline
		\textbf{Notes}           & None.                                                                                                               \\ \hline
	\end{tabularx}
	\caption{Search for a Path Process Detail}
\end{table}

\begin{figure}[H]
	\centering
	\includegraphics[width=0.7\textwidth]{Images/SequenceDiagrams/search_path.png}
	\caption{Search for a Path Sequence Diagram}
	\label{fig:use_case_search_path}
\end{figure}

\textbf{[UC5]} - Select a Path
\begin{table}[H]
	\centering
	\label{tab:UC5}
	\begin{tabularx}{\textwidth}{|p{0.3\textwidth}|X|}
		\hline
		\textbf{Actor(s)}        & Guest User, Logged-in User                                                                                                   \\ \hline
		\textbf{Entry Condition} & The actor has just performed a search or is browsing the catalog and is viewing the list of suggested paths.                 \\ \hline
		\textbf{Event Flow}      & 1. The actor selects one path from the results list.                                                                         \\
		                         & 2. The system displays the path details (overview, distance, ranking, reports).                                              \\
		                         & 3. The actor confirms the selection.                                                                                         \\ \hline
		\textbf{Exit Condition}  & The path is selected. The system is ready to start the trip.                                                                 \\ \hline
		\textbf{Exceptions}      & If the selected path is unavailable (removed/updated), the system displays an error message and returns to the results list. \\
		                         & If there is a temporary connection or server error, the system displays an error message and allows retry.                   \\ \hline
		\textbf{Notes}           & Only Logged-in Users can store trip data; Guest Users can follow the path without recording.                                 \\ \hline
	\end{tabularx}
	\caption{Select a Path Process Detail}
\end{table}

\begin{figure}[H]
	\centering
	\includegraphics[width=0.7\textwidth]{Images/SequenceDiagrams/select_path.png}
	\caption{Select a Path Sequence Diagram}
	\label{fig:use_case_select_path}
\end{figure}

\textbf{[UC6]} - Create a Path in Manual Mode
\begin{table}[H]
	\centering
	\label{tab:UC6}
	\begin{tabularx}{\textwidth}{|p{0.3\textwidth}|X|}
		\hline
		\textbf{Actor(s)}        & Logged-in User                                                                                                                                            \\ \hline
		\textbf{Entry Condition} & The actor is authenticated and wants to create a new path manually.                                                                                       \\ \hline
		\textbf{Event Flow}      & 1. The actor selects 'Create Path'.                                                                                                                       \\
		                         & 2. The system displays the available creation modes.                                                                                                      \\
		                         & 3. The actor chooses 'Manual'.                                                                                                                            \\
		                         & 4. The system displays the input page for manual creation (metadata + map editor).                                                                        \\
		                         & 5. The actor fills in metadata (e.g., name, description, visibility) and draws/adds segments on the map.                                                  \\
		                         & 6. The system validates the provided metadata and the geometry of the segments.                                                                           \\
		                         & 7. If validation succeeds, the system saves the path.                                                                                                     \\
		                         & 8. The system displays a success message and the new path appears in the actor's path list.                                                               \\ \hline
		\textbf{Exit Condition}  & The new path is successfully created, stored in the database, and associated with the actor’s account.                                                    \\ \hline
		\textbf{Exceptions}      & If the data is invalid or incomplete, the system displays an error message and prevents saving until corrected.                                           \\
		                         & If the geometry is invalid (e.g., disconnected segments, self-intersections), the system displays an error message and prompts the actor to fix the path. \\
		                         & If there is a temporary connection or server error, the system displays an error message and allows retry.                                                \\ \hline
		\textbf{Notes}           & The path visibility follows the selected setting. Geometry validation includes continuity and consistency checks.                                         \\ \hline
	\end{tabularx}
	\caption{Create a Path in Manual Mode — Process Detail}
\end{table}

\begin{figure}[H]
	\centering
	\includegraphics[width=0.7\textwidth]{Images/SequenceDiagrams/manually_create.png}
	\caption{Create a Path in Manual Mode — Sequence Diagram}
	\label{fig:use_case_create_path_manual}
\end{figure}

\textbf{[UC7]} - Create a Path in Automatic Mode
\begin{table}[H]
	\centering
	\label{tab:UC7}
	\begin{tabularx}{\textwidth}{|p{0.3\textwidth}|X|}
		\hline
		\textbf{Actor(s)}        & Logged-in User                                                                                                                                             \\ \hline
		\textbf{Entry Condition} & The actor is authenticated and wants to create a new path using GPS tracking.                                                                              \\ \hline
		\textbf{Event Flow}      & 1. The actor selects 'Create Path'.                                                                                                                        \\
		                         & 2. The system displays the available creation modes.                                                                                                       \\
		                         & 3. The actor chooses 'Automatic'.                                                                                                                          \\
		                         & 4. The system displays the input page for metadata configuration.                                                                                          \\
		                         & 5. The actor fills in basic metadata (e.g., name, visibility, description).                                                                                \\
		                         & 6. The system validates the provided metadata.                                                                                                             \\
		                         & 7. If the validation succeeds, the system confirms readiness and starts GPS tracking.                                                                      \\
		                         & 8. The system continuously collects geo-coordinates from the GPS while the actor is moving.                                                                \\
		                         & 9. Once tracking stops, the system validates the recorded GPS data.                                                                                        \\
		                         & 10. If validation succeeds, the system saves the new path.                                                                                                 \\
		                         & 11. The system displays a success message confirming the path creation.                                                                                    \\ \hline
		\textbf{Exit Condition}  & The automatically recorded path is successfully validated and stored in the database, associated with the actor’s account.                                 \\ \hline
		\textbf{Exceptions}      & If the data is invalid or incomplete, the system displays an error message and prevents saving until corrected.                                            \\
		                         & If GPS signal is lost or unstable, the system displays an error message and prompts the actor to retry.                                                    \\
		                         & If temporary connection or server error occurs, the system displays an error message and and allows retry later.                                           \\ \hline
		\textbf{Notes}           & The system continuously records and buffers GPS data locally during tracking. Only after successful validation is the path permanently stored server-side. \\ \hline
	\end{tabularx}
	\caption{Create a Path in Automatic Mode — Process Detail}
\end{table}

\begin{figure}[H]
	\centering
	\includegraphics[width=0.7\textwidth]{Images/SequenceDiagrams/automatically_create.png}
	\caption{Create a Path in Automatic Mode — Sequence Diagram}
	\label{fig:use_case_create_path_automatic}
\end{figure}


% --------------------------------------------------------------------------
%  Performance requirements % 
% --------------------------------------------------------------------------
\section{Performance Requirements}
\label{sec:performance_requirements}


% --------------------------------------------------------------------------
%  Design Constraints %
% --------------------------------------------------------------------------
\section{Design Constraints}
\label{sec:design_constraints}

\subsection{Standards Compliance}
\label{subsec:standards_compliance}

\subsection{Hardware Limitations}
\label{subsec:hardware_limitations}


% --------------------------------------------------------------------------
%  Software System Attributes %
% --------------------------------------------------------------------------
\section{Software System Attributes}
\label{sec:software_system_attributes}

\subsection{Reliability}
\label{subsec:reliability}

\subsection{Availability}
\label{subsec:availability}

\subsection{Security}
\label{subsec:security}

\subsection{Maintainability}
\label{subsec:maintainability}

\subsection{Portability}
\label{subsec:portability}