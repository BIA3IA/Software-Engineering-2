% --------------------------------------------------------------------------
%  Specific requirements % 
% --------------------------------------------------------------------------
\section{External Interface Requirements}
\label{sec:external_interface_requirements}

\subsection{User Interfaces}
\label{subsec:user_interfaces}

\subsection{Hardware Interfaces}
\label{subsec:hardware_interfaces}

\subsection{Software Interfaces}
\label{subsec:software_interfaces}

\subsection{Communication Interfaces}
\label{subsec:communication_interfaces}

% --------------------------------------------------------------------------
%  Functional requirements % 
% --------------------------------------------------------------------------
\section{Functional Requirements}
\label{sec:functional_requirements}

\subsection{Requirements}
\label{subsec:requirements}

\subsection{Use Case Diagrams}
\label{subsec:use_case_diagrams}
This subsection illustrates the main interactions between actors and the BBP system.
Each diagram focuses on a specific actor category, distinguishing what a guest user can do from the actions reserved to logged-in users.

\subsubsection{Authentication and Profile Management}
This diagram shows how users access and manage their personal area within the system.
Guest users can register to create an account and then log in to gain full access to BBP functionalities.
Both the Register and Log-in cases include the Validate Inputs use case, ensuring that all user-provided data meet format and consistency rules.
Once authenticated, users can log out, edit their personal information, and control the visibility of their profile data.

\begin{figure}[H]
	\centering
	\includegraphics[width=0.8\textwidth]{Images/Use_Case_Diagrams/register_login_uc.png}
	\caption{Authentication and Profile Management Use Case Diagram }
	\label{fig:use_case_authentication_and_profile_management}
\end{figure}

\subsubsection{Guest User Use Cases}
Guest users can freely access public information related to the available bike paths without being authenticated.
They can search for paths, browse their details, and check the overall condition of each segment.
When a user selects a path, the system retrieves and displays the corresponding aggregated condition.
Guests can also start and stop a trip, but the data is not stored or associated with a user account.

\begin{figure}[H]
	\centering
	\includegraphics[width=0.8\textwidth]{Images/Use_Case_Diagrams/guest_uc.png}
	\caption{Guest User Use Case Diagram}
	\label{fig:use_case_guest_user}
\end{figure}

\subsubsection{Logged-in User Use Cases}
Logged-in users have access to all interactive and community features of the BBP platform.
They can create new paths manually or automatically, by recording their route or by manually adding segments.
Each new path can be saved only after passing the Validate Inputs step and may optionally be made public.
Users can also delete paths they have created.
During trip recording, data from sensors are analyzed to automatically detect anomalies; these detections trigger the Confirm/Reject Detection extension, allowing the user to verify their validity.
Users can also report path conditions manually and review other users' reports.
Once a trip is stopped, a summary enriched with weather data is generated.
Finally, users can view their activity history and per-trip or overall statistics.

\begin{figure}[H]
	\centering
	\includegraphics[width=0.9\textwidth]{Images/Use_Case_Diagrams/loggedin_uc.png}
	\caption{Logged-in User Use Case Diagram}
	\label{fig:use_case_logged_in_user}
\end{figure}

\subsubsection{Use Cases}
\label{subsubsec:use_cases}

\textbf{[UC1]} - User Registration
\begin{table}[H]
	\centering
	\label{tab:UC1}
	\begin{tabularx}{\textwidth}{|p{0.3\textwidth}|X|}
		\hline
		\textbf{Actor(s)}        & none    \\ \hline
		\textbf{Entry Condition} & none    \\ \hline
		\textbf{Event Flow}      & 1. none \\
		                         & 2. none \\
		                         & 3. none \\ \hline
		\textbf{Exit Condition}  & none    \\ \hline
		\textbf{Exceptions}      & none    \\ \hline
		\textbf{Notes}           & none    \\ \hline
	\end{tabularx}
	\caption{User Registration Process Detail}
\end{table}

% --------------------------------------------------------------------------
%  Performance requirements % 
% --------------------------------------------------------------------------
\section{Performance Requirements}
\label{sec:performance_requirements}


% --------------------------------------------------------------------------
%  Design Constraints %
% --------------------------------------------------------------------------
\section{Design Constraints}
\label{sec:design_constraints}

\subsection{Standards Compliance}
\label{subsec:standards_compliance}

\subsection{Hardware Limitations}
\label{subsec:hardware_limitations}


% --------------------------------------------------------------------------
%  Software System Attributes %
% --------------------------------------------------------------------------
\section{Software System Attributes}
\label{sec:software_system_attributes}

\subsection{Reliability}
\label{subsec:reliability}

\subsection{Availability}
\label{subsec:availability}

\subsection{Security}
\label{subsec:security}

\subsection{Maintainability}
\label{subsec:maintainability}

\subsection{Portability}
\label{subsec:portability}