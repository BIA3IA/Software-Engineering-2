% --------------------------------------------------------------------------
% Reference Documents % 
% --------------------------------------------------------------------------

\section{Reference Documents}
\label{sec:refdocs}

The preparation of this document was supported by the following reference materials:

\begin{itemize}
	\item IEEE Standard for Software Requirement Specifications \cite{ieee};
	\item Assignment specification for the RASD and DD of the Software Engineering II course,
	      held by professors Matteo Rossi, Elisabetta Di Nitto, and Matteo Camilli at the
	      Politecnico di Milano, A.Y. 2025/2026 \cite{rasd};
	\item Slides of the Software Engineering 2 course available on WeBeep \cite{slides};
	\item The paper *Deriving Specifications from Requirements: An Example* by Jackson and Zave \cite{paper}.
\end{itemize}

% --------------------------------------------------------------------------
% Software Used
% --------------------------------------------------------------------------

\section{Software Used}
\label{sec:software}

The following software tools have been used during the development of this project:

\begin{itemize}
	\item \textbf{Visual Studio Code}: editing of source code and documentation (LaTeX), with project-wide search and formatting support \cite{vscode}.
	\item \textbf{LaTeX}: typesetting system used to produce the final RASD document in a consistent format \cite{latex}.
	\item \textbf{Git}: version control used to track changes and support collaborative development \cite{git}.
	\item \textbf{GitHub}: remote repository hosting and collaboration platform used for versioning, reviews, and issue tracking \cite{github}.
	\item \textbf{Lucidchart}: creation of UML diagrams (use case diagrams, state diagrams, domain class diagram) \cite{lucidchart}.
	\item \textbf{Alloy }: execution of the Alloy model, trace generation, and assertion checking \cite{alloy6}.
\end{itemize}

% --------------------------------------------------------------------------
% AI Tools Usage
% --------------------------------------------------------------------------

\section{Use of AI Tools}
\label{sec:ai}

AI tools were part of our workflow in a way similar to other software we used during the
project. Rather than producing content autonomously, they helped us reason about how to
present certain sections, reorganize ideas, and make the document more coherent.

We used AI mainly while drafting the text, for example to compare alternative ways of
explaining scenarios, reorganize long paragraphs, or check whether the wording of some
requirements could lead to misinterpretations. In several cases, discussing a passage
with an AI assistant helped us clarify our own understanding of the underlying concept
before writing the final version.

\subsection{Tools Used}
The AI tools employed during the project were:
\begin{itemize}
	\item Gemini
	\item ChatGPT
\end{itemize}

\subsection{Typical Prompts}
AI tools were queried using prompts such as:
\begin{itemize}
	\item "Suggest a clearer way to express this requirement without changing its meaning."
	\item "Can this sentence be misinterpreted? If yes, propose an alternative phrasing."
	\item "Provide a more concise version of this paragraph while keeping the content intact."
	\item "Format the following text in LaTeX"
	\item "Write the basic layout of a LaTeX table with a header row."
	\item "Help debug formatting or build issues related to VS Code or LaTeX"
	\item "Apply the Alloy LaTeX style to this code snippet."
	\item "Given the predicate: predadd[b, bpost: Book, n: Name, a: Addr] \{ bpost.addr= b.addr+ n-> a \}. What is the meaning of "->"?"
	\item "What is the difference between doing b.addr instead of addr[b]"
\end{itemize}

\subsection{Input Provided}
The input given to AI tools consisted mainly of:
\begin{itemize}
	\item Early drafts of paragraphs.
	\item Short text fragments requiring clarity checks.
	\item Sections with repeated structure where consistent wording was needed.
\end{itemize}

\subsection{Constraints Applied}
When using AI tools, the following constraints were strictly enforced:
\begin{itemize}
	\item Preserve the intended meaning of the original text.
	\item Avoid introducing new requirements or assumptions.
	\item Maintain terminology aligned with the definitions provided in this document.
\end{itemize}

\subsection{Outputs Obtained}
The interaction with AI tools resulted in:
\begin{itemize}
	\item Clearer or more concise formulations of existing statements.
	\item Identification of potentially ambiguous sentences.
	\item Terminology suggestions to improve internal coherence.
	\item LaTeX formatting assistance for tables and code snippets.
\end{itemize}

\subsection{Refinement Process}
All AI-generated outputs were subject to a manual refinement process that included:
\begin{itemize}
	\item Critical review of all suggestions.
	\item Verification against the original intent to avoid unintended changes.
	\item Manual integration to ensure consistency with the overall writing style.
	\item Alignment checks with established terminology and definitions.
\end{itemize}
