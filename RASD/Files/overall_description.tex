% --------------------------------------------------------------------------
% Product Perspective % 
% --------------------------------------------------------------------------
\section{Product Perspective}
\label{sec:product_perspective}
This section describes how the Best Bike Paths (BBP) system fits into its context.

\subsection{Scenarios}
\label{sec:scenarios}
Here we describe representative situations that illustrate how users will interact with the system in the real world.  

\begin{itemize}
    \item \textbf{Some Scenario}:

    Some Explication.
    \item \textbf{Another Scenario}:

    Another Explication.
\end{itemize}

\subsection{Domain Class Diagram}
\label{sec:domain_class_diagram}

This subsection contains the class diagram representing the main concepts of the application domain.  

\subsection{State Diagrams}
\label{sec:state_diagrams}

This subsection includes state or activity diagrams that describe the dynamic behavior of key domain entities or processes.  

\begin{itemize}
    \item \textbf{Some State Diagram}:

    Some Explication.
    \item \textbf{Another State Diagram}:

    Another Explication.
\end{itemize}

% --------------------------------------------------------------------------
% Product Functions % 
% --------------------------------------------------------------------------
\section{Product functions}
\label{sec:product_functions}

This section provides an overview of the main system functionalities as perceived by users.  

\begin{itemize}
    \item \textbf{Some Function}:

    Some Explication.
    \item \textbf{Another Function}:

    Another Explication.
\end{itemize}

% --------------------------------------------------------------------------
% User Characteristics % 
% --------------------------------------------------------------------------
\section{User Characteristics}
\label{sec:user_characteristics}

The Best Bike Paths (BBP) platform is designed to support two main user categories: \textbf{unregistered users} and \textbf{registered users}.  
Each group has distinct goals, permissions, and levels of interaction with the system.

\begin{itemize}
    \item \textbf{Unregistered users}:

    This user interacts with the system without performing any authentication.
    They can explore existing bike paths, view global information such as distance and ranking, and consult public reports about obstacles or road conditions.  
    Their interaction with the system is limited to visualization and browsing; they are not allowed to create, edit, or report any paths or obstacles.  
   
    \item \textbf{Registered users}:

    Registered users have full access to the system’s functionalities after creating an account and logging in.
    They can create new bike paths using GPS tracking, or by manually selecting points on a map interface.  
    They can also manually or automatically submit obstacle reports, and validate or confirm existing ones shared by other users. 
    If the automatic mode is used, the user can confirm the detected obstacles before submission.
    Furthermore, registered users can record trips, which are then stored within their personal profiles, allowing them to view historical data and general and per-trip statistics such as distance covered, average speed, and total time spent cycling.
\end{itemize}

% --------------------------------------------------------------------------
% Assumption, dependencies and constraints % 
% --------------------------------------------------------------------------
\section{Assumptions, dependencies and constraints}
\label{sec:assumptions_dependencies_constraints}

This section lists external factors that influence the system but are beyond its direct control.  

\subsection{Dependencies}
\label{sec:dependencies}

This subsection lists the external systems or services on which BBP depends.

\subsection{Constraints}
\label{sec:constraints}

This subsection specifies any technical, legal, or environmental restriction that affects the system design.

\subsection{Domain Assumptions}
\label{sec:domain_assumptions}

This subsection lists the assumptions about the real world that are taken as true for the requirements to hold.

\begin{itemize}
    \item[\textbf{[D1]}] The system has access to a geographical map service and geocoding capabilities to process origin/destination and render paths. 

    \item[\textbf{[D2]}] The manual input provided by the Registered users is accurate and complete as intended for a path entry.
    
    \item[\textbf{[D3]}] The external meteorological service is operational and accessible by the system. 

    \item [\textbf{[D4]}] The Registered users' mobile devices are equipped with GPS, and they have granted the BBP application location permissions and motion sensor access.

    \item [\textbf{[D5]}] The mobile device sensors (accelerometer, gyroscope) provide sufficiently reliable data to suggest the presence of obstacles.

    \item[\textbf{[D6]}] A \emph{bike path} is considered either a dedicated bicycle track or a low-traffic road where motor-vehicle speed limits are compatible with average cycling speed.
    
    \item[\textbf{[D7]}] Notification or Pop ups sent by the BBP system reaches the users within 5 min.
\end{itemize}