% --------------------------------------------------------------------------
% Product Perspective % 
% --------------------------------------------------------------------------
\section{Product Perspective}
\label{sec:product_perspective}
The Best Bike Paths (BBP) system operates within an environment that includes users, mobile devices, and external services. The following sections describe its main interactions and contextual role.

\subsection{Scenarios}
\label{sec:scenarios}
Representative scenarios illustrate how different types of users interact with BBP in realistic situations, showing the system’s main functionalities and objectives.

\begin{enumerate}

	\item \textbf{A user wants to use the system}:

	      Giuseppe is a former professional cyclist who wants to start cycling again to stay fit. Among the various options available, he decides to use Best Bike Paths (BBP) to track his cycling activities and discover new routes.
	      He chooses BBP because it offers cyclist-oriented features such as creating personalized paths and sharing the current conditions of bike tracks with other users. \newline
	      Once he opens the BBP app, Giuseppe realizes that he can already use some functionalities of the system without creating an account, such as searching for bike paths and receiving suggestions based on other users’ rankings.
	      However, he soon notices that he cannot record his own cycling activities without registering. \newline
	      Therefore, he decides to create an account to take full advantage of all BBP’s features.
	      The registration process is simple and quick, requiring only basic information such as name, email, and password.
	      After successfully registering, Giuseppe gains access to the full set of functionalities, including creating personalized routes, recording his cycling sessions, and sharing information about the condition of bike paths with the BBP community.

	\item \textbf{A user starts to browse and create paths}:

	      Irene accesses the BBP app to plan her next cycling trip.
	      From the main page, she specifies a starting point and a destination. BBP displays on a map all available bike paths connecting the two points.
	      Each path is shown with its current condition based on feedback provided by other users. The paths are ranked according to an overall score computed by the system,
	      taking into account their condition and effectiveness in connecting the selected origin and destination. \newline
	      After reviewing the suggested routes, Irene realizes that none of them fully meet her preferences.
	      Since she is logged into her BBP account, she decides to create a new custom path. \newline
	      BBP offers her either a manual or an automatic mode for path creation. She chooses the manual mode, where she can specify the names of the street segments composing her desired route and provide initial information about their condition.
	      Irene carefully enters the street names that form her preferred path, adds a brief description, and saves it to her personal collection. \newline
	      The newly created path is now available in Irene's account and can be selected for future trips. If she decides to make it publishable later, other users will also be able to discover and use it.

	\item \textbf{A logged-in user starts a trip}:

	      Cesare decides to start a new bike trip. After selecting a suggested bike path, BBP asks him which data acquisition mode he prefers: manual or automatic.
	      Cesare chooses the automatic mode and presses the Start button to begin his ride. \newline
	      As he starts, BBP begins collecting data from the sensors available on his mobile device.
	      The GPS continuously updates his position, allowing the system to reconstruct the exact path he follows.
	      At the same time, the accelerometer and gyroscope detect small irregular movements that may indicate the presence of bumps or potholes along the road. \newline
	      On the screen, BBP dynamically shows his current position.
	      The system also enriches Cesare's trip data with weather conditions such as temperature, humidity, and wind speed retrieved from an external meteorological service.\newline
	      Once he reaches his destination BBP processes the collected data and generates a detailed summary, including total distance, duration, average speed, and other performance metrics that Cesare can review later in his personal activity history.

	\item \textbf{A logged-in user browse his cycling activities}:

	      Luca, a logged-in user of BBP, opens the app to view his previous trips.
	      From the main page, he accesses the "My Activities' section", where BBP displays a list of his recorded trips over time.\newline
	      Each activity is accompanied by summary information such as the date, distance traveled, duration, and primary location.
	      Luca can sort trips by date or distance and scrolling through the list, he recognizes some of his favorite routes and notices that BBP automatically updates aggregate statistics such as total number of trips, kilometers traveled, and weekly average giving him a general overview of his progress over time. \newline
	      After identifying a recent trip he's interested in analyzing in more detail, Luca decides to open it to view the details.

	\item \textbf{A logged-in user access his trip and per trip statistics}:

	      Pietro selects his latest trip from his activity list. BBP opens the route details page, displaying the entire route on a map.
	      In the same page, the app presents a complete summary of the recorded data: total distance, duration, average and maximum speed.
	      Thanks to integration with the device's sensors, Pietro can also view the weather conditions he encountered during the trip, including average temperature and wind direction.\newline
	      Scrolling through the map, Pietro notices some points where the accelerometer detected abnormal movement.\newline
	      Satisfied with his analysis, Pietro closes the trip tab and returns to the general overview of his activities.

	\item \textbf{A logged-in user wants to insert information about a bike path condition}:

	      Giulia is cycling along a local bike path using the manual reporting mode.
	      During her ride, she notices that a section of the route is partially blocked by some fallen branches.
	      Since she wants to help other cyclists avoid potential issues, she decides to report the current condition of the path to the BBP community.\newline
	      From the app menu, Giulia selects Report Path Condition and once the interactive map is opened, selects the affected segment, and writes a short description of the obstacle, choosing “requires maintenance” as the current status of that path.
	      After confirming the report, BBP stores the information and sends it for aggregation with other users’ data about the same route.

	\item \textbf{A logged-in user wants to confirm the condition of a bike path}:

	      While cycling through a familiar route, Marco receives a notification with a small pop-up alert which states that there is a possible obstacle ahead reported by another user.
	      Marco slows down and observes the road segment. He notices that the obstacle is indeed still present, so he decides to confirm the report.
	      Directly from the notification, he presses Confirm, and BBP registers his feedback as an additional confirmation for that path condition. \newline
	      Later, while continuing his ride, Marco receives another notification about a reported bump on the same road, but this time the road surface looks perfectly fine.
	      He therefore selects Reject, indicating that the problem has been resolved or no longer exists. \newline
	      BBP updates the internal data accordingly. Each confirmation or rejection contributes to refining the aggregated information that determines the official status of that bike path.
	      This way, Marco’s participation, along with that of other cyclists, helps the system keep path conditions accurate and up to date for the entire community.

\end{enumerate}

\subsection{Domain Class Diagram}
\label{sec:domain_class_diagram}

\begin{figure}[H]
	\centering
	\includegraphics[width=\textwidth]{Images/Domain_Class_Diagram/diagram.png}
	\caption{Domain Class Diagram}
\end{figure}

The Domain Class Diagram for the Best Bike Paths (BBP) system establishes a comprehensive framework for managing cycling infrastructure and user activity. At the foundation of the system is the User entity, which tracks a unique identifier and a current position via a GeoPoint. This entity branches into two distinct roles: the Guest and the LoggedInUser. \newline 
A Guest user has the flexibility to search for existing paths, create new ones, and initiate trips, while also having the option to register for a permanent account. Once registered, the LoggedInUser is identified by a username, email, and password, allowing them to own specific trip records and maintain a long-term profile within the system.\newline 
The structural core of the application is the Path, which includes its description, public visibility, and a status enumeration that tracks the path's condition. A Path is built through a composition of one or more PathSegment entities. These segments link the origin and destination GeoPoints; each segment contains an array of specific coordinates and maintains a recursive reference to the next segment in the sequence, creating a traversable chain. These geographical structures are visualized through a MapService.\newline 
Activity within the system is captured through the Trip entity, which records the temporal and physical details of a specific journey. A Trip stores attributes such as start and end times, total distance, and average speed. \newline 
It is important to note that while the Trip contains its own specific performance data, it is also aggregated into the OverallStatistics entity owned by a LoggedInUser. This distinction ensures that individual trip metrics are preserved while contributing to the user's cumulative performance history, such as their lifetime average speed and total distance.\newline 
Furthermore, trips are contextualized by WeatherInfo provided by a WeatherProvider which logs environmental factors like temperature and humidity at the time of the ride.\newline 
Crowd-sourced data and safety are managed through a robust reporting system. LoggedInUsers can generate Report entities to identify various obstacles, such as potholes or flooding, categorized by an ObstacleType. These reports can be created manually or triggered automatically via an AcquisitionMode. To facilitate automatic detection, the system integrates with hardware Sensors, including accelerometers and gyroscopes, which suggest potential reports based on physical movement. Each report undergoes a validation process reflected in its status, moving from creation to confirmation or rejection. All temporal data throughout these processes is standardized using the DateTime entity as an epoch-time integer, ensuring consistency across all system logs.


\subsection{State Diagrams}
\label{sec:state_diagrams}
The UML State Diagrams shown below describe the system’s main behavioral processes, focusing on the most relevant user interactions. These include starting a trip, creating paths (manually and automatically), and reporting obstacles (manually and automatically).

\subsubsection{Starting a trip}

The following diagram illustrates the possible states involved when a user starts a trip through the system.
This process begins when the user opens the app and decides to start a new trip by inserting the desired start and end points. Once these points are defined, the system activates the Path Suggestion System, which is responsible for generating and ranking candidate bike paths that best fit the user’s request.

Inside the Path Suggestion System, the process goes through several internal states. First, the system enters the Waiting for Data state, where it collects environmental and contextual information (such as existing paths and reports). When all required data have been gathered, the system transitions to the Evaluating state to verify their validity and consistency.
If the data are valid, the system proceeds to Scoring, where each segment is analyzed and assigned a score based on its quality and reliability. Afterward, in the Merging and Building Candidates states, segments are merged and combined to form complete candidate paths. Finally, in the Ranking state, the system orders these candidates according to computed scores.

If an error occurs at any stage (for example, insufficient data, invalid inputs, or scoring issues), the process moves to a failure path and returns to the Idle state. Otherwise, once the ranking is completed, the system exits the Path Suggestion System, displaying the generated recommendations to the user.
At this point, the user enters the Browsing state, where they can inspect all suggested paths. If a suitable path is found, the user can select it, triggering the transition to the next phase of the trip initialization. If no suitable option is available, the system returns to the idle state, ready for a new search request.

\begin{figure}[H]
	\centering
	\includegraphics[width=0.8\textwidth]{Images/State_Diagrams/starting_trip.png}
	\caption{Starting Trip State Diagram}
\end{figure}

\subsubsection{Manually creating a new path}

The following diagram illustrates the states related to the manual creation of a new path by a logged-in user.
A logged-in user can start the manual creation process by selecting the “Create Path” option. Once this action is triggered, the system leaves the idle state and enters the Editing Metadata state, where the user provides general information about the path, such as its name, description, visibility, and other optional details.

After the metadata are confirmed and saved, the process continues in the Adding Segments state. In this phase, the user manually draws or selects the path segments on the map to define the complete route. When all segments have been added, the user proceeds to the Validating Path state, where the system checks the consistency and continuity of the selected route.
If the validation succeeds, the process transitions to the Saving Path state, during which the system stores the new path in the database. In case of a validation failure or a save error, the system moves to an error path and returns to the idle state, allowing the user to retry or discard the creation.

If the saving process completes successfully, the system reaches the final state, confirming that the new path has been correctly created and stored.
Throughout the process, the user can cancel the creation, which immediately returns the system to the idle state.

\begin{figure}[H]
	\centering
	\includegraphics[width=0.8\textwidth]{Images/State_Diagrams/manual_path_creation.png}
	\caption{Manual Path Creation State Diagram}
\end{figure}

\subsubsection{Automatically creating a new path}

The following diagram illustrates the states involved in the automatic creation of a new path by a logged-in user using GPS tracking.
After accessing the application, the user can start the automatic creation process by selecting the “Automatic Mode” option. Once this action is triggered, the system leaves the idle state and enters the Editing Metadata State, where the user can provide information about the path. After confirming the metadata, the system transitions to the Waiting GPS Lock state, where it requests location permissions and attempts to establish a stable GPS connection.
If the GPS signal is successfully acquired, the system transitions to the Recording state, during which the user’s movements are continuously tracked and converted into path segments.

While recording, the user can manually stop the tracking (userStop), or the process can automatically stop if a long period of inactivity is detected.
If the GPS connection is lost for a prolonged time or the user denies permission, the process transitions to the failed path, ending unsuccessfully and returning to the idle state.
When the user stops the recording or the system detects the end of movement, the process enters the Validating Path state, where the recorded route is checked for continuity, consistency, and potential GPS anomalies.

If the validation succeeds, the system moves to the Saving Path state, where the path data are stored in the database. Any validation or saving errors cause a transition to a failure path, returning to the idle state.
Once the path has been successfully saved, the process reaches its final state, confirming that the automatically tracked path has been correctly recorded and stored.

\begin{figure}[H]
	\centering
	\includegraphics[width=0.6\textwidth]{Images/State_Diagrams/automatic_path_creation.png}
	\caption{Automatic Path Creation State Diagram}
\end{figure}

\subsubsection{Manually inserting a report about an obstacle}

The following diagram illustrates the states related to the manual reporting process, which allows a logged-in user to create a report manually.
Starting from the Idle state, the user initiates the manual report creation process by selecting the appropriate option from the interface. Once the process begins, the system moves to the Editing Metadata state, where the user can enter or modify relevant information, such as the description, and obstacle type.

After the metadata are saved, the process continues in the Waiting for GPS state. In this phase, the system attempts to obtain the user’s current position to associate the report with a specific geographic location.
If the GPS lock is successfully acquired, the process transitions to the Saving Report state, where the report is stored in the database.

If any error occurs during the GPS acquisition or the saving phase (GPSError, saveError), or if the user decides to cancel the operation (reportCanceled), the process terminates prematurely and returns to the Idle state.
When the report is successfully saved, the system reaches the final state, confirming that the manual report has been correctly created and recorded.

\begin{figure}[H]
	\centering
	\includegraphics[width=0.8\textwidth]{Images/State_Diagrams/manual_report.png}
	\caption{Manual Report State Diagram}
\end{figure}

\subsubsection{Automatically detecting and reporting an obstacle}

The following diagram represents the states involved in the automatic reporting process, where a logged-in user can generate a report automatically when a relevant event or sensor input is detected.
The process starts from the Idle state. When the system detects a sensor signal or an anomaly (SensorDetection), it transitions to the Confirming Report state, where the user is asked to confirm whether the detected event should be reported.

If the user confirms, the process continues in the Editing Metadata state, allowing the user to optionally edit or add information related to the report, such as a short description or obstacle type. Once the metadata are saved, the system moves to the Waiting for GPS state, where it tries to obtain the current position to associate it with the report.

If the GPS signal becomes active, the system transitions to the Saving Report state, where the report data are finalized and stored in the database.
If at any point the user cancels the report (reportCanceled), or the GPS fails to provide a valid location (GPSError), the process terminates prematurely and returns to the idle state.

Finally, if the saving procedure completes successfully (saveComplete), the system reaches its final state, confirming that the automatic report has been correctly generated and saved.

\begin{figure}[H]
	\centering
	\includegraphics[width=0.8\textwidth]{Images/State_Diagrams/automatic_report.png}
	\caption{Automatic Report State Diagram}
\end{figure}

% --------------------------------------------------------------------------
% Product Functions % 
% --------------------------------------------------------------------------
\section{Product functions}
\label{sec:product_functions}

The following points outline the main functionalities offered by the system from the user’s perspective.

\begin{itemize}
	\item \textbf{User Registration and Login}:

	      The system allows new users to register and existing users to log in through a dedicated interface accessible from the main page of BBP.
	      Registration requires only basic personal information such as name, email, and password.
	      Once logged-in, users gain access to additional functionalities that are not available to guest users, including the ability to record and track personal cycling trips, create personalized bike paths, view personal activity history and performance statistics, and contribute information to the community by reporting or confirming bike path conditions.
	      Guest Users can browse available bike paths, search for routes between two locations, and receive route suggestions based on community rankings, but they cannot record trips, create new paths, or contribute reports.

	\item \textbf{Bike Path Search and Visualization}:

	      Any user can search for bike paths by specifying an origin and a destination.
	      BBP displays on a map all available routes connecting the two selected points. Each path has information about its current condition, derived from aggregated feedback provided by logged-in users.
	      The displayed paths are ranked according to an overall score computed by the system. This score takes into account both the physical condition of the path (e.g., optimal, medium, requires maintenance) and its effectiveness in connecting the specified origin and destination.
	      Users can explore the visualized paths, view their details, and select the most suitable route for their needs. However, only logged-in users can start recording a trip or contribute new information about the selected paths.

	\item \textbf{Bike Path Creation}:

	      Logged-in users can create new custom bike paths using two different approaches.
	      In manual mode, the user specifies the names of the street segments that compose the desired route and provides information about the characteristics of each segment.
	      This allows cyclists to define personalized routes that may not yet be present in the system or that better match their preferences.
	      Alternatively, in automatic mode, the system dynamically creates a new path by recording GPS data while the user follows their desired route during an actual trip.
	      As the user bikes, BBP collects positional information to reconstruct the exact path in real time. This approach enables users to create paths simply by cycling along them, without needing to manually specify street names or segments beforehand.
	      Regardless of the creation method, the newly created path can remain private to the user or be made publishable for the entire community.

	\item \textbf{Trip Recording and Data Acquisition}:

	      A logged-in user starts a trip by choosing a route on the map and selecting the data acquisition mode: manual (distance, time, and average speed only, without automatic detections) or automatic.
	      In automatic mode, BBP uses GPS, an accelerometer, and a gyroscope to track the route and identify potential irregularities in the road surface.
	      Pressing Start begins the trip: the app displays the user's real-time location and provides navigation to the destination.
	      The activity stops automatically upon arrival or manually with Stop, then the system processes the data and generates a summary.
	      If recording is automatic, BBP sends pop-ups to confirm or correct any detected anomalies before the data is made available to the community.

	\item \textbf{Cycling Activity History and Aggregate Statistics}:

	      Logged-in users can access the "My Activities' section" to review their personal cycling history.
	      This section displays both a chronological list of all recorded trips and aggregate statistics summarizing the users overall performance over time.
	      The aggregate statistics are presented on the same page as the trip list and include metrics such as total number of trips, total kilometers traveled, average speed across all activities, and weekly or monthly riding frequency.
	      Each trip entry in the list includes summary information such as the date, distance covered, duration, and primary route followed.
	      Users can sort the list by date or distance to easily find specific activities.
	      The trip list and aggregate statistics remain securely linked to the users account, allowing them to track their performance and monitor improvements over time.

	\item \textbf{Trip Specific Statistics and Environmental Data}:

	      Logged-in users can select any trip from their activity list to access detailed trip specific statistics.
	      For each individual trip, BBP provides total distance, duration, average and maximum speed.
	      The trip page also displays a map showing the complete route followed by the user. If environmental data were available during the ride either from the users external device or from an external meteorological service they are presented alongside the trip statistics.
	      These environmental data may include average temperature, humidity levels, and wind speed and direction.
	      Additionally, either if the user recorded the trip in automatic mode or manual mode, the map highlights points where the accelerometer detected abnormal movements, indicating possible obstacles or rough surfaces encountered along the route or points where the user manually reported issues.

	\item \textbf{Manual Reporting of Bike Path Conditions}:

	      BBP enables logged-in users to manually report the condition of any bike path they encounter.
	      This feature helps keep route information accurate, especially when users spot issues like obstacles or closures.
	      By selecting Report Path Condition in the app, users can mark the affected segment on a map, describe the issue, and assign a status such as “optimal,” “medium,” “requires maintenance,” or “closed.”

	\item \textbf{Real Time Confirmation and Rejection of Reported Path Conditions}:

	      During a ride, logged-in users may receive pop-up alerts about nearby issues reported by others, such as “Possible pothole ahead” or “Path requires maintenance.”
	      As they approach the location, users can confirm that the issue persists or reject it if the path is clear.
	      Their feedback is instantly recorded by BBP and influences the reliability of reports-confirmed ones gain credibility, while frequently rejected ones are downgraded or removed.

	\item \textbf{Trip Confirmation of Automatically Detected Anomalies}:

	      When a logged-in user records a trip in automatic mode, BBP uses the device’s sensors to detect irregularities such as potholes or obstacles.
	      To prevent false positives, detected anomalies are not immediately published. Instead, the user is shown a confirmation screen to review and validate or correct each detection.
	      Only confirmed data becomes publishable, ensuring accuracy and reliability of shared path information.

	\item \textbf{Aggregation and Merging of Bike Path Condition Reports}:

	      BBP combines multiple user reports about the same bike path into a single, updated condition overview.
	      The system prioritizes recent and consistent reports, giving more weight to those confirmed by several users.
	      In case of conflicting information, BBP determines the path’s status based on the most recent and widely supported data, ensuring an up-to-date representation of path conditions.

\end{itemize}

% --------------------------------------------------------------------------
% User Characteristics % 
% --------------------------------------------------------------------------
\section{User Characteristics}
\label{sec:user_characteristics}

The Best Bike Paths (BBP) platform is designed to support two main user categories: \textbf{guest users} and \textbf{logged-in users}.
Each group has distinct goals, permissions, and levels of interaction with the system.

\begin{itemize}
	\item \textbf{Guest Users}:

	      Guest users interact with the system without performing any authentication.
	      They may either be users who have not created an account, or users who have an existing account but are currently logged out.
	      Guest users can explore existing bike paths, view global information such as distance, rankings, and general statistics, and consult public reports about obstacles or road conditions.
	      Their interaction with the system is limited to browsing and visualization; they are not allowed to create, edit, or report any paths or obstacles.

	\item \textbf{Logged-in users}:

	      Logged-in users have full access to the system’s functionalities after creating an account and logging in.
	      They can create new bike paths using GPS tracking, or by manually selecting points on a map interface.
	      They can also manually or automatically submit obstacle reports, and validate or confirm existing ones shared by other users.
	      If the automatic mode is used, the user can confirm the detected obstacles before submission.
	      Furthermore, logged-in users can record trips, which are then stored within their personal profiles, allowing them to view historical data and general and per-trip statistics such as distance covered, average speed, and total time spent cycling.
\end{itemize}

% --------------------------------------------------------------------------
% Assumption, dependencies and constraints % 
% --------------------------------------------------------------------------
\section{Assumptions, dependencies and constraints}
\label{sec:assumptions_dependencies_constraints}

\subsection{Dependencies}
\label{sec:dependencies}
The BBP application depends on several external systems and services to provide its core functionalities:

\begin{itemize}
	\item The system requires a connection with the remote database through the backend APIs to retrieve, store, and update user and path data.
	\item The application depends on a reliable Internet connection to access the backend, synchronize trip data, and retrieve updated path information.
	\item A geographic map service is required to display maps, compute routes, and process geocoding and reverse-geocoding requests.
	\item External meteorological APIs are used to collect environmental data (e.g., temperature, humidity, wind speed) associated with trips.
\end{itemize}

\subsection{Constraints}
\label{sec:constraints}

The design and implementation of the BBP system are subject to the following constraints:

\begin{itemize}
	\item The system must fully comply with the principles and requirements established by the GDPR (General Data Protection Regulation, EU 2016/679), ensuring that all collected and processed data respect user privacy, security, and transparency standards.
	\item The application must conform to accessibility and usability guidelines defined by major mobile platforms (Android and iOS).
	\item Any third-party APIs used for maps, weather, or notifications must be employed in compliance with their respective terms of service and usage quotas.
\end{itemize}

\subsection{Domain Assumptions}
\label{sec:domain_assumptions}
This subsection lists the assumptions about the real world that are taken as true for the requirements to hold.

\begin{itemize}
	\item[\textbf{[D1]}] The system has access to a geographic map service and geocoding capabilities to process origin/destination and render paths.
	\item[\textbf{[D2]}] External meteorological services are reachable and operational with acceptable latency during normal operation.
	\item[\textbf{[D3]}] Users’ mobile devices are equipped with GPS and have granted the BBP app access to location and motion sensors.
	\item[\textbf{[D4]}] Motion sensors (accelerometer, gyroscope) and GPS provide data of sufficient accuracy and reliability for obstacle detection.
	\item[\textbf{[D5]}] Manual input provided by logged-in users is assumed to be truthful, accurate, and sufficiently complete for a path/report entry.
	\item[\textbf{[D6]}] A bike path is either a dedicated bicycle track or a low-traffic road where motor-vehicle speed limits are compatible with average cycling speed.
	\item[\textbf{[D7]}] Notifications or pop-ups sent by the BBP system reach users within 5 minutes.
	\item[\textbf{[D8]}] Reported conditions may become obsolete over time; data “freshness” decays and is considered by the aggregation/ranking logic.
\end{itemize}