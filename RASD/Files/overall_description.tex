% --------------------------------------------------------------------------
% Product Perspective % 
% --------------------------------------------------------------------------
\section{Product Perspective}
\label{sec:product_perspective}
This section describes how the Best Bike Paths (BBP) system fits into its context.

\subsection{Scenarios}
\label{sec:scenarios}
Here we describe representative situations that illustrate how users will interact with the system in the real world.

\begin{itemize}
	\item \textbf{Some Scenario}:

	      Some Explication.
	\item \textbf{Another Scenario}:

	      Another Explication.
\end{itemize}

\subsection{Domain Class Diagram}
\label{sec:domain_class_diagram}

This subsection contains the class diagram representing the main concepts of the application domain.

\subsection{State Diagrams}
\label{sec:state_diagrams}

In this section, the UML State Diagrams describing the main behavioral processes of the system are presented.
These diagrams model the possible states and transitions involved when a user interacts with the application during the starting of a trip, the creation of new paths (both manually and automatically), and the reporting of obstacles (both manually and automatically).
These five scenarios have been analyzed because they represent the core functionalities of the entire system.

\textbf{Starting a trip}:

The following diagram illustrates the possible states involved when a user starts a trip through the system.
This process begins when the user opens the application and decides to start a new trip by inserting the desired start and end points. Once these points are defined, the system activates the Path Suggestion System, which is responsible for generating and ranking candidate bike paths that best fit the user’s request.

Inside the Path Suggestion System, the process goes through several internal states. First, the system enters the Waiting for Data state, where it collects environmental and contextual information (such as existing paths and reports). When all required data have been gathered, the system transitions to the Evaluating state to verify their validity and consistency.
If the data are valid, the system proceeds to Scoring, where each segment is analyzed and assigned a score based on its quality and reliability. Afterward, in the Merging and Building Candidates states, segments are merged and combined to form complete candidate paths. Finally, in the Ranking state, the system orders these candidates according to computed scores.

If an error occurs at any stage (for example, insufficient data, invalid inputs, or scoring issues), the process moves to a failure path and returns to the Idle state. Otherwise, once the ranking is completed, the system exits the Path Suggestion System, displaying the generated recommendations to the user.
At this point, the user enters the Browsing state, where they can inspect all suggested paths. If a suitable path is found, the user can select it, triggering the transition to the next phase of the trip initialization. If no suitable option is available, the system returns to the idle state, ready for a new search request.

\begin{figure}[H]
	\centering
	\includegraphics[width=\textwidth]{Images/State_Diagrams/starting_trip.png}
	\caption{Starting Trip State Diagram}
\end{figure}

\textbf{Manually creating a new path}:

The following diagram illustrates the states related to the manual creation of a new path by a registered user.
A registered user can start the manual creation process by selecting the “Create Path” option. Once this action is triggered, the system leaves the idle state and enters the Editing Metadata state, where the user provides general information about the path, such as its name, description, visibility, and other optional details.

After the metadata are confirmed and saved, the process continues in the Adding Segments state. In this phase, the user manually draws or selects the path segments on the map to define the complete route. When all segments have been added, the user proceeds to the Validating Path state, where the system checks the consistency and continuity of the selected route.
If the validation succeeds, the process transitions to the Saving Path state, during which the system stores the new path in the database. In case of a validation failure or a save error, the system moves to an error path and returns to the idle state, allowing the user to retry or discard the creation.

If the saving process completes successfully, the system reaches the final state, confirming that the new path has been correctly created and stored.
Throughout the process, the user can cancel the creation, which immediately returns the system to the idle state.

\begin{figure}[H]
	\centering
	\includegraphics[width=\textwidth]{Images/State_Diagrams/manual_path_creation.png}
	\caption{Manual Path Creation State Diagram}
\end{figure}

\textbf{Automatically creating a new path}:

The following diagram illustrates the states involved in the automatic creation of a new path by a registered user using GPS tracking.
After accessing the application, the user can start the automatic creation process by selecting the “Automatic Mode” option. Once this action is triggered, the system leaves the idle state and enters the Waiting GPS Lock state, where it requests location permissions and attempts to establish a stable GPS connection.
If the GPS signal is successfully acquired, the system transitions to the Recording state, during which the user’s movements are continuously tracked and converted into path segments.

While recording, the user can manually stop the tracking (userStop), or the process can automatically stop if a long period of inactivity is detected.
If the GPS connection is lost for a prolonged time or the user denies permission, the process transitions to the failed path, ending unsuccessfully and returning to the idle state.
When the user stops the recording or the system detects the end of movement, the process enters the Validating Path state, where the recorded route is checked for continuity, consistency, and potential GPS anomalies.

If the validation succeeds, the system moves to the Saving Path state, where the path data are stored in the database. Any validation or saving errors cause a transition to a failure path, returning to the idle state.
Once the path has been successfully saved, the process reaches its final state, confirming that the automatically tracked path has been correctly recorded and stored.

\begin{figure}[H]
	\centering
	\includegraphics[width=\textwidth]{Images/State_Diagrams/automatic_path_creation.png}
	\caption{Automatic Path Creation State Diagram}
\end{figure}

\textbf{Manually inserting a report about an obstacle}:

The following diagram illustrates the states related to the manual reporting process, which allows a registered user to create a report manually.
Starting from the Idle state, the user initiates the manual report creation process by selecting the appropriate option from the interface. Once the process begins, the system moves to the Editing Metadata state, where the user can enter or modify relevant information, such as the description, and obstacle type.

After the metadata are saved, the process continues in the Waiting for GPS state. In this phase, the system attempts to obtain the user’s current position to associate the report with a specific geographic location.
If the GPS lock is successfully acquired, the process transitions to the Saving Report state, where the report is stored in the database.

If any error occurs during the GPS acquisition or the saving phase (GPSError, saveError), or if the user decides to cancel the operation (reportCanceled), the process terminates prematurely and returns to the Idle state.
When the report is successfully saved, the system reaches the final state, confirming that the manual report has been correctly created and recorded.

\begin{figure}[H]
	\centering
	\includegraphics[width=\textwidth]{Images/State_Diagrams/manual_report.png}
	\caption{Manual Report State Diagram}
\end{figure}

\textbf{Automatically detecting and reporting an obstacle}:

The following diagram represents the states involved in the automatic reporting process, where a registered user can generate a report automatically when a relevant event or sensor input is detected.
The process starts from the Idle state. When the system detects a sensor signal or an anomaly (SensorDetection), it transitions to the Confirming Report state, where the user is asked to confirm whether the detected event should be reported.

If the user confirms, the process continues in the Editing Metadata state, allowing the user to optionally edit or add information related to the report, such as a short description or obstacle type. Once the metadata are saved, the system moves to the Waiting for GPS state, where it tries to obtain the current position to associate it with the report.

If the GPS signal becomes active, the system transitions to the Saving Report state, where the report data are finalized and stored in the database.
If at any point the user cancels the report (reportCanceled), or the GPS fails to provide a valid location (GPSError), the process terminates prematurely and returns to the idle state.

Finally, if the saving procedure completes successfully (saveComplete), the system reaches its final state, confirming that the automatic report has been correctly generated and saved.

\begin{figure}[H]
	\centering
	\includegraphics[width=\textwidth]{Images/State_Diagrams/automatic_report.png}
	\caption{Automatic Report State Diagram}
\end{figure}

% --------------------------------------------------------------------------
% Product Functions % 
% --------------------------------------------------------------------------
\section{Product functions}
\label{sec:product_functions}

This section provides an overview of the main system functionalities as perceived by users.

\begin{itemize}
	\item \textbf{Some Function}:

	      Some Explication.
	\item \textbf{Another Function}:

	      Another Explication.
\end{itemize}

% --------------------------------------------------------------------------
% User Characteristics % 
% --------------------------------------------------------------------------
\section{User Characteristics}
\label{sec:user_characteristics}

This section describes the categories of users expected to interact with the system, their skills, and their goals.

\begin{itemize}
	\item \textbf{Some User}:

	      Some Explication.
	\item \textbf{Another User}:

	      Another Explication.
\end{itemize}

% --------------------------------------------------------------------------
% Assumption, dependencies and constraints % 
% --------------------------------------------------------------------------
\section{Assumptions, dependencies and constraints}
\label{sec:assumptions_dependencies_constraints}

This section lists external factors that influence the system but are beyond its direct control.

\subsection{Dependencies}
\label{sec:dependencies}

This subsection lists the external systems or services on which BBP depends.

\subsection{Constraints}
\label{sec:constraints}

This subsection specifies any technical, legal, or environmental restriction that affects the system design.

\subsection{Domain Assumptions}
\label{sec:domain_assumptions}

This subsection lists the assumptions about the real world that are taken as true for the requirements to hold.

\begin{itemize}
	\item[\textbf{[D1]}] The system has access to a geographical map service and geocoding capabilities to process origin/destination and render paths.

	\item[\textbf{[D2]}] The manual input provided by the Registered users is accurate and complete as intended for a path entry.

	\item[\textbf{[D3]}] The external meteorological service is operational and accessible by the system.

	\item [\textbf{[D4]}] The Registered users' mobile devices are equipped with GPS, and they have granted the BBP application location permissions and motion sensor access.

	\item [\textbf{[D5]}] The mobile device sensors (accelerometer, gyroscope) provide sufficiently reliable data to suggest the presence of obstacles.

	\item[\textbf{[D6]}] A \emph{bike path} is considered either a dedicated bicycle track or a low-traffic road where motor-vehicle speed limits are compatible with average cycling speed.

	\item[\textbf{[D7]}] Notification or Pop ups sent by the BBP system reaches the users within 5 min.
\end{itemize}