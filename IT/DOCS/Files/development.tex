
\section{Adopted Frameworks}
\label{sec:frameworks}

\subsection{Frontend}
\label{sec:frontend}

React Native
Expo
React Paper -> Theming
Lucide Icons -> Icons
React Native Maps -> Maps
Axios -> Api Calls
Zod -> Data Validation
Zustand -> State Management
Expo Router -> Navigation

\subsection{Backend}
\label{sec:backend}

NestJS \\
Openmeteo service

The backend infrastructure for BBP is developed using NestJS, a progressive Node.js framework. This choice provides a highly structured, modular architecture that ensures scalability and maintainability as the platform grows.
The key architectural decisions and components of the backend are as follows:

\begin{itemize} 
    \item \textbf{Modular Architecture:} The system is organized into self-contained modules (e.g., UsersModule, TripsModule, PathsModule). This separation of concerns allows for independent development and testing of core features like the automated pothole detection logic and the path scoring algorithm.

    \item \textbf{Dependency Injection and Type Safety:} Leveraging TypeScript and NestJS built-in dependency injection, the backend ensures robust data handling. Business logic is encapsulated in Services, which are injected into Controllers to handle incoming HTTP requests from the mobile application.

    \item \textbf{External Service Integration (Open-Meteo):} To fulfill the requirement of enriching trip data with meteorological information, the system integrates the Open-Meteo API.
    
    \begin{itemize} % This nested list is already correct
        \item \textbf{Data Acquisition:} A dedicated weather service within the backend performs asynchronous calls to Open-Meteo's REST endpoints using geographic coordinates (latitude/longitude) collected during user trips.
        \item \textbf{Performance:} Since Open-Meteo provides high-resolution data without requiring an API key, it allows the BBP system to retrieve temperature, wind speed, and weather conditions in real-time or historically to match the exact timeframe of a recorded cycling session.
    \end{itemize}
\end{itemize}

\subsection{Data Layer}
\label{subsec:data_layer}

PostgreSQL
Prisma

\section{Adopted Programming Languages}
\label{subsec:languages}

TypeScript

\section{Development Tools}
\label{sec:dev_tools}

Docker + Shared Nginx + Cloudflare + Docker

\section{API Calls}
\label{sec:api_calls}

Any API not included in the DD should be mentioned here.
