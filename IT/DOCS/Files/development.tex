
\section{Adopted Frameworks}
\label{sec:frameworks}

The BBP platform relies on a set of technologies selected to cover the needs of the mobile client, the server
side, and the persistence layer. In the sections below, we outline the core frameworks used in each area and
briefly describe their role, key characteristics, and why they were chosen for this project.

\subsection{Mobile App}
\label{sec:mobile_app}

The BBP mobile application is developed with React Native in the Expo ecosystem, allowing a single codebase
to target both iOS and Android while keeping GPS access consistent across devices. Expo also simplifies build
and deployment steps, which reduces overhead during development and testing.

Navigation is handled through Expo Router, which offers file-based routing and keeps screen structure easy to
maintain as the app grows. Global state is managed with Zustand, selected for its minimal boilerplate and
efficient updates, which are important for user session handling.

The user interface relies on React Native Paper to provide a Material Design-compliant component library with
built-in theming support. Lucide Icons supplies a lightweight and consistent icon set, while React Native Maps
is used for the core map experience, including path visualization and score overlays. Expo Linear Gradient is
used in selected screens to provide branded visual accents in the interface.

Location tracking is implemented via Expo Location, which handles permission requests and continuous GPS
updates needed for map views and trip-related features.

For data exchange, Axios manages HTTP requests to the backend. Zod defines schemas for client-side input
validation (e.g., authentication and profile forms) and integrates with React Hook Form via resolvers to
enforce those constraints in the UI. Expo Secure Store persists authentication tokens on device.

\subsection{Backend}
\label{sec:backend}

NestJS
Openmeteo service

\subsection{Data Layer}
\label{subsec:data_layer}

For data storage and management, the BBP platform relies on \textbf{PostgreSQL} as its primary \textbf{DBMS}. 
This choice is motivated by its proven reliability and by its relational data model, which fits well
with the platform’s core entities, such as users, paths, segments, and trips, and supports the enforcement 
of consistency constraints across related data.

A relational approach is particularly suitable for BBP, where path-related information evolves over time and
must remain consistent despite frequent user-generated updates. \textbf{PostgreSQL} provides the transactional 
guarantees and integrity mechanisms required to manage this evolving dataset in a robust and predictable way.

The interaction between the modular \textbf{NestJS} backend and the database is handled through \textbf{Prisma}, which is 
adopted as the \textbf{ORM}. \textbf{Prisma} generates a type-safe client starting from a single schema
definition, ensuring that database queries are strongly typed and aligned with the \textbf{TypeScript} types 
used throughout the application. This approach helps detect data access errors at compile time and keeps 
the persistence layer consistent as the data model evolves during development.

\section{Adopted Programming Languages}
\label{subsec:languages}

The platform is mainly developed using \textbf{TypeScript}, which is adopted across both the mobile application, 
built with \textbf{React Native} and \textbf{Expo}, and the backend services, built with \textbf{NestJS} and \textbf{Prisma}.
Using a single, strongly typed language
across different layers of the system improves maintainability and reduces the likelihood of runtime 
errors, while also simplifying development workflows.

On the server side, \textbf{TypeScript} integrates smoothly with \textbf{Prisma}, as the generated client types help
identify data access issues at compile time. On the client side, it works well with form validation
and API interaction, making input models and component properties explicit and easier to evolve over time.

Overall, this language choice results in clearer and more robust code, supports safer refactoring, and
reduces the effort required for developers to work across multiple components of the platform.

\section{Development Tools}
\label{sec:dev_tools}

The development workflow is based on \textbf{Node.js} and \textbf{npm}, which are used to manage dependencies and execute 
project scripts for both the backend and the mobile application. On the backend side, \textbf{Docker} and \textbf{Docker
Compose} are adopted to build and run the service in a reproducible and isolated environment, ensuring 
consistency across development and production setups. Database-related tasks, including client generation 
and schema migrations, are handled through the \textbf{Prisma CLI}.

For the mobile application, development and local testing are carried out using the \textbf{Expo CLI}. The \textbf{Expo Go} 
application enables real-time previews on physical devices by scanning the Metro QR code, allowing rapid 
iteration and immediate feedback during development. For distributable builds, the \textbf{EAS CLI} is used to 
generate \textbf{Android} and \textbf{iOS} build artifacts, with APK files produced from the Android build output.

In the production environment, the backend is exposed through a shared \textbf{Nginx} reverse proxy. HTTPS 
termination and certificate management are handled using \textbf{Cloudflare Origin Certificates}, which centralize 
TLS configuration and request routing while keeping backend services isolated from direct internet exposure.

To support integration testing and a shared development workflow, \textbf{EchoAPI} is used to mock and inspect 
backend responses. This tool allows team members to validate client-side behavior against expected API 
outputs, facilitating coordination between frontend and backend development and reducing coupling during 
implementation phases.

\section{API Calls}
\label{sec:api_calls}

Any API not included in the DD should be mentioned here.
