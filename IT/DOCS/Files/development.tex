
\section{Adopted Frameworks}
\label{sec:frameworks}

The BBP platform relies on a set of technologies selected to cover the needs of the mobile client, the server
side, and the persistence layer. In the sections below, we outline the core frameworks used in each area and
briefly describe their role, key characteristics, and why they were chosen for this project.

\subsection{Mobile App}
\label{sec:mobile_app}

The BBP mobile application is developed with React Native in the Expo ecosystem, allowing a single codebase
to target both iOS and Android while keeping GPS access consistent across devices. Expo also simplifies build
and deployment steps, which reduces overhead during development and testing.

Navigation is handled through Expo Router, which offers file-based routing and keeps screen structure easy to
maintain as the app grows. Global state is managed with Zustand, selected for its minimal boilerplate and
efficient updates, which are important for user session handling.

The user interface relies on React Native Paper to provide a Material Design-compliant component library with
built-in theming support. Lucide Icons supplies a lightweight and consistent icon set, while React Native Maps
is used for the core map experience, including path visualization and score overlays. Expo Linear Gradient is
used in selected screens to provide branded visual accents in the interface.

Location tracking is implemented via Expo Location, which handles permission requests and continuous GPS
updates needed for map views and trip-related features.

For data exchange, Axios manages HTTP requests to the backend. Zod defines schemas for client-side input
validation (e.g., authentication and profile forms) and integrates with React Hook Form via resolvers to
enforce those constraints in the UI. Expo Secure Store persists authentication tokens on device.

\subsection{Backend}
\label{sec:backend}

NestJS
Openmeteo service

\subsection{Data Layer}
\label{subsec:data_layer}

PostgreSQL
Prisma

\section{Adopted Programming Languages}
\label{subsec:languages}

TypeScript

\section{Development Tools}
\label{sec:dev_tools}

Docker + Shared Nginx + Cloudflare + Docker

\section{API Calls}
\label{sec:api_calls}

Any API not included in the DD should be mentioned here.
