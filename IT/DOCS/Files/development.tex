
\section{Adopted Frameworks}
\label{sec:frameworks}

The BBP platform relies on a set of technologies selected to cover the needs of the mobile client, the server
side, and the persistence layer. In the sections below, we outline the core frameworks used in each area and
briefly describe their role, key characteristics, and why they were chosen for this project.

\subsection{Mobile Application}
\label{sec:mobile_application}

The BBP mobile application is developed using \textbf{React Native} within the \textbf{Expo} ecosystem.
This choice allows the use of a single codebase to target both Android and iOS platforms, while ensuring
consistent access to device features such as GPS. Expo also simplifies build, deployment,
and permission management, reducing development overhead and improving iteration speed during testing.

Application navigation is handled through \textbf{Expo Router}, which provides a file-based routing system.
This approach helps maintain a clear and scalable screen structure as the application grows.
Global state management is implemented using \textbf{Zustand}, selected for its minimal boilerplate and efficient
update mechanism, which is particularly suitable for managing user sessions and cross-screen shared state.

The user interface is built using \textbf{React Native Paper}, which offers a Material Design-compliant component
library with built-in theming support. This enables a consistent visual appearance across devices and
simplifies the implementation of light, dark, and system-defined themes.
\textbf{Lucide Icons} is used to provide a lightweight and consistent icon set across the application.

Map visualization and interaction represent a core feature of the app and are implemented using
\textbf{React Native Maps}. This library is used to display bike paths, recorded trips, reports, and ranking-related
visual overlays. Location tracking is managed through \textbf{Expo Location}, which handles permission requests and
continuous GPS updates required for navigation and trip recording functionalities.

Communication with the backend services is handled through \textbf{Axios}, which manages HTTP requests and
response handling. Client-side input validation is implemented using \textbf{Zod}, which defines schemas for
forms such as authentication and profile management and integrates with React Hook Form through submit-time validation.
Authentication tokens and sensitive session data are securely stored on the device using \textbf{Expo Secure Store}.

\subsection{Backend}
\label{sec:backend}

\textbf{Express.js}, or simply Express, is a back end web application framework for Node.js, released as free and
open-source software under the MIT License. It is designed for building web applications and APIs.

The main reason for choosing Express.js as the backend framework is because it is very easy-to-use and flexible, together with
its ability to maintain high performance while offering a comprehensive set of features. Express provides pre-built functions,
libraries, and tools that help accelerate the web development process.

It has also one of the most powerful and robust routing system that assists your application in responding to a client request via
a particular endpoint. With the routing system in Express.js, we were able to split the routing system into manageable files using
the framework’s router instance. It is very helpful in managing the application structure grouping different routes into a
single folder/directory.

ExpressJS uses middleware to process incoming requests before they reach their final destination. This allows us to
perform tasks such as authentication, validation, and logging in a reusable and modular manner.

\textbf{Jest} is a testing framework for JavaScript applications, developed and maintained by Meta and distributed
as open-source software.
It provides an all-in-one solution for writing, executing, and maintaining automated tests, including unit tests and integration tests.
Jest includes built-in support for test runners, mocking utilities, and code coverage analysis, reducing the need for additional
external dependencies.

Jest was selected primarily for its integration with modern JavaScript and Node.js environments.
It allowed us to quickly set up a testing environment with minimal overhead, while still offering advanced configuration options
when needed. The framework supports asynchronous testing out of the box, which is essential for backend applications that rely heavily on
asynchronous operations such as database access and HTTP requests.

\subsection{Data Layer}
\label{subsec:data_layer}

For data storage and management, the BBP platform relies on \textbf{PostgreSQL} as its primary \textbf{DBMS}.
This choice is motivated by its proven reliability and by its relational data model, which fits well
with the platform’s core entities, such as users, paths, segments, and trips, and supports the enforcement
of consistency constraints across related data.

A relational approach is particularly suitable for BBP, where path-related information evolves over time and
must remain consistent despite frequent user-generated updates. \textbf{PostgreSQL} provides the transactional
guarantees and integrity mechanisms required to manage this evolving dataset in a robust and predictable way.

The interaction between the modular \textbf{Express} backend and the database is handled through \textbf{Prisma}, which is
adopted as the \textbf{ORM}. \textbf{Prisma} generates a type-safe client starting from a single schema
definition, ensuring that database queries are strongly typed and aligned with the \textbf{TypeScript} types
used throughout the application. This approach helps detect data access errors at compile time and keeps
the persistence layer consistent as the data model evolves during development.

\section{Adopted Programming Languages}
\label{sec:languages}

The platform is mainly developed using \textbf{TypeScript}, which is adopted across both the mobile application,
built with \textbf{React Native} and \textbf{Expo}, and the backend services, built with \textbf{Express} and \textbf{Prisma}.
Using a single, strongly typed language
across different layers of the system improves maintainability and reduces the likelihood of runtime
errors, while also simplifying development workflows.

On the server side, \textbf{TypeScript} integrates smoothly with \textbf{Prisma}, as the generated client types help
identify data access issues at compile time. On the client side, it works well with form validation
and API interaction, making input models and component properties explicit and easier to evolve over time.

Overall, this language choice results in clearer and more robust code, supports safer refactoring, and
reduces the effort required for developers to work across multiple components of the platform.

\section{Development Tools}
\label{sec:dev_tools}

The development workflow is based on \textbf{Node.js} and \textbf{npm}, which are used to manage dependencies and execute
project scripts for both the backend and the mobile application. On the backend side, \textbf{Docker} and \textbf{Docker
	Compose} are adopted to build and run the service in a reproducible and isolated environment, ensuring
consistency across development and production setups. Database-related tasks, including client generation
and schema migrations, are handled through the \textbf{Prisma CLI}.

For the mobile application, development and local testing are carried out using the \textbf{Expo CLI}. The \textbf{Expo Go}
application enables real-time previews on physical devices by scanning the Metro QR code, allowing rapid
iteration and immediate feedback during development. For distributable builds, the \textbf{EAS CLI} is used to
generate \textbf{Android} and \textbf{iOS} build artifacts, with APK files produced from the Android build output.

In the production environment, the backend is exposed through a shared \textbf{Nginx} reverse proxy. HTTPS
termination and certificate management are handled using \textbf{Cloudflare Origin Certificates}, which centralize
TLS configuration and request routing while keeping backend services isolated from direct internet exposure.

To support integration testing and a shared development workflow, \textbf{EchoAPI} is used to mock and inspect
backend responses. This tool allows team members to validate client-side behavior against expected API
outputs, facilitating coordination between frontend and backend development and reducing coupling during
implementation phases.

\section{Technologies Used}
\label{sec:technologies_used}

The following is a list of the main technologies, libraries, and tools used in the development of the BBP platform:

\begin{itemize}
  	\item \textbf{React Native}: framework for building native mobile applications using JavaScript and React,
	      enabling code reuse across Android and iOS platforms.\cite{rn:react_native}
	\item \textbf{Expo}: development platform built on top of React Native that simplifies configuration,
	      permissions management, build processes, and access to native device features.\cite{expo:expo}
	\item \textbf{Expo Router}: file-based routing system used to structure navigation and manage application screens
	      in a scalable and maintainable way.\cite{expo:expo_router}
	\item \textbf{Zustand}: lightweight state management library used to handle global application state such as
	      authentication, user preferences, and shared UI data.\cite{zustand:docs}
	\item \textbf{React Native Paper}: UI component library implementing Material Design principles, providing
	      accessible and themable components for a consistent user interface.\cite{rnpaper:react_native_paper}
	\item \textbf{Lucide Icons}: icon library used to supply a coherent and lightweight set of vector icons
	      across the application.\cite{lucide:docs}
	\item \textbf{React Native Maps}: library used for interactive map rendering, including bike path visualization,
	      trip tracking, and report overlays.\cite{rn_maps:docs}
	\item \textbf{Expo Location}: module used to manage location permissions and retrieve continuous GPS updates
	      required for navigation and trip recording features.\cite{expo:location}
	\item \textbf{Axios}: HTTP client used to perform asynchronous requests to backend APIs and handle responses.\cite{axios:docs}
	\item \textbf{Zod}: schema validation library used for client-side data validation, integrated with
	      React Hook Form to enforce input constraints in forms.\cite{zod:docs}
	\item \textbf{React Hook Form}: form management library used to handle user input efficiently and reduce
	      unnecessary re-renders in complex forms.\cite{rhf:react_hook_form}
	\item \textbf{Expo Secure Store}: secure storage mechanism used to persist authentication tokens and
	      sensitive session data on the device.\cite{expo:securestore}
  	\item \textbf{Joi}: data validation library used to define schemas for request payloads, ensuring that incoming data conforms to
	      expected formats and constraints before being processed by the application.\cite{node:joi}
	\item \textbf{bcrypt}: cryptographic hashing library used to securely store user passwords by applying adaptive hashing algorithms,
	      protecting against brute-force and rainbow table attacks.\cite{node:bcrypt}
	\item \textbf{jsonwebtoken}: library used to implement stateless authentication through JSON Web Tokens (JWT), enabling secure
	      user session management and authorization across API endpoints.\cite{node:jsonwebtoken}
	\item \textbf{cors}: middleware that enables and configures Cross-Origin Resource Sharing, allowing controlled access to backend
	      resources from client applications hosted on different domains.\cite{node:cors}
	\item \textbf{Pino}: high-performance logging library designed for Node.js applications, used to record structured logs for monitoring,
	      debugging, and auditing purposes.\cite{node:pino}
	\item \textbf{pino-pretty}: development tool used to format Pino logs into a human-readable form, improving log readability during
	      debugging and local development.\cite{node:pino-pretty}
	\item \textbf{Supertest}: HTTP testing library that allows automated testing of RESTful APIs by simulating HTTP requests and validating
	      responses without requiring a running network server.\cite{node:supertest}
  \item \textbf{PostgreSQL}: relational database management system used as the primary data store for the platform.
	      It provides strong transactional guarantees, support for complex relational schemas, and integrity
	      constraints required to manage evolving entities such as users, paths, segments, trips, and reports.\cite{postgresql:docs}
	\item \textbf{Prisma}: Object-Relational Mapping (ORM) tool used to interface the Express backend with the database.
	      Prisma generates a type-safe client from a centralized schema definition, ensuring consistency between
	      the database model and the TypeScript types used in the application, and enabling compile-time detection
	      of data access errors.\cite{prisma:docs}
	\item \textbf{Node.js}: JavaScript runtime environment based on the V8 engine, enabling the execution of server-side
	      applications using an event-driven, non-blocking I/O model that is well suited for scalable backend services.\cite{node:Node.js}
	\item \textbf{npm}: package manager used to manage project dependencies and execute predefined scripts
	      for development, testing, and build tasks.\cite{node:npm}
	\item \textbf{Docker}: containerization platform used to package the backend application and its dependencies
	      into isolated and reproducible runtime environments.\cite{docker:docker}
	\item \textbf{Docker Compose}: orchestration tool used to define and manage multi-container setups, including
	      backend services and database instances, during local development and deployment.\cite{docker:compose}
	\item \textbf{Prisma CLI}: command-line tool used to manage database schemas, generate type-safe clients,
	      and apply migrations in a controlled and consistent manner.\cite{prisma:cli}
	\item \textbf{Expo CLI}: development tool used to run, test, and debug the mobile application locally within
	      the Expo ecosystem.\cite{expo:cli}
	\item \textbf{Expo Go}: companion application that enables real-time testing of the mobile app on physical
	      devices by loading the development bundle via QR code.\cite{expo:go}
	\item \textbf{EAS CLI}: build and deployment tool used to generate distributable Android and iOS application
	      artifacts, including APK files for Android.\cite{expo:eas}
	\item \textbf{Nginx}: reverse proxy used in production to expose backend services, handle request routing,
	      and provide an additional security layer between clients and internal services.\cite{nginx:docs}
	\item \textbf{Cloudflare Origin Certificates}: TLS certificate mechanism used to terminate HTTPS connections
	      at the proxy level while keeping backend services isolated from direct internet exposure \cite{cloudflare:origin_cert}.
	\\item \textbf{EchoAPI}: API development and testing tool used to perform and inspect HTTP requests
      against backend endpoints and validate request and response formats \cite{echoapi:docs}.
\end{itemize}

\section{API Calls}
\label{sec:api_calls}

Any API not included in the DD should be mentioned here.
