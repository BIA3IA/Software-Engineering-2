
\section{Adopted Frameworks}
\label{sec:frameworks}

\subsection{Frontend}
\label{sec:frontend}

React Native
Expo
React Paper -> Theming
Lucide Icons -> Icons
React Native Maps -> Maps
Axios -> Api Calls
Zod -> Data Validation
Zustand -> State Management
Expo Router -> Navigation

\subsection{Backend}
\label{sec:backend}

\textbf{Express.js} Express.js, or simply Express, is a back end web application framework for Node.js, released as free and 
open-source software under the MIT License. It is designed for building web applications and APIs. It has been called the de 
facto standard server framework for Node.js.\cite{enwiki:1331088521}

The main reason for choosing Express.js as the backend framework is because it is very easy-to-use and flexible, together with 
its ability to maintain high performance while offering a comprehensive set of features. Express provides pre-built functions, 
libraries, and tools that help accelerate the web development process.

It has also one of the most powerful and robust routing system that assists your application in responding to a client request via
a particular endpoint. With the routing system in Express.js, we were able to split the routing system into manageable files using 
the framework’s router instance. It is very helpful in managing the application structure grouping different routes into a 
single folder/directory.

ExpressJS uses middleware to process incoming requests before they reach their final destination. This allows us to 
perform tasks such as authentication, validation, and logging in a reusable and modular manner.

\textbf{Jest} is a testing framework for JavaScript applications, developed and maintained by Meta and distributed 
as open-source software.
It provides an all-in-one solution for writing, executing, and maintaining automated tests, including unit tests and integration tests.
Jest includes built-in support for test runners, mocking utilities, and code coverage analysis, reducing the need for additional
external dependencies.

Jest was selected primarily for its integration with modern JavaScript and Node.js environments.
It allowed us to quickly set up a testing environment with minimal overhead, while still offering advanced configuration options 
when needed. The framework supports asynchronous testing out of the box, which is essential for backend applications that rely heavily on 
asynchronous operations such as database access and HTTP requests.

\subsubsection{Technologies used in the Backend}

The backend relies on the following technologies and libraries:

\begin{itemize}
  \item \textbf{Node.js}: JavaScript runtime environment based on the V8 engine, enabling the execution of server-side 
                        applications using an event-driven, non-blocking I/O model that is well suited for scalable backend services\cite{node:Node.js}.
  \item \textbf{Joi}: data validation library used to define schemas for request payloads, ensuring that incoming data conforms to 
                        expected formats and constraints before being processed by the application\cite{node:joi}.
  \item \textbf{bcrypt}: cryptographic hashing library used to securely store user passwords by applying adaptive hashing algorithms,
                        protecting against brute-force and rainbow table attacks\cite{node:bcrypt}.
  \item \textbf{jsonwebtoken}: library used to implement stateless authentication through JSON Web Tokens (JWT), enabling secure 
                        user session management and authorization across API endpoints\cite{node:jsonwebtoken}.
  \item \textbf{cors}: middleware that enables and configures Cross-Origin Resource Sharing, allowing controlled access to backend 
                        resources from client applications hosted on different domains\cite{node:cors}.
  \item \textbf{Pino}: high-performance logging library designed for Node.js applications, used to record structured logs for monitoring,
                        debugging, and auditing purposes\cite{node:pino}.
  \item \textbf{pino-pretty}: development tool used to format Pino logs into a human-readable form, improving log readability during 
                        debugging and local development\cite{node:pino-pretty}.
  \item \textbf{Supertest}: HTTP testing library that allows automated testing of RESTful APIs by simulating HTTP requests and validating 
                        responses without requiring a running network server\cite{node:supertest}.
\end{itemize}


\subsection{Data Layer}
\label{subsec:data_layer}

PostgreSQL
Prisma

\section{Adopted Programming Languages}
\label{subsec:languages}

TypeScript

\section{Development Tools}
\label{sec:dev_tools}

Docker + Shared Nginx + Cloudflare + Docker

\section{API Calls}
\label{sec:api_calls}

Any API not included in the DD should be mentioned here.
