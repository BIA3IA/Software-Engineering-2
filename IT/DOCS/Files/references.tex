% --------------------------------------------------------------------------
% Reference Documents
% --------------------------------------------------------------------------

\section{Reference Documents}
\label{sec:refdocs}%

The preparation of this document was supported by the following reference materials:
\begin{itemize}
	\item IEEE Standard for Software Requirement Specifications \cite{ieee};
	\item Assignment specification for the ITD of the Software Engineering II course,
	      held by professors Matteo Rossi, Elisabetta Di Nitto, and Matteo Camilli at the
	      Politecnico di Milano, Academic Year 2025/2026 \cite{itd};
	\item Slides of the Software Engineering II course available on WeBeep \cite{slides}.
\end{itemize}

% --------------------------------------------------------------------------
% Software Used
% --------------------------------------------------------------------------

\section{Software Used}
\label{sec:software}

The following software tools have been used to support the development of this project:

\begin{itemize}
	\item \textbf{Visual Studio Code}: editing of source code and documentation (LaTeX), with project-wide search and formatting support \cite{vscode}.
	\item \textbf{LaTeX}: typesetting system used to produce the final RASD document in a consistent format \cite{latex}.
	\item \textbf{Git}: version control used to track changes and support collaborative development \cite{git}.
	\item \textbf{GitHub}: remote repository hosting and collaboration platform used for versioning, reviews, and issue tracking \cite{github}.
	\item \textbf{Lucidchart}: creation of UML diagrams (use case diagrams, state diagrams, domain class diagram) \cite{lucidchart}.
\end{itemize}

% --------------------------------------------------------------------------
% AI Tools Usage
% --------------------------------------------------------------------------

\section{Use of AI Tools}
\label{sec:ai}

AI tools were used during the project in the same way as other supporting software tools.
Their role was not to autonomously generate content, but to assist in
improving the presentation of the document, supporting the organisation of ideas and enhancing
overall textual coherence.

Their use was mainly limited to the drafting phase, where they helped compare different ways
of explaining scenarios, simplify long paragraphs, and check whether certain sentences could
be misunderstood. In several cases, interacting with an AI assistant helped clarify the
underlying concepts before writing the final version of the text.

\subsection{Tools Used}
The AI tools employed during the project were:
\begin{itemize}
	\item Gemini
	\item ChatGPT
\end{itemize}

\subsection{Typical Prompts}
AI tools were queried using prompts such as:
\begin{itemize}
	\item "Rephrase this design description to make the interaction flow clearer."
	\item "Does this explanation of the component interaction sound ambiguous?"
	\item "Help restructure this paragraph describing a UI flow to improve readability."
	\item "Format this design description or table using LaTeX"
	\item "Help debug formatting or build issues related to VS Code or LaTeX"
\end{itemize}

\subsection{Input Provided}
The input given to AI tools consisted mainly of:
\begin{itemize}
	\item Early drafts of paragraphs.
	\item Short text fragments requiring clarity checks.
	\item Sections with repeated structure where consistent wording was needed.
\end{itemize}

\subsection{Constraints Applied}
When using AI tools, the following constraints were strictly enforced:
\begin{itemize}
	\item Preserve the intended meaning of the original text.
	\item Avoid introducing new design decisions or assumptions.
	\item Maintain terminology aligned with the definitions provided in this document.
\end{itemize}

\subsection{Outputs Obtained}
The interaction with AI tools resulted in:
\begin{itemize}
	\item Clearer or more concise formulations of existing statements.
	\item Identification of potentially ambiguous sentences.
	\item Terminology suggestions to improve internal coherence.
	\item LaTeX formatting assistance for tables and code snippets.
\end{itemize}

\subsection{Refinement Process}
All AI-generated outputs were subject to a manual refinement process that included:
\begin{itemize}
	\item Critical review of all suggestions.
	\item Verification against the original intent to avoid unintended changes.
	\item Manual integration to ensure consistency with the overall writing style.
	\item Alignment checks with established terminology and definitions.
\end{itemize}
