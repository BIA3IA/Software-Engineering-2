% --------------------------------------------------------------------------
% Reference Documents
% --------------------------------------------------------------------------

\section{Reference Documents}
\label{sec:refdocs}%

The preparation of this document was supported by the following reference materials:
\begin{itemize}
	\item Assignment specification for the ITD of the Software Engineering II course,
	      held by professors Matteo Rossi, Elisabetta Di Nitto, and Matteo Camilli at the
	      Politecnico di Milano, Academic Year 2025/2026 \cite{itd};
	\item Slides of the Software Engineering II course available on WeBeep \cite{slides}.
\end{itemize}

% --------------------------------------------------------------------------
% Software Used
% --------------------------------------------------------------------------

\section{Software Used}
\label{sec:software}

The following software tools have been used to support the development of this project:

\begin{itemize}
	\item \textbf{Visual Studio Code}: editing of source code and documentation (LaTeX), with project-wide search and formatting support \cite{vscode}.
	\item \textbf{LaTeX}: typesetting system used to produce the final RASD document in a consistent format \cite{latex}.
	\item \textbf{Git}: version control used to track changes and support collaborative development \cite{git}.
	\item \textbf{GitHub}: remote repository hosting and collaboration platform used for versioning, reviews, and issue tracking \cite{github}.
\end{itemize}

% --------------------------------------------------------------------------
% AI Tools Usage
% --------------------------------------------------------------------------

\section{Use of AI Tools}
\label{sec:ai}

AI tools were used during the project as supporting tools, in the same way as other software
adopted in the development process. Their role was not to automatically generate content, but
to help improve the clarity, structure, and overall quality of the documentation.

Their use was mainly limited to the writing and revision phases. They were helpful in rephrasing sentences,
simplifying long or unclear passages, and checking whether explanations could be misunderstood. In some cases,
interacting with an AI assistant also helped clarify ideas before writing the final version of the text.

\subsection{Tools Used}
The AI tools employed during the project were:
\begin{itemize}
	\item Gemini
	\item ChatGPT
	\item Copilot
\end{itemize}

\subsection{Typical Prompts}
AI tools were queried using prompts such as:
\begin{itemize}
	\item "Rephrase this description to make it clearer."
	\item "Does this explanation of the backend deployment process contain any ambiguities?"
	\item "Suggest alternative wording for this technical paragraph to improve flow."
	\item "Identify any terms in this section that may be inconsistent with the definitions provided earlier
	in the document."
	\item "Format this design description or table using LaTeX".
	\item "What's the difference between interface and type in TypeScript, and in what 
	specific scenarios is one recommended over the other? How do you achieve type safety?"
	\item "What is the config.ts file and how do you configure it in the best way?"
	\item "What's the ideal structure for organizing a Node.js/Express project to clearly
	 separate routes, controllers, and business logic?"
	\item "What are the best practices for structuring a React Native project?"
	\item "Explain how mocking works in Jest. What are the best practices?"
	\item "How does a Dockerfile work, and what are the best practices for writing one for a Node.js application?"
	\item "How does a Docker Compose file work, and what are the best practices
	 to set up Docker Compose for a multi-service Node.js application?"
	\item "How to monitor logs of Docker containers effectively?"
\end{itemize}

\subsection{Input Provided}
The input given to AI tools consisted mainly of:
\begin{itemize}
	\item Early drafts of paragraphs.
	\item Short text fragments requiring clarity checks.
	\item Sections with repeated structure where consistent wording was needed.
	\item Technical descriptions needing LaTeX formatting.
	\item Code snippets or configuration files requiring explanations or best practices.
\end{itemize}

\subsection{Constraints Applied}
When using AI tools, the following constraints were strictly enforced:
\begin{itemize}
	\item Preserve the intended meaning of the original text.
	\item Avoid introducing new design decisions or assumptions.
	\item Maintain terminology aligned with the definitions provided in this document.
\end{itemize}

\subsection{Outputs Obtained}
The interaction with AI tools resulted in:
\begin{itemize}
	\item Clearer or more concise formulations of existing statements.
	\item Identification of potentially ambiguous sentences.
	\item Terminology suggestions to improve internal coherence.
	\item LaTeX formatting assistance for tables and code snippets.
	\item Explanations of technical concepts and best practices.
\end{itemize}

\subsection{Refinement Process}
All AI-generated outputs were subject to a manual refinement process that included:
\begin{itemize}
	\item Critical review of all suggestions.
	\item Verification against the original intent to avoid unintended changes.
	\item Manual integration to ensure consistency with the overall writing style.
	\item Alignment checks with established terminology and definitions.
\end{itemize}
