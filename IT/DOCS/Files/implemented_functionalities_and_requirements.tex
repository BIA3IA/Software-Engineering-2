\section{Product Functions}
\label{sec:product_functions}

This section describes the core functionalities of the app, organized to support the main goals and
requirements defined for the project.

\subsection{User Access and Identity Management}
\label{subsec:user_access_identity}

The app allows users to access the system either as guests or as authenticated users, with different levels of
interaction. Guest users can explore bike paths, visualize ranked routes between two locations, start trips, and
access general information. Their interaction is limited, as they cannot access restricted areas of the app, submit
or validate reports, and any data generated during their usage is not saved.

Guest users can register at any time by providing basic identification information. After logging in, users gain
access to the full set of functionalities, including viewing previously recorded trips, accessing personal
statistics, managing created bike paths, and contributing to the system by submitting or validating reports.
Authenticated users can also access and modify their personal profile information.

\subsection{Trip Recording and Management}
\label{subsec:trip_recording_management}

Users can start cycling trips both as guests and as authenticated users, but any data generated during the activity
as a guest user is not saved once the trip ends.

Logged-in users can record and manage their cycling trips with full data persistence. A trip represents a single
cycling activity and includes information such as duration, distance, and the route followed.
Additional contextual information, such as weather conditions, may also be retrieved to provide further
context to the recorded activity. When the trip ends, all collected data is saved and becomes part of the user’s
trip history.

\subsection{Path Discovery, Creation and Selection}
\label{subsec:path_discovery_creation_selection}

The app allows users to search and explore bike paths. Users can select an
origin and a destination, and the app suggests one or more bike paths connecting the two locations. Suggested paths
are ordered according to their overall quality and suitability, based on the information available in the system.
Users can select a specific path and view its details, including its condition, length, and any associated reports.

Logged-in users can also create new bike paths to extend the set of available routes. Paths can be created
by drawing the route directly on the map. When creating a path, users can choose whether it should remain private
or be shared with the community. Either way, the created path is stored in the system and can be accessed by the user.

\subsection{Map-based Visualization and Navigation Support}
\label{subsec:map_visualization_navigation}

The app uses an interactive map as the main interface to display bike paths and trips. Suggested paths, selected
routes, and recorded trips are shown directly on the map, allowing users to easily understand the layout and
characteristics of each route.

While navigating, the app shows the user’s current position along the selected path, making it easy to track
progress in real time. The map may also display reports, providing useful contextual information while the user
is moving.

\subsection{Report Submission and Confirmation}
\label{subsec:report_submission_confirmation}

Logged-in users can contribute information about bike paths conditions by submitting reports. Reports describe
obstacles, anomalies, or the overall condition of a path segment and can be created manually by the user
during an active trip. In all cases, reports are linked to a specific path.

To improve the reliability of the collected information, the app allows users to confirm or reject existing reports,
while cycling. User feedback helps reduce false positives and outdated data, making
the reported information more accurate over time.

\subsection{Path Condition Evaluation and Ranking} % TODO this may need to be reviewed after implementation
\label{subsec:path_condition_evaluation_ranking}

The app evaluates the condition of bike paths by combining information collected from user reports and recorded trips.
Each path is associated with indicators that describe its current condition, taking into account reported issues,
detected obstacles, confirmations provided by multiple users over time and their freshness.
This allows the app to keep an updated and reliable view of path quality.

\subsubsection*{Path Status Scoring Model}
\label{subsubsec:path_status_scoring}

Each report submitted by a user refers to a specific path condition, represented as a status. To enable
aggregation and quantitative reasoning, each status is mapped to a numerical score as follows:

\begin{table}[H]
	\centering
	\begin{tabularx}{\textwidth}{|X|X|}
		\hline
		\rowcolor[HTML]{CFE2F3} \textbf{Path Status} & \textbf{Numerical Score} \\ \hline
		Optimal                                      & 5                        \\ \hline
		Medium                                       & 4                        \\ \hline
		Sufficient                                   & 3                        \\ \hline
		Requires Maintenance                         & 2                        \\ \hline
		Closed                                       & 1                        \\ \hline
	\end{tabularx}
	\caption{Mapping of path status to numerical scores}
	\label{tab:path_status_scores}
\end{table}

\subsubsection*{Report Freshness Model}
\label{subsubsec:report_freshness}

Since real-world path conditions evolve over time, the algorithm assigns a freshness weight to each report in order to
prioritize recent information. For a report $i$, freshness is computed as:

\[
	fresh_i = 2^{-\frac{ageMin_i}{H_{min}}}
\]

where $ageMin_i$ represents the age of the report in minutes, computed as the difference between the current time and
the report confirmation timestamp.
This exponential decay ensures that older reports progressively lose influence over the aggregation process.

\subsubsection*{Report Validation Contribution}
\label{subsubsec:report_validation_contribution}

Reports may undergo validation by other users and can be confirmed or rejected. Only validated interactions contribute
to the reliability of a report. For each report, two partial scores are computed:

\[
	confirmedScore = \sum fresh_i \quad \text{for each confirmation}
\]

\[
	rejectedScore = \sum fresh_i \quad \text{for each rejection}
\]

Reports in the \emph{IGNORED} state do not contribute to either score, while those in the \emph{CREATED} state contribute
to the confirmed score.

\subsubsection*{Report Reliability Computation}
\label{subsubsec:report_reliability}

The overall reliability of a report is computed by combining confirmation and rejection scores through a weighted
difference:

\[
	reportReliability = clamp(1 + \alpha \cdot confirmedScore - \beta \cdot rejectedScore,\; min,\; max)
\]

where $\alpha$ and $\beta$ are weighting parameters controlling the impact of confirmations and rejections respectively,
and $min$ and $max$ define lower and upper bounds for reliability. In the current configuration, $\alpha = 0.6$,
$\beta = 0.8$, $min = 0.1$, and $max = 2.5$.

Reports whose reliability falls below a minimum threshold are excluded from further aggregation, as they are considered
no longer representative of the current path condition.

\subsubsection*{Path Status Aggregation}
\label{subsubsec:path_status_aggregation}

Reports first update the status of the specific path segment they refer to. The segment-level status is derived
from report scores weighted by reliability and freshness. The path status is then computed
by combining segment statuses with a weighted mix:

\[
	PathStatusScore = 0.7 \cdot \overline{score}_{reportedSegments} + 0.3 \cdot \overline{score}_{allSegments}
\]

where $\overline{score}_{reportedSegments}$ is the average score of segments that have at least one valid report,
and $\overline{score}_{allSegments}$ is the average score over all segments in the path. If no segments have valid
reports, the path status defaults to the average over all segments. This approach avoids overly diluting short-path
reports while still keeping long paths stable.

\subsubsection*{Path Status Determination}
\label{subsubsec:path_status}

The final numerical score is mapped back to a discrete path status according to predefined thresholds:

\begin{table}[H]
	\centering
	\begin{tabularx}{\textwidth}{|X|X|}
		\hline
		\rowcolor[HTML]{CFE2F3} \textbf{Path Status} & \textbf{Score Range} \\ \hline
		Optimal                                      & $[4.5, 5]$           \\ \hline
		Medium                                       & $[3.5, 4.5)$         \\ \hline
		Sufficient                                   & $[2.5, 3.5)$         \\ \hline
		Requires Maintenance                         & $[1.5, 2.5)$         \\ \hline
		Closed                                       & $[1, 1.5)$           \\ \hline
	\end{tabularx}
	\caption{Mapping of numerical scores to discrete path statuses}
	\label{tab:path_status}
\end{table}

\subsubsection*{Impact on Path Ranking}
\label{subsubsec:path_ranking_impact}

The computed path status directly influences the ranking of paths during route discovery. Paths with higher evaluated
conditions are prioritized when suggesting routes between an origin and a destination. Since the merging algorithm is
continuously updated as new reports and validations are received, path rankings dynamically adapt to evolving real-world
conditions.

\subsection{Statistics Computation and Caching}
\label{subsec:statistics_computation_caching}

The app computes statistics to help users better understand their cycling activities and overall performance.
Statistics are generated from recorded trip data.

To keep the system efficient, the app avoids unnecessary recomputations. Aggregated statistics are updated only when
new trips are recorded, while previously computed results are reused whenever possible. Per-trip statistics are
generated when needed and then stored, so they do not have to be recalculated every time they are accessed.
As a result, users can access up-to-date statistics without affecting the overall performance of the app.

\subsection{Weather Data Acquisition and Enrichment}
\label{subsec:weather_data_acquisition}

The app retrieves weather information from external meteorological services to provide additional context for cycling
activities. Weather data, such as temperature, wind, and other relevant conditions, is associated with recorded trips
based on the time and location of the activity.

Weather information is collected when a trip is completed, ensuring that the recorded conditions accurately reflect
the environment in which the activity took place. This allows users to better understand their trips and interpret
performance data in relation to external factors.

\section{Requirements}
\label{sec:requirements}

\begin{longtable}{|p{0.07\textwidth}|>{\raggedright\arraybackslash}p{0.65\textwidth}|>{\raggedright\arraybackslash}p{0.23\textwidth}|}
	\hline
	\textbf{Rx}  & \textbf{Description}                                                                                                                            & \textbf{Implemented}                                                                                       \\ \hline
	\endfirsthead

	\hline
	\textbf{Rx}  & \textbf{Description}                                                                                                                            & \textbf{Implemented}                                                                                       \\ \hline
	\endhead

	\hline
	\multicolumn{3}{|r|}{Continued on next page}                                                                                                                                                                                                                                \\ \hline
	\endfoot

	\endlastfoot

	\textbf{R1}  & The system shall allow guest users to create an account by providing personal information and credentials.                                      & Yes                                                                                                        \\ \hline

	\textbf{R2}  & The system shall allow registered users to log into the application using valid credentials.                                                    & Yes                                                                                                        \\ \hline

	\textbf{R3}  & The system shall allow logged-in users to view their profile and account settings.                                                              & Yes                                                                                                        \\ \hline

	\textbf{R4}  & The system shall allow logged-in users to update their profile and account settings.                                                            & Yes                                                                                                        \\ \hline

	\textbf{R5}  & The system shall allow logged-in users to log out of the application, ending their current session.                                             & Yes                                                                                                        \\ \hline

	\textbf{R6}  & The system shall allow guest users to start a cycling trip.                                                                                     & Yes                                                                                                        \\ \hline

	\textbf{R7}  & The system shall allow guest users to stop a currently active trip, but shall not store any trip data after the trip ends.                      & Yes                                                                                                        \\ \hline

	\textbf{R8}  & The system shall allow the user to start a trip only when their GPS position matches the path origin.                                           & Yes                                                                                                        \\ \hline

	\textbf{R9}  & The system shall display a pop-up suggesting to start a trip when cycling is detected while no trip is active and the app is open.              & TODO                                                                                                       \\ \hline

	\textbf{R10} & The system shall set the current GPS position as trip origin when starting from auto-detection.                                                 & TODO                                                                                                       \\ \hline

	\textbf{R11} & The system shall automatically stop the active trip when the user's GPS position deviates from the selected path within a certain threshold.    & TODO                                                                                                       \\ \hline

	\textbf{R12} & The system shall allow logged-in users to start a cycling trip in manual or automatic mode.                                                     & Partial - Only Manual Mode was required                                                                    \\ \hline

	\textbf{R13} & The system shall allow logged-in users to stop a currently active trip and save the recorded data.                                              & Yes                                                                                                        \\ \hline

	\textbf{R14} & The system shall collect GPS data during trip recording.                                                                                        & Yes                                                                                                        \\ \hline

	\textbf{R15} & The system shall collect motion sensor data (accelerometer, gyroscope) during trip recording only when Automatic Mode is enabled.               & No - Not required                                                                                          \\ \hline

	\textbf{R16} & The system shall allow logged-in users to view the list of their recorded trips.                                                                & Yes                                                                                                        \\ \hline

	\textbf{R17} & The system shall allow logged-in users to view a summary of their overall cycling statistics (total distance, total time, average speed, etc.). & Yes                                                                                                        \\ \hline

	\textbf{R18} & The system shall allow logged-in users to view statistics for each trip (distance, speed, duration, etc.).                                      & Yes                                                                                                        \\ \hline

	\textbf{R19} & The system shall display the route and reported obstacles associated with a recorded trip.                                                      & Yes                                                                                                        \\ \hline

	\textbf{R20} & The system shall allow logged-in users to delete a recorded trip.                                                                               & Yes                                                                                                        \\ \hline

	\textbf{R21} & The system shall communicate with external weather services to retrieve meteorological data related to the time and location of a trip.         & Yes                                                                                                        \\ \hline

	\textbf{R22} & The system shall detect when a user is cycling based on speed and acceleration patterns.                                                        & TODO                                                                                                       \\ \hline

	\textbf{R23} & The system shall detect irregular movements from sensor data that may suggest potholes or surface defects when Automatic Mode is enabled.       & No - Not required                                                                                          \\ \hline

	\textbf{R24} & The system shall present automatically detected path and obstacle data to the logged-in user for manual confirmation before publishing.         & No - Not required                                                                                          \\ \hline

	\textbf{R25} & The system shall allow logged-in users to manually create a new bike path by drawing segments.                                                  & Yes                                                                                                        \\ \hline

	\textbf{R26} & The system shall allow logged-in users to manually report obstacles or problems on a bike path while performing an active trip.                 & Yes                                                                                                        \\ \hline

	\textbf{R27} & The system shall allow logged-in users to manually confirm or reject the presence of obstacles reported by other users.                         & Yes                                                                                                        \\ \hline

	\textbf{R28} & The system shall allow logged-in users to create a new bike path in automatic mode using GPS tracking.                                          & No - Not required  (but I think it wont be difficult actually to implement, so let's think about this one) \\ \hline

	\textbf{R29} & The system shall allow logged-in users to delete their previously created paths.                                                                & Yes                                                                                                        \\ \hline

	\textbf{R30} & The system shall allow logged-in users to set the visibility of their created paths as public or private.                                       & Yes                                                                                                        \\ \hline

	\textbf{R31} & The system shall aggregate multiple user reports referring to the same path segment.                                                            & Yes                                                                                                        \\ \hline

	\textbf{R32} & The system shall evaluate the reliability of each path segment based on the number of confirmations and report freshness.                       & Yes                                                                                                        \\ \hline

	\textbf{R33} & The system shall determine the current status of a path (optimal, medium, sufficient, requires maintenance, closed).                            & Yes                                                                                                        \\ \hline

	\textbf{R34} & The system shall allow any user (guest or logged-in) to view the detailed status and latest reports of a selected bike path.                    & Yes                                                                                                        \\ \hline

	\textbf{R35} & The system shall allow any user (guest or logged-in) to browse available public bike paths on a map.                                            & Yes                                                                                                        \\ \hline

	\textbf{R36} & The system shall allow any user to search for bike paths connecting two locations.                                                              & Yes                                                                                                        \\ \hline

	\textbf{R37} & The system shall compute suggested routes based on path quality and distance.                                                                   & Yes                                                                                                        \\ \hline

	\textbf{R38} & The system shall rank suggested routes according to their safety and quality.                                                                   & Yes                                                                                                        \\ \hline

	\textbf{R39} & The system shall display the user's current GPS position during navigation along a selected path.                                               & Yes                                                                                                        \\ \hline

	\textbf{R40} & The system shall send pop-ups to warn users about nearby obstacles or closed path segments during an active trip.                               & Yes                                                                                                        \\ \hline

	\textbf{R41} & The system shall interface with map and geocoding services to translate addresses into coordinates and render paths.                            & Yes                                                                                                        \\ \hline

	\textbf{R42} & The system shall ensure that communication with all external services (map, weather) handles temporary unavailability gracefully.               & Yes                                                                                                        \\ \hline

	\caption{Mapping between BBP Requirements and implemented functionalities}
	\label{tab:R_MAP}
\end{longtable}
