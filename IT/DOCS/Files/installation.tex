The BBP platform can be used in different configurations, depending on the intended use case and on whether
the user aims to simply run the application or to actively develop and test it.

In particular, three main usage scenarios are supported:
\begin{itemize}
    \item \textbf{Remote backend with prebuilt mobile application}: the user installs a prebuilt APK and interacts with an already deployed backend instance.
    \item \textbf{Fully local execution}: both the backend services and the mobile application are executed locally for development and testing purposes.
    \item \textbf{Hybrid configuration}: the mobile application is run locally while relying on a remotely deployed backend.
\end{itemize}

\section{Prerequisites}
\label{sec:prerequisites}

To run the BBP platform locally, a set of prerequisites must be satisfied to ensure a consistent and 
reproducible development environment.

\begin{itemize}
    \item \textbf{Node.js and npm}: required to manage dependencies and execute project scripts for both the
    backend services and the mobile application. All build, test, and development workflows rely on the Node.js runtime.
    
    \item \textbf{Docker and Docker Compose}: required to run backend services and auxiliary components in isolated
    containers. In the local setup, Docker Compose is used to start the BBP backend, the PostgreSQL database,
    and the OSRM routing service, ensuring a configuration that closely mirrors the production environment without
    requiring manual installation of these components on the host system.
    
    \item \textbf{Expo CLI}: required to run the mobile application locally during development. It is used to
    start the Metro bundler and to enable live previews of the application on emulators or physical devices.
    
    \item \textbf{Expo Go}: required on a physical mobile device to preview the application during development
    by connecting to the local Metro bundler started via Expo CLI. May be replaced with an emulator.
\end{itemize}

\section{Backend Setup}
\label{sec:backend_setup}

\subsection{Production Deployment}
\label{subsec:backend_production}

The BBP backend is deployed on a \textbf{VPS} using a \textbf{container-based architecture} built on Docker
and Docker Compose. The service is designed to coexist with other applications hosted on the same server
while remaining isolated and not directly exposed to the public network.

The backend application runs as a stateless service inside \textbf{Docker containers} and listens on port 3000,
which is reachable only within the container network. No backend container exposes public ports on the
host system. All incoming traffic is handled by a \textbf{shared NGINX reverse proxy}, which is the only component
exposing ports 80 and 443.

NGINX acts as the \textbf{single entry point} for the system and is responsible for HTTPS termination, request
routing, load balancing, and basic traffic filtering. TLS is managed using \textbf{Cloudflare Origin Certificates},
with Cloudflare acting as \textbf{DNS provider and upstream security layer}. Requests validated by Cloudflare are
forwarded to the reverse proxy, which then routes them to the backend service.

The backend is horizontally scalable and is deployed using \textbf{multiple container replicas}. In the current
configuration, three backend instances are executed in parallel. NGINX resolves backend containers
dynamically using \textbf{Docker’s embedded DNS} and \textbf{balances incoming requests} across replicas through an
upstream definition. This approach allows the system to scale without introducing a dedicated load
balancer and ensures high availability of the API layer.

The deployment process starts by copying the backend project files to the server. This can be performed
from a local machine using a secure file transfer mechanism, for example:

\begin{verbatim}
rsync -avz ./BACKEND/ user@<SERVER_IP>:/opt/bbp-backend/
\end{verbatim}

Once the project is available on the server, environment-specific configuration values are provided
through a dedicated \textbf{.env} file. This file defines runtime parameters such as authentication
secrets, database connection details, and service timeouts. Sensitive values are \textbf{never committed} to
version control and are configured directly on the server.

The backend is built and started using \textbf{Docker Compose}. A shared \textbf{Docker network}
is created once and is used to connect the reverse proxy, backend replicas, and auxiliary services.
The backend containers join this network and expose port 3000 only internally. The services can be
built and started with the following commands:

\begin{verbatim}
docker compose build --no-cache
docker compose --profile tools run --rm migrate
docker compose up -d --scale api=3
\end{verbatim}

Database \textbf{migrations} are executed explicitly through a dedicated migration container and are only run
when schema changes are introduced. This avoids unintended schema modifications during routine
restarts.

Persistent data storage is handled through \textbf{PostgreSQL} accessed via \textbf{Prisma Accelerate}. This configuration
allows the backend to rely on a \textbf{managed database connection layer} without hosting a database instance
directly on the server, while still preserving transactional guarantees and schema consistency.

In addition to the core backend service, snapping-related functionalities are delegated to a separate
\textbf{OSRM service}, which runs in its own container. OSRM is responsible for snapping user-defined paths to
the OpenStreetMap road network using a cycling profile. The backend communicates with OSRM through an
internal HTTP interface and exposes a dedicated API endpoint to the mobile application.

After deployment, the backend can be verified by inspecting container status and logs:

\begin{verbatim}
docker compose ps
docker logs -f bbp_api --tail=50
\end{verbatim}

A dedicated \textbf{health endpoint} is exposed and can be used by the reverse proxy to verify service
availability. If correctly deployed, the backend responds to authenticated API requests through the
public API endpoint exposed by the reverse proxy.

\subsection{Local Backend Execution}
\label{subsec:backend_local}

\section{Frontend Setup}
\label{sec:frontend_setup}

Use Expo CLI to run the app locally on an emulator or a physical device.
You can also build an APK for Android and install it directly.

If you want to use a local server you should change the API URL in the .env file and build it or run it on the emulator,
since the built app points to the production server.
