
\section{Purpose}
\label{sec:purpose}
The purpose of the Best Bike Paths (BBP) platform is to help cyclists find and use the most suitable routes by
collecting real-world path data, aggregating reports, and ranking alternatives. 
This document focuses on the technical implementation and architectural choices behind the platform. 
While the overall functional
requirements and high-level design are described in the Requirements Analysis and Specification Document
(RASD) and the Design Document (DD),this document focuses specifically on the realization of the system.

\section{Scope}
\label{sec:scope}
This document, Implementation and Test Document (ITD), provides a comprehensive description of the implementation and testing phases 
of the BBP platform. Specifically, it focuses on the functionalities developed, the adopted frameworks, and the structure of the source code. 
Additionally, it includes a detailed testing strategy, covering the procedures, tools, and methodologies used during the developmentprocess.
This document also serves as a guide for installing and running the platform, offering installation instructions and addressing any prerequisites 
or potential issues. The effort spent by the team members is also summarized to provide insight into the workload distribution.

\section{Definitions, Acronyms}
This section provides definitions and explanations of the terms and acronyms used through out the document, 
making it easier for readers to understand and reference them.

\subsection{Definitions}
\label{subsec:definitions}
\begin{itemize}
	\item \textbf{Path:} A route created by users (manual drawing or GPS-based creation). A path consists of one or more Path Segments.
	\item \textbf{Path Segment:} A portion of a Path defined by its polyline geometry and linked to adjacent segments.
	\item \textbf{Path Status:} The overall condition of a Path or Path Segment, computed from user reports.
	\item \textbf{Path Score/Ranking:} The value used to order suggested paths, derived from status and distance when searching routes.
	\item \textbf{Trip:} A cycling activity tracked through the BBP application. If the user is logged in, the trip is stored and becomes part of the trip history, including temporal and spatial data (e.g., duration, distance, route).
	\item \textbf{Report:} A submission of path information by a logged-in user. Reports describe obstacles and path condition for a specific segment.
	\item \textbf{Freshness:} A metric used when aggregating reports; newer reports carry more weight than older ones when determining Path Status.
	\item \textbf{Obstacle:} An element on a path that negatively impacts cycling conditions, such as potholes or flooding, as identified by users.
	\item \textbf{Manual Creation Mode:} The creation mode in which a logged-in user defines a new path by drawing it on the map.
	\item \textbf{Automatic Creation Mode:} The creation mode in which a logged-in user defines a new path by cycling along it, allowing the system to reconstruct the path using GPS data.
	\item \textbf{Manual Report:} The functionality where a logged-in user manually creates a report by selecting the path condition and obstacle through the application interface.
\end{itemize}

\subsection{Acronyms}
\label{subsec:acronyms}

\begin{itemize}
    \item \textbf{BBP:} Best Bike Paths. 
    \item \textbf{API:} Application Programming Interface.
    \item \textbf{APK:} Android Package.
    \item \textbf{CLI:} Command Line Interface.
    \item \textbf{CRUD:} Create, Read, Update, Delete.
    \item \textbf{DBMS:} Database Management System.
    \item \textbf{DD:} Design Document.
    \item \textbf{EAS:} Expo Application Services.
    \item \textbf{GPS:} Global Positioning System.
    \item \textbf{HTTP:} HyperText Transfer Protocol.
    \item \textbf{IPA:} iOS App Store Package.
    \item \textbf{ITD:} Implementation and Test Document.
    \item \textbf{JSON:} JavaScript Object Notation.
    \item \textbf{JWT:} JSON Web Token.
    \item \textbf{NGINX:} Engine X (reverse proxy).
    \item \textbf{ORM:} Object-Relational Mapping.
    \item \textbf{OSRM:} Open Source Routing Machine.
    \item \textbf{QR Code:} Quick Response Code.
    \item \textbf{RASD:} Requirements Analysis and Specification Document.
    \item \textbf{REST:} Representational State Transfer.
    \item \textbf{SDK:} Software Development Kit.
    \item \textbf{TLS:} Transport Layer Security.
    \item \textbf{UI:} User Interface.
    \item \textbf{URL:} Uniform Resource Locator.
    \item \textbf{UX:} User Experience.
    \item \textbf{VPS:} Virtual Private Server.
\end{itemize}

\section{Revision History}
\label{sec:revhistory}

\begin{itemize}
	\item Version 1.0 (01 February 2026);
\end{itemize}

\section{Document Structure}
\label{sec:docstructure}

Mainly the current document is divided into six chapters:
\begin{enumerate}
	\item \textbf{Introduction:} provides an overview of the document, outlining its purpose, scope, and relevance to the project.
	\item \textbf{Implemented Functionalities and Requirements:} details the functionalities and requirements that have been implemented in the project.
    \item \textbf{Adopted Development Frameworks:} describes the development frameworks utilized in the project, explaining their roles and benefits.
    \item \textbf{Source Code Structure:} outlines the organization and structure of the source code, facilitating understanding and navigation.
    \item \textbf{Testing Strategy:} presents the testing methodologies and strategies employed to ensure the quality and reliability of the software.
    \item \textbf{Installation Instructions:} provides step-by-step guidance on how to install and set up the software.
    \item \textbf{References:} lists the references and resources used in the creation of the document and the project.
    \item \textbf{Effort Spent:} details the distribution of work and time spent by each team member throughout the project.
\end{enumerate}
