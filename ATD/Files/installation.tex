This section reports the installation process we executed to run the prototype
delivered by the group under analysis.
We followed the Installation Instructions provided in the ITD (Chapter 6), and we report
both the documented procedure and the issues encountered during the setup, together with
the corrective steps required to successfully run the system.

\section{Prerequisites}
\label{sec:prerequisites}

The ITD lists the following software requirements:
\begin{itemize}
	\item Java (JDK 8 or higher, JDK 11 recommended)
	\item MySQL 8.0+ with spatial support,
	\item Node.js (>= 14),
	\item npm (>= 6).
\end{itemize}

During the installation, we confirmed that these prerequisites are necessary.

Moreover, the document states that a Google Maps API key is required, and it explicitly lists
several required Google APIs (Maps JavaScript, Routes, Directions, and Geocoding).

However, during functional testing, we encountered a minor inconsistency between
the documented API requirements and the actual implementation.
The location search functionality relies on the Google Places API (Autocomplete), 
which is not mentioned in the installation requirements section.
This dependency is documented elsewhere in the ITD (specifically in Chapter 3 and 4), but
it is missing from the explicit list of required APIs in the installation prerequisites,
which caused an initial functional failure of the search bar until the Places API was
manually enabled in Google Cloud Console.

\section{Database Setup and Issues}
\label{sec:dbsetup}

The ITD provides a SQL script located at \texttt{backend/sql\_file/bbp.sql} and suggests
executing it using:
\texttt{mysql -u root -p < backend/sql\_file/bbp.sql}

While the script correctly creates the schema and the tables, the documented procedure
is not sufficient to set up the database from scratch in a clean environment.
The backend configuration expects a dedicated database user (through environment variables
such as \texttt{DB\_USERNAME} and \texttt{DB\_PASSWORD}).
However, the installation guide does not describe the creation of such user, nor the
privilege grants required to import the schema and let the application connect.

When executing the import using the configured application user, the setup fails
with a MySQL authentication error (ERROR 1045, access denied), because the user does
not exist yet or does not have the required privileges.

To complete the installation, we can set the DB\_USERNAME environment variable to \texttt{root} or those administrative steps 
should be performed which were not documented in the ITD:

\begin{verbatim}
cd backend
sudo mysql

CREATE DATABASE IF NOT EXISTS BestBikePathDB;
CREATE USER IF NOT EXISTS 'tester'@'localhost' IDENTIFIED BY 'bbp';
GRANT ALL PRIVILEGES ON BestBikePathDB.* TO 'tester'@'localhost';
FLUSH PRIVILEGES;
EXIT;
\end{verbatim}

After creating the user and granting privileges, we imported the schema by executing:

\begin{verbatim}
mysql -u tester -p BestBikePathDB < sql_file/bbp.sql
\end{verbatim}

This procedure successfully completed the database setup and allowed the backend
to connect correctly.

\section{Backend Setup and Environment Variables}
\label{sec:backendsetup}

The backend is delivered as a runnable Spring Boot JAR.
The ITD instructs to export environment variables (\texttt{DB\_USERNAME}, \texttt{DB\_PASSWORD}, \texttt{GOOGLE\_API\_KEY})
and then run the JAR from the backend directory using:
\texttt{java -jar best\_bike\_path-0.0.1-SNAPSHOT.jar}.

In our execution, a critical usability issue is that the documentation implies a straightforward
execution, but it does not clearly emphasize that the environment variables must be loaded in
the same shell session where the JAR is launched.
In our setup, we used a \texttt{.env} file and manually sourced it before starting the
backend, otherwise the application fails due to missing configuration values.

Final working procedure:

\begin{verbatim}
source .env
java -jar best_bike_path-0.0.1-SNAPSHOT.jar
\end{verbatim}

The backend starts on port 8080 as stated in the ITD.

\section{Frontend Setup and External API Inconsistency}
\label{sec:frontendsetup}

The ITD instructs to install frontend dependencies in \texttt{frontend/} and start
the development server with \texttt{npm start}.
The ITD also specifies that the Google Maps API key must be placed inside
\texttt{js/config.js}.

Following the documented procedure, the frontend setup completed successfully.
The application loaded as expected in the browser, and the overall frontend
initialization process can be considered smooth and straightforward.

\section{Repository Structure and Configuration Management}
\label{sec:repo_quality}

Upon inspecting the delivered source code, we identified issues regarding repository
organization and configuration management.

The project is split into two main directories, \texttt{backend/} and \texttt{frontend/},
coherently with the ITD description of the repository-level structure.
However, the repository lacks a \texttt{.gitignore} file.
This omission introduces two major problems.

\begin{itemize}
	\item \textbf{Security risk:} without ignore rules, sensitive configuration files
	      such as \texttt{.env} (containing database credentials and API keys) can be accidentally
	      committed to version control.
	      Even if such files were not committed in the delivered snapshot, the repository
	      configuration does not prevent accidental leaks in future commits.
	\item \textbf{Repository bloat and noise:} without excluding folders such as
	      \texttt{node\_modules/}, dependency files can be tracked by Git, increasing repository size
	      and raising the probability of merge conflicts and irrelevant diffs.
\end{itemize}
