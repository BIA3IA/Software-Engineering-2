% --------------------------------------------------------------------------
% Architectural Design
% --------------------------------------------------------------------------

% --------------------------------------------------------------------------
% Overview
% --------------------------------------------------------------------------
\section{Overview: High-level components and their interactions}
\label{sec:overview}%

% --------------------------------------------------------------------------
% Component View
% --------------------------------------------------------------------------

\section{Component View}
\label{sec:component_view}%

% --------------------------------------------------------------------------
% Deployment View
% --------------------------------------------------------------------------

\section{Deployment View}
\label{sec:deployment_view}%

\begin{figure}[H]
    \centering
    \includegraphics[width=0.9\textwidth]{../Images/deployment_view/deployment_view.png}
    \caption{Deployment View of the BBP System}
    \label{fig:deployment_view}
\end{figure}

The deployment view describes the hardware and software infrastructure supporting the \textit{BBP} system. Each tier is 
executed on dedicated hardware nodes and communicates with the others using secure protocols.

\begin{itemize}
    \item \textbf{Presentation Tier:} 
    this tier includes all devices through which end users interact with the BBP system. 
    The primary clients are smartphones or tablets running iOS or Android, where the BBP mobile application is installed. 
    These devices communicate with the backend exclusively via HTTPS, ensuring confidentiality and data integrity. 
    Although the mobile app represents the main access point, any device equipped with a modern web browser and a stable 
    internet connection could technically interact with the system, 
    since the backend exposes standard RESTful endpoints. 
    No application logic is executed at this level, the devices simply capture user input, display results, and forward 
    authenticated HTTP requests toward the Application Tier.

    \item \textbf{Application Tier:}
    this tier is responsible for handling incoming traffic, enforcing security, applying routing and load balancing policies, and 
    executing the core business logic of the system. 
    All external requests first pass through a Linux-based server configured as a Web Application Firewall using \textit{ModSecurity}.  
    ModSecurity blocks malicious traffic such as SQL injection attempts, cross-site scripting payloads, and abnormal request patterns,
     while supporting anomaly detection.  
    Validated HTTPS traffic is then forwarded to the gateway node running \textit{Traefik}, which acts as the system’s 
    reverse proxy and load balancer.  
    Traefik terminates HTTPS connections, exposes a single public entrypoint for clients, and distributes incoming requests 
    across multiple backend replicas using load balancing strategies.  
    Additional middleware functionalities (request logging or rate limiting) can be applied as needed at this level.
    The backend application itself is executed on one or more stateless Application Servers, each running the BBP RESTful backend.  
    Since authentication tokens are included in each request header, no server-side session state is maintained, enabling horizontal 
    scaling and dynamic replica management.  
    Communication between Traefik and the backend replicas is confined to a protected internal network, reducing the system’s 
    exposure to external threats.

    \item \textbf{Data Tier:}
    the Data Tier consists of a dedicated server running a PostgreSQL database instance, which stores all persistent system data 
    such as user information, paths, trips, and analytics.  
    Backend servers interact with the database using standard PostgreSQL drivers over TCP/IP within an isolated internal network 
    segment.  
    Centralizing the database simplifies backup strategies, consistency enforcement, and maintenance operations, while still 
    allowing for potential future extensions such as replication or clustering without altering the upper tiers of the system.

\end{itemize}

% --------------------------------------------------------------------------
% Runtime View
% --------------------------------------------------------------------------

\section{Runtime View}
\label{sec:runtime_view}%

% --------------------------------------------------------------------------
% Component Interfaces
% --------------------------------------------------------------------------

\section{Component Interfaces}
\label{sec:component_interfaces}%

% --------------------------------------------------------------------------
% Architectural Styles and Patterns
% --------------------------------------------------------------------------

\section{Selected architectural styles and patterns}
\label{sec:arch_patterns}%

% --------------------------------------------------------------------------
% Other Design Decisions
% --------------------------------------------------------------------------

\section{Other Design Decisions}
\label{sec:other_design_decisions}%