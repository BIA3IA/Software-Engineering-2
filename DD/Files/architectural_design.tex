% --------------------------------------------------------------------------
% Architectural Design
% --------------------------------------------------------------------------

% --------------------------------------------------------------------------
% Overview
% --------------------------------------------------------------------------
\section{Overview: High-level components and their interactions}
\label{sec:overview}%
\section{Architectural Overview}
\label{sec:architectural_overview}

The architecture of the Best Bike Paths (BBP) system follows a classic
\textbf{three-tier structure}, separating the software into a presentation layer,
an application layer, and a data layer. This architectural style ensures a clear
division of responsibilities and simplifies the evolution of the system over time.
Since BBP is conceived as a mobile-centric platform, the presentation tier is
implemented entirely through the BBP mobile application, which communicates with
the backend via RESTful APIs over HTTPS.

\subsection*{Presentation Layer}

The presentation layer consists solely of the \textbf{BBP mobile application},
which serves as the primary interface between users and the system.
This layer is responsible for:
\begin{itemize}
	\item rendering the user interface and handling user interactions;
	\item acquiring device-level data (GPS, accelerometer, gyroscope) during trips;
	\item displaying bike paths, trip summaries, statistics, and reports;
	\item invoking backend functionalities through HTTP requests.
\end{itemize}

The mobile app is intentionally designed as a \textbf{thin client}.
All domain logic, decision processes, ranking operations, and aggregation of path
information are delegated to the application layer. The app merely collects inputs
from the user and device sensors, forwards them to the backend, and presents the
returned results.

\subsection*{Application Layer}

The application layer embodies the \textbf{core business logic} of BBP.
It is implemented as a modular backend composed of independent yet cooperating \textbf{components},
each encapsulating a well–defined responsibility.
These components are logically independent in terms of responsibilities and interfaces,
but they are part of a single backend application.
The main functional components include:

\begin{itemize}
	\item \textbf{User Module}: contains the \textit{AuthManager} and the
	      \textit{UserManager}, responsible for authentication, credential verification,
	      and management of user profiles.
	\item \textbf{Trip Module}: contains the \textit{TripManager}, which handles
	      the lifecycle of a cycling trip and produces trip summaries enriched with
	      contextual weather data.
	\item \textbf{Path Module}: contains the \textit{PathManager}, responsible for
	      maintaining bike-path data, computing routes, and ranking candidate paths according to
	      their condition and effectiveness..
	\item \textbf{Report Module}: contains the \textit{ReportManager}, responsible
	      for storing and aggregating reports, managing confirmations, and updating
	      path-condition indicators.
	\item \textbf{Statistics Module}: contains the \textit{StatsManager}, which
	      computes and stores user statistics and per-trip metrics.
\end{itemize}


The application layer also includes the \textbf{WeatherManager}, which interacts
with an external weather service to retrieve meteorological data.
All modules are accessed through the \textbf{API Gateway}, which exposes a set of
RESTful sub-APIs and routes incoming requests to the appropriate Manager.

\subsection*{Data Layer}

The data layer consists of a \textbf{relational DBMS} storing all persistent
information relevant to the system’s domain, including:
\begin{itemize}
	\item user profiles and authentication credentials;
	\item trip records and associated GPS data;
	\item bike path segments and their aggregated conditions;
	\item reports, confirmations, and metadata about obstacles;
	\item computed statistics and weather snapshots.
\end{itemize}

All interactions with the DBMS are mediated by a single \textbf{QueryManager},
which centralises data-access operations and offers a uniform interface for
executing queries. This design keeps persistence concerns separated from the
application logic and reduces duplication across components.

% --------------------------------------------------------------------------
% Component View
% --------------------------------------------------------------------------
\section{Component View}
\label{sec:component_view}

This section describes the main software components that constitute the BBP backend
and their responsibilities. As required by the three-tier architecture adopted by the system,
the backend is structured into a set of independent yet cooperating modules, each exposing
well-defined interfaces and encapsulating a cohesive subset of the application logic.

\subsection{API Gateway}
The \textbf{API Gateway} acts as the entry point for all interactions between the BBP mobile
application and the backend. It is responsible for routing incoming requests to the appropriate
internal services, enforcing authentication and authorization requirements, validating inputs,
and translating domain errors into HTTP responses.

The API Gateway exposes the following logical sub-APIs:
\begin{itemize}
	\item \textbf{AuthAPI}: endpoints for token generation and validation.
	\item \textbf{UserAPI}: endpoints for registration, login, token refresh, and profile retrieval.
	\item \textbf{TripAPI}: endpoints for starting, updating, and stopping a trip.
	\item \textbf{RoutingAPI}: endpoints for route computation and retrieval of path metadata.
	\item \textbf{ReportAPI}: endpoints for creating and confirming obstacle reports.
	\item \textbf{StatisticsAPI}: endpoints for retrieving per-trip and aggregated statistics.
\end{itemize}

\subsection{User Module}
The \textbf{User Module} groups two Managers:
\begin{itemize}
	\item \textbf{AuthManager}, responsible for authentication and token issuance.
	\item \textbf{UserManager}, responsible for registration, profile updates,
	      and credential-related operations.
\end{itemize}
Both Managers use the \textit{QueryManager} for data retrieval and persistence.

\subsection{Trip Manager}
The \textbf{Trip Manager} manages the entire lifecycle of a cycling session.
It receives GPS data from the mobile application, records the user’s trajectory, and
stores trip metadata such as timestamps, duration, distance, and average speed.
When a trip is completed, the component generates its summary and associates it with
relevant environmental data retrieved from the Weather Manager.
The component relies on the \textit{QueryManager} for storage and communicates
with the Weather Manager to obtain contextual weather information.

\subsection{Trip Module}
The \textbf{Trip Module} contains the \textbf{TripManager}, that manages the entire
lifecycle of a cycling session. It receives GPS data from the mobile application, records
the user’s trajectory, and stores trip metadata such as timestamps, duration, distance, and average speed.
When a trip is completed, the component generates its summary and associates it with
relevant environmental data retrieved from the Weather Manager.
The component relies on the \textit{QueryManager} for storage and communicates
with the Weather Manager to obtain contextual weather information.

\subsection{Path Module}
The \textbf{Path Module} contains the \textbf{PathManager}, responsible for retrieving graph
data from the database, computing optimal routes between two locations,
and ranking alternative routes according to their quality and reported conditions.
This service exposes the routing logic to the API Gateway and interacts with the
\textit{QueryManager} to retrieve and update path and report information needed for route computation.

\subsection{Reports Manager}
The \textbf{Reports Manager} handles obstacle reports submitted
by users or automatically detected during trips. It stores and aggregates reports, manages
confirmation and rejection flows, updates path-quality indicators, and exposes relevant data to the mobile application.
This component interacts with the \textit{QueryManager} for persistence and with the
Routing \& Path Manager when updated conditions affect route evaluation.

\subsection{Report Module}
The \textbf{Report Module} contains the \textbf{ReportManager}, which stores and aggregates reports, manages
confirmation and rejection flows, updates path-quality indicators, and exposes relevant data to the mobile application.
This component interacts with the \textit{QueryManager} for persistence.

\subsection{Statistics Module}
The \textbf{Statistics Module} contains the \textbf{StatsManager}, which computes
and retrieves aggregated cycling statistics. It retrieves historical data through
the \textit{QueryManager}.

\subsection{WeatherManager}
The \textbf{WeatherManager} interacts with the external weather API to obtain
meteorological information. It provides weather snapshots associated with trip start and end points
It provides weather snapshots associated with trip start and end points
It stores weather snapshots using the \textit{QueryManager}.

\subsection{QueryManager}
The \textbf{QueryManager} is the data-access component of the backend.
It acts as the single entry point for interacting with the relational DBMS and
provides a set of methods to retrieve, insert, update, and delete domain data.
Centralising data access in one component simplifies consistency checks, reduces duplicated logic, and
keeps the domain layer independent of database details.

% --------------------------------------------------------------------------
% Deployment View
% --------------------------------------------------------------------------

\section{Deployment View}
\label{sec:deployment_view}%

% --------------------------------------------------------------------------
% Runtime View
% --------------------------------------------------------------------------

\section{Runtime View}
\label{sec:runtime_view}%

% --------------------------------------------------------------------------
% Component Interfaces
% --------------------------------------------------------------------------

\section{Component Interfaces}
\label{sec:component_interfaces}%

% --------------------------------------------------------------------------
% Architectural Styles and Patterns
% --------------------------------------------------------------------------

\section{Selected architectural styles and patterns}
\label{sec:arch_patterns}%

% --------------------------------------------------------------------------
% Other Design Decisions
% --------------------------------------------------------------------------

\section{Other Design Decisions}
\label{sec:other_design_decisions}%