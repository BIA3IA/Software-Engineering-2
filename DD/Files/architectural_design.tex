% --------------------------------------------------------------------------
% Architectural Design
% --------------------------------------------------------------------------

% --------------------------------------------------------------------------
% Overview
% --------------------------------------------------------------------------
\section{Overview: High-level components and their interactions}
\label{sec:overview}%
\section{Architectural Overview}
\label{sec:architectural_overview}

The architecture of the Best Bike Paths (BBP) system follows a classic
\textbf{three-tier structure}, separating the software into a presentation layer,
an app layer, and a data layer. This architectural style ensures a clear
division of responsibilities and simplifies the evolution of the system over time.
Since BBP is conceived as a mobile-centric platform, the presentation tier is
implemented entirely through the BBP mobile app, which communicates with
the backend via RESTful APIs over HTTPS.

\subsection*{Presentation Layer}

The presentation layer consists solely of the \textbf{BBP mobile app},
which serves as the primary interface between users and the system.
This layer is responsible for:
\begin{itemize}
	\item rendering the user interface and handling user interactions;
	\item acquiring device-level data (GPS, accelerometer, gyroscope) during trips;
	\item displaying bike paths, trip summaries, statistics, and reports;
	\item invoking backend functionalities through HTTP requests.
\end{itemize}

The mobile app is intentionally designed as a \textbf{thin client}.
All domain logic, decision processes, ranking operations, and aggregation of path
information are delegated to the app layer. The mobile app also uses
the secure storage facilities provided by the operating system (iOS Keychain / Android Keystore)
to safely store authentication tokens and other sensitive data.

\subsection*{Application Layer}

The app layer embodies the \textbf{core business logic} of BBP.
It is implemented as a modular backend composed of independent yet cooperating \textbf{components},
each encapsulating a well–defined responsibility.
These components are logically independent in terms of responsibilities and interfaces,
but they are part of a single backend app.
The main functional components include:

\begin{itemize}
	\item \textbf{User Module}: contains the \textit{AuthManager} and the
	      \textit{UserManager}, responsible for authentication, credential verification,
	      and management of user profiles.
	\item \textbf{Trip Module}: contains the \textit{TripManager}, which handles
	      the lifecycle of a cycling trip and produces trip summaries enriched with
	      contextual weather data.
	\item \textbf{Path Module}: contains the \textit{PathManager}, responsible for
	      maintaining bike-path data, computing routes, and ranking candidate paths according to
	      their condition and effectiveness..
	\item \textbf{Report Module}: contains the \textit{ReportManager}, responsible
	      for storing and aggregating reports, managing confirmations, and updating
	      path-condition indicators.
	\item \textbf{Statistics Module}: contains the \textit{StatsManager}, which
	      computes and stores user statistics and per-trip metrics.
\end{itemize}


The app layer also includes the \textbf{WeatherManager}, which interacts
with an external weather service to retrieve meteorological data.
All modules are accessed through the \textbf{API Gateway}, which exposes a set of
RESTful sub-APIs and routes incoming requests to the appropriate Manager.

\subsection*{Data Layer}

The data layer consists of a \textbf{relational DBMS} storing all persistent
information relevant to the system’s domain, including:
\begin{itemize}
	\item user profiles and authentication credentials;
	\item trip records and associated GPS data;
	\item bike path segments and their aggregated conditions;
	\item reports, confirmations, and metadata about obstacles;
	\item computed statistics and weather snapshots.
\end{itemize}

All interactions with the DBMS are mediated by a single \textbf{QueryManager},
which centralises data-access operations and offers a uniform interface for
executing queries. This design keeps persistence concerns separated from the
app logic and reduces duplication across components.

% --------------------------------------------------------------------------
% Component View
% --------------------------------------------------------------------------
\section{Component View}
\label{sec:component_view}

This section describes the main software components that constitute the BBP backend
and their responsibilities. As required by the three-tier architecture adopted by the system,
the backend is structured into a set of independent yet cooperating modules, each exposing
well-defined interfaces and encapsulating a cohesive subset of the app logic.

\figure[H]
\centering
\includegraphics[width=\textwidth]{Images/Architectural_Overview/component_view.png}
\caption{Component View Diagram}
\label{fig:component_view}
\endfigure

\subsubsection{API Gateway}
The \textbf{API Gateway} acts as the entry point for all interactions between the BBP mobile
app and the backend. It is responsible for routing incoming requests to the appropriate
internal services, enforcing authentication and authorization requirements, validating inputs,
and translating domain errors into HTTP responses.

The API Gateway exposes the following logical sub-APIs:
\begin{itemize}
	\item \textbf{AuthAPI}: endpoints for token generation and validation.
	\item \textbf{UserAPI}: endpoints for registration, login, token refresh, and profile retrieval.
	\item \textbf{TripAPI}: endpoints for starting, updating, and stopping a trip.
	\item \textbf{RoutingAPI}: endpoints for route computation and retrieval of path metadata.
	\item \textbf{ReportAPI}: endpoints for creating and confirming obstacle reports.
	\item \textbf{StatisticsAPI}: endpoints for retrieving per-trip and aggregated statistics.
\end{itemize}

\subsubsection{User Module}
The \textbf{User Module} groups two Managers:
\begin{itemize}
	\item \textbf{AuthManager}, responsible for authentication and token issuance.
	\item \textbf{UserManager}, responsible for registration, profile updates,
	      and credential-related operations.
\end{itemize}
Both Managers use the \textit{QueryManager} for data retrieval and persistence.

\subsubsection{Trip Manager}
The \textbf{Trip Manager} manages the entire lifecycle of a cycling session.
It receives GPS data from the mobile app, records the user’s trajectory, and
stores trip metadata such as timestamps, duration, distance, and average speed.
When a trip is completed, the component generates its summary and associates it with
relevant environmental data retrieved from the Weather Manager.
The component relies on the \textit{QueryManager} for storage and communicates
with the Weather Manager to obtain contextual weather information.

\subsubsection{Trip Module}
The \textbf{Trip Module} contains the \textbf{TripManager}, that manages the entire
lifecycle of a cycling session. It receives GPS data from the mobile app, records
the user’s trajectory, and stores trip metadata such as timestamps, duration, distance, and average speed.
When a trip is completed, the component generates its summary and associates it with
relevant environmental data retrieved from the Weather Manager.
The component relies on the \textit{QueryManager} for storage and communicates
with the Weather Manager to obtain contextual weather information.

\subsubsection{Path Module}
The \textbf{Path Module} contains the \textbf{PathManager}, responsible for retrieving graph
data from the database, computing optimal routes between two locations,
and ranking alternative routes according to their quality and reported conditions.
This service exposes the routing logic to the API Gateway and interacts with the
\textit{QueryManager} to retrieve and update path and report information needed for route computation.

\subsubsection{Reports Manager}
The \textbf{Reports Manager} handles obstacle reports submitted
by users or automatically detected during trips. It stores and aggregates reports, manages
confirmation and rejection flows, updates path-quality indicators, and exposes relevant data to the mobile app.
This component interacts with the \textit{QueryManager} for persistence and with the
Routing \& Path Manager when updated conditions affect route evaluation.

\subsubsection{Report Module}
The \textbf{Report Module} contains the \textbf{ReportManager}, which stores and aggregates reports, manages
confirmation and rejection flows, updates path-quality indicators, and exposes relevant data to the mobile app.
This component interacts with the \textit{QueryManager} for persistence.

\subsubsection{Statistics Module}
The \textbf{Statistics Module} contains the \textbf{StatsManager}, which computes
and retrieves aggregated cycling statistics. It retrieves historical data through
the \textit{QueryManager}.

\subsubsection{WeatherManager}
The \textbf{WeatherManager} interacts with the external weather API to obtain
meteorological information. It provides weather snapshots associated with trip start and end points
It provides weather snapshots associated with trip start and end points
It stores weather snapshots using the \textit{QueryManager}.

\subsubsection{QueryManager}
The \textbf{QueryManager} is the data-access component of the backend.
It acts as the single entry point for interacting with the relational DBMS and
provides a set of methods to retrieve, insert, update, and delete domain data.
Centralising data access in one component simplifies consistency checks, reduces duplicated logic, and
keeps the domain layer independent of database details.

% --------------------------------------------------------------------------
% Deployment View
% --------------------------------------------------------------------------

\section{Deployment View}
\label{sec:deployment_view}%

% --------------------------------------------------------------------------
% Runtime View
% --------------------------------------------------------------------------

\section{Runtime View}
\label{sec:runtime_view}%

The Runtime View describes how the components of the BBP system collaborate
to realise the behaviour specified in the functional requirements.
While the Component View focuses on the static organisation of the backend
(API Gateway, Managers, QueryManager) and their responsibilities,
the Runtime View illustrates how these components interact dynamically
during the execution of the main use cases.

All diagrams follow the modular structure of the backend:
the mobile client invokes the API Gateway, which routes each request to the
appropriate Manager.
Persistence operations are centralised in the QueryManager, whereas
weather-related data requests involve the WeatherManager and its external API.

The following pages report the sequence diagrams for all core use cases,
from user authentication to trip management, path exploration, reporting,
statistics retrieval, and profile updates.
Together, these diagrams provide a comprehensive understanding of how the BBP
system behaves during execution and how responsibilities are distributed among
its components.

\textbf{[UC1]} - User Registration

A new user wants to register into the BBP system.
The process begins on the mobile app, where the guest user opens the registration page,
and fills in the required details: email, password, and personal information.
A first local validation step checks for malformed inputs before contacting the backend.

If the data is valid, the mobile app sends a registration request to the API Gateway,
which forwards it to the AuthManager. This component checks that the email is not
already in use by querying the database through the QueryManager.
If the email already exists, the system returns an error and the mobile app notifiesthe user accordingly.

If the email does not exist, the AuthManager inserts the new user into the database
and generates an authentication token. The token is returned to the mobile app, which
stores it securely using the device’s secure storage mechanism. The flow concludes with a success message being shown to the user.

\begin{figure}[H]
	\centering
	\includegraphics[width=\textwidth]{Images/Sequence_Diagrams/user_registration.png}
	\caption{User Registration Sequence Diagram}
	\label{fig:use_case_user_registration}
\end{figure}
\pagebreak

\textbf{[UC2]} - User Log In

A guest user wants to authenticate and obtain access to the system. The process starts
when the user opens the login form and submits credentials through the mobile app. After
a local validation, the app sends an HTTP request to the login endpoint exposed by the API Gateway.

The API Gateway forwards the request to the AuthManager, which first checks whether the provided
email exists by querying the DBMS through the QueryManager. If the email is not found,
the backend returns a 404 Not Found error, which the mobile app displays to the user.
If the user exists, the AuthManager verifies the submitted password. Invalid credentials
lead to a 401 Unauthorized response and the corresponding error message on the client.

When the credentials are correct, the AuthManager generates an authentication token and
returns it to the mobile app, which stores it securely using the device's
secure storage facility. The user is then successfully logged in and the app
proceeds to show the appropriate authenticated UI.

\begin{figure}[H]
	\centering
	\includegraphics[width=\textwidth]{Images/Sequence_Diagrams/user_login.png}
	\caption{User Log In Sequence Diagram}
	\label{fig:use_case_user_login}
\end{figure}
\pagebreak

\textbf{[UC3]} - User Log Out

When a logged-in user initiates the logout operation from the mobile app, the client first clears
the locally stored authentication token from the secure storage.
Afterward, the mobile app sends an HTTP request to the backend.

The API Gateway forwards the request to the AuthManager, which handles the logout process by deleting
the corresponding refresh token through the QueryManager. The QueryManager executes a DELETE
operation on the database to invalidate the stored refresh token.

If the operation succeeds, the server returns a 204 NO\_CONTENT response, and the app
displays a success message to the user. If instead a network or server error occurs,
the system interrupts the flow and the mobile app notifies the user with a “Network unavailable” error message.

\begin{figure}[H]
	\centering
	\includegraphics[width=\textwidth]{Images/Sequence_Diagrams/user_logout.png}
	\caption{User Log Out Sequence Diagram}
	\label{fig:use_case_user_logout}
\end{figure}
\pagebreak

\textbf{[UC4]} - Search for a Path

A user can search for bike paths between two locations. He enters the start and end points
and submits the request. The app performs a preliminary local validation
and, if the data is valid, forwards the request to the backend through the API Gateway.

The API Gateway forwards the request to the PathManager, which loads the relevant
portion of the path graph from the database using the QueryManager. Once the graph
data is retrieved, the PathManager computes the optimal path(s) according to the
requested constraints. If valid routes are found, the API Gateway returns them to
the mobile app, which displays the corresponding suggestions.

If no route satisfies the user’s constraints, the PathManager signals a NO\_ROUTE
condition, which results in a 404 error. In the case of network or server issues,
the app notifies the user with an appropriate error message.

\begin{figure}[H]
	\centering
	\includegraphics[width=\textwidth]{Images/Sequence_Diagrams/search_path.png}
	\caption{Search for a Path Sequence Diagram}
	\label{fig:use_case_search_path}
\end{figure}
\pagebreak

\textbf{[UC5]} - Select a Path

The Select a Path sequence diagram illustrates how the system retrieves the details of
a path selected by a user. The interaction begins when the user chooses a specific path
from the list of suggested routes displayed by the mobile app. The app sends a request
to the backend through the API Gateway, which forwards it to the PathManager.

The PathManager retrieves the corresponding path information by querying the database
through the QueryManager. If a matching record is found, the PathManager returns the
path details to the API Gateway, which sends them back to the mobile app for
presentation to the user.

If the database does not contain a path with the specified identifier, the PathManager
signals a NOT\_FOUND condition, resulting in a 404 response. In that case, the mobile app
notifies the user that the selected route is unavailable. As in other interactions,
network or server errors trigger an appropriate error message on the client side.

\begin{figure}[H]
	\centering
	\includegraphics[width=\textwidth]{Images/Sequence_Diagrams/select_path.png}
	\caption{Select a Path Sequence Diagram}
	\label{fig:use_case_select_path}
\end{figure}
\pagebreak

\textbf{[UC6]} - Create a Path in Manual mode

When a logged-in user wants to create a new path, he will be asked to choose between
manual and automatic creation modes. If he opts for creating the path in manual mode,
he will be presented with a form to fill in the required metadata and segment list.

Before sending the request, the app performs local validation to ensure that all
mandatory fields and segments are correctly specified.

If the input is valid, the mobile app submits the creation request to the
backend through the API Gateway. The gateway forwards the request to the PathManager,
which is responsible for handling the creation workflow. The PathManager stores the new
path by delegating the persistence task to the QueryManager, which executes the
corresponding INSERT operation on the DBMS.

Once the database confirms the insertion and returns the generated path identifier,
the PathManager sends the result back to the API Gateway.
The gateway responds with a 201 Created status and the new pathId. The mobile app then
displays a confirmation message to the user. In the event of network failures or
server-side errors, the app notifies the user accordingly.

\begin{figure}[H]
	\centering
	\includegraphics[width=\textwidth]{Images/Sequence_Diagrams/create_manual.png}
	\caption{Create a Path in Manual Mode Sequence Diagram}
	\label{fig:use_case_create_path_manual}
\end{figure}
\pagebreak

\textbf{[UC7]} - Create a Path in Automatic Mode

When a logged-in user chooses to create a new bike path using, he first selects the
automatic creation mode from the mobile app. The app displays a form for entering
the required metadata, such as the path name, description, and other relevant details.
The app performs a first local validation. If the fields are invalid, the user is immediately notified.

Once the input is valid, the app attempts to activate GPS tracking. If activation fails
(e.g., permissions or hardware issues), an error is shown. If tracking succeeds, the GPS
module begins providing continuous location samples (latitude, longitude, speed, timestamp).
The app collects these values during the entire movement loop.

If the GPS signal temporarily fails while moving, the app detects the error and
interrupts the procedure, informing the user. When the user completes the movement
session, GPS tracking is deactivated and the collected samples undergo a final local
validation; if invalid, the user is notified. When both metadata and sensor samples are
valid, the app sends a POST request to the backend. If a network timeout or server error
occurs, an appropriate message is shown to the user.

The API forwards the request to the PathManager, which calls the QueryManager to store
the path in the DBMS. The system inserts metadata and user association into the database,
and upon successful insertion returns a 201 Created response containing the new pathId.
The mobile app receives the response and displays a success message,
indicating that the new automatically generated path has been saved.

\begin{figure}[H]
	\centering
	\includegraphics[width=\textwidth]{Images/Sequence_Diagrams/create_automatic.png}
	\caption{Create a Path in Automatic Mode Sequence Diagram}
	\label{fig:use_case_create_path_automatic}
\end{figure}
\pagebreak

\textbf{[UC8]} - Delete a Path

When a logged-in user wants to delete one of his path, he firstly navigates to the list of
his created paths in the mobile app. The app sends a request to the backend to retrieve
all paths associated with the user. The API forwards this request to the PathManager,
which retrieves the corresponding records through the QueryManager and the DBMS.
Once the user selects a specific path to delete, the app sends a DELETE request to the backend.

The PathManager first verifies that the path exists and that the requesting user is its
owner. If the ownership check fails, the backend returns a 403 FORBIDDEN error.
If the path does not exist, a 404 NOT\_FOUND response is generated.

When the user is authorised and the path exists, the PathManager performs the deletion
through the QueryManager, which issues the appropriate SQL\ DELETE operation to the DBMS.
Successful deletion results in a 204 response, upon which the mobile app
confirms the removal to the user. Network or server-side failures prompts the mobile app
to display a generic error message.

\begin{figure}[H]
	\centering
	\includegraphics[width=\textwidth]{Images/Sequence_Diagrams/delete_path.png}
	\caption{Delete a Path Sequence Diagram}
	\label{fig:use_case_delete_path}
\end{figure}
\pagebreak

\textbf{[UC9]} - Start a Trip as Guest User

When a guest user wants to start a trip using the BBP mobile app, he first selects a path
from the available options. The app then attempts to activate GPS tracking to monitor
the user's movement along the selected path.

If GPS activation fails, the mobile app immediately notifies the user with an error
message. Otherwise, once tracking is active, the app continuously receives location
updates while the user is moving and refreshes the map accordingly.

If at any point the GPS module reports a loss of signal, the app displays an appropriate
error message to the user. No backend interaction occurs in this use case, as guest
trips are not recorded or stored.

\begin{figure}[H]
	\centering
	\includegraphics[width=\textwidth]{Images/Sequence_Diagrams/start_guest.png}
	\caption{Start a Trip as Guest User Sequence Diagram}
	\label{fig:use_case_start_trip_guest}
\end{figure}
\pagebreak

\textbf{[UC10]} - Start a Trip in Manual Mode as a Logged-in User

In this scenario, a logged-in user starts a trip by selecting a path and enabling manual tracking.
The interaction begins when the user initiates the trip from the mobile app, which
displays the chosen path on the map. The user then selects the Manual mode,
and then the app activates the GPS module.

If the GPS fails to activate, the mobile app immediately notifies the user with an error
message. Otherwise, GPS tracking begins, and the device periodically emits position
updates containing latitude, longitude, speed, and time.
The mobile app stores these samples locally and updates the on-screen map in real time.

During the trip, if the GPS signal is lost at any point, the GPS module reports an error
and the mobile app displays a corresponding warning to the user, interrupting the normal
flow of position updates.

\begin{figure}[H]
	\centering
	\includegraphics[width=\textwidth]{Images/Sequence_Diagrams/start_manual.png}
	\caption{Start a Trip in Manual Mode as a Logged-in User Sequence Diagram}
	\label{fig:use_case_start_trip_manual_logged}
\end{figure}
\pagebreak

\textbf{[UC11]} - Start a Trip in Automatic Mode as a Logged-in user

When a logged-in user selects a path and chooses to start the trip in automatic mode,
the Mobile App first displays the map and then attempts to establish a connection with
the external sensors required for automatic detection.

If sensor activation fails, the app immediately informs the user with an error message.
Otherwise, sensors respond successfully and the app proceeds by enabling GPS tracking.
Once tracking is active, the GPS Module periodically sends position updates which the app
stores locally and reflects on the UI Map.

During the trip, if the GPS signal is lost, the system interrupts the loop and displays
an appropriate error message. If instead the connection to the external sensors drops
during the session, the app stops the automatic process and notifies the user of the
failure.

\begin{figure}[H]
	\centering
	\includegraphics[width=\textwidth]{Images/Sequence_Diagrams/start_automatic.png}
	\caption{Start a Trip in Automatic Mode as a Logged-in User Sequence Diagram}
	\label{fig:use_case_start_trip_automatic_logged}
\end{figure}
\pagebreak

\textbf{[UC12]} - Stop a Trip as Guest User

\begin{figure}[H]
	\centering
	\includegraphics[width=0.7\textwidth]{Images/Sequence_Diagrams/stop_guest.png}
	\caption{Stop a Trip as a Guest User Sequence Diagram}
	\label{fig:use_case_stop_trip_guest}
\end{figure}
\pagebreak

\textbf{[UC13]} - Stop a Trip as a Logged-in user

\begin{figure}[H]
	\centering
	\includegraphics[width=\textwidth]{Images/Sequence_Diagrams/stop_loggedin.png}
	\caption{Stop a Trip as a Logged-in User Sequence Diagram}
	\label{fig:use_case_stop_trip_loggedin}
\end{figure}
\pagebreak

\textbf{[UC14]} - Make a Report in Manual Mode

\begin{figure}[H]
	\centering
	\includegraphics[width=\textwidth]{Images/Sequence_Diagrams/report_manual.png}
	\caption{Make a Report in Manual Mode Sequence Diagram}
	\label{fig:use_case_report_problem_manual}
\end{figure}
\pagebreak

\textbf{[UC15]} - Make a Report in Automatic Mode

\begin{figure}[H]
	\centering
	\includegraphics[width=\textwidth]{Images/Sequence_Diagrams/report_automatic.png}
	\caption{Make a Report in Automatic Mode Sequence Diagram}
	\label{fig:use_case_report_problem_automatic}
\end{figure}
\pagebreak

\textbf{[UC16]} - Confirm a report

\begin{figure}[H]
	\centering
	\includegraphics[width=\textwidth]{Images/Sequence_Diagrams/report_confirm.png}
	\caption{Confirm a Report Sequence Diagram}
	\label{fig:use_case_confirm_other_reports}
\end{figure}
\pagebreak

\textbf{[UC17]} - Manage Path visibility

\begin{figure}[H]
	\centering
	\includegraphics[width=\textwidth]{Images/Sequence_Diagrams/manage_visibility.png}
	\caption{Manage Path Visibility Sequence Diagram}
	\label{fig:use_case_manage_visibility}
\end{figure}
\pagebreak

\textbf{[UC18]} - View Trip History and Trip Details

\begin{figure}[H]
	\centering
	\includegraphics[width=\textwidth]{Images/Sequence_Diagrams/trip_history.png}
	\caption{View Trip History and Trip Details Sequence Diagram}
	\label{fig:use_case_view_trip_history}
\end{figure}
\pagebreak

\textbf{[UC19]} - View Overall Statistics

\begin{figure}[H]
	\centering
	\includegraphics[width=\textwidth]{Images/Sequence_Diagrams/stats_overall.png}
	\caption{View Overall Statistics Sequence Diagram}
	\label{fig:use_case_view_overall_statistics}
\end{figure}
\pagebreak

\textbf{[UC20]} - View Trip Statistics

\begin{figure}[H]
	\centering
	\includegraphics[width=\textwidth]{Images/Sequence_Diagrams/stats_trip.png}
	\caption{View Trip Statistics Sequence Diagram}
	\label{fig:use_case_view_trip_statistics}
\end{figure}
\pagebreak

\textbf{[UC21]} - Edit Personal Profile

\begin{figure}[H]
	\centering
	\includegraphics[width=\textwidth]{Images/Sequence_Diagrams/edit_profile.png}
	\caption{Edit Personal Profile Sequence Diagram}
	\label{fig:use_case_edit_profile}
\end{figure}
\pagebreak

% --------------------------------------------------------------------------
% Component Interfaces
% --------------------------------------------------------------------------

\section{Component Interfaces}
\label{sec:component_interfaces}%

% --------------------------------------------------------------------------
% Architectural Styles and Patterns
% --------------------------------------------------------------------------

\section{Selected architectural styles and patterns}
\label{sec:arch_patterns}%

% --------------------------------------------------------------------------
% Other Design Decisions
% --------------------------------------------------------------------------

\section{Other Design Decisions}
\label{sec:other_design_decisions}%