% --------------------------------------------------------------------------
% Architectural Design
% --------------------------------------------------------------------------

% --------------------------------------------------------------------------
% Overview
% --------------------------------------------------------------------------
\section{Overview: High-level components and their interactions}
\label{sec:overview}%
\section{Architectural Overview}
\label{sec:architectural_overview}

The architecture of the Best Bike Paths (BBP) system follows a classic
\textbf{three-tier structure}, separating the software into a presentation layer,
an app layer, and a data layer. This architectural style ensures a clear
division of responsibilities and simplifies the evolution of the system over time.
Since BBP is conceived as a mobile-centric platform, the presentation tier is
implemented entirely through the BBP mobile app, which communicates with
the backend via RESTful APIs over HTTPS.

\subsection*{Presentation Layer}

The presentation layer consists solely of the \textbf{BBP mobile app},
which serves as the primary interface between users and the system.
This layer is responsible for:
\begin{itemize}
	\item rendering the user interface and handling user interactions;
	\item acquiring device-level data (GPS, accelerometer, gyroscope) during trips;
	\item displaying bike paths, trip summaries, statistics, and reports;
	\item invoking backend functionalities through HTTP requests.
\end{itemize}

The mobile app is intentionally designed as a \textbf{thin client}.
All domain logic, decision processes, ranking operations, and aggregation of path
information are delegated to the app layer. 
The app interacts with device-level subsystems such as the GPS module
and external sensors (when available), but these elements are not part of the backend architecture.
The mobile app also uses the secure storage facilities provided by the operating system 
(iOS Keychain / Android Keystore) to safely store authentication tokens and other sensitive data.

\subsection*{Application Layer}

The app layer embodies the \textbf{core business logic} of BBP.
It is implemented as a modular backend composed of independent yet cooperating \textbf{components},
each encapsulating a well–defined responsibility.
These components are logically independent in terms of responsibilities and interfaces,
but they are part of a single backend app.
The main functional components include:

\begin{itemize}
	\item \textbf{User Module}: contains the \textbf{AuthManager} and the
	      \textbf{UserManager}, responsible for authentication, credential verification,
	      and management of user profiles.
	\item \textbf{Trip Module}: contains the \textbf{TripManager}, which handles
	      the lifecycle of a cycling trip and produces trip summaries enriched with
	      contextual weather data.
	\item \textbf{Path Module}: contains the \textbf{PathManager}, responsible for
	      maintaining bike-path data, computing routes, and ranking candidate paths according to
	      their condition and effectiveness..
	\item \textbf{Report Module}: contains the \textbf{ReportManager}, responsible
	      for storing and aggregating reports, managing confirmations, and updating
	      path-condition indicators.
	\item \textbf{Statistics Module}: contains the \textbf{StatsManager}, which
	      computes and stores user statistics and per-trip metrics.
\end{itemize}

The app layer also includes the \textbf{WeatherManager}, which interacts
with an external weather service to retrieve meteorological data.
All modules are accessed through the \textbf{API Gateway}, which exposes a set of
RESTful sub-APIs and routes incoming requests to the appropriate Manager.

\subsection*{Data Layer}

The data layer consists of a \textbf{relational DBMS} storing all persistent
information relevant to the system’s domain, including:
\begin{itemize}
	\item user profiles and authentication credentials;
	\item trip records and associated GPS data;
	\item bike path segments and their aggregated conditions;
	\item reports, confirmations, and metadata about obstacles;
	\item computed statistics and weather snapshot.
\end{itemize}

All interactions with the DBMS are mediated by a single \textbf{QueryManager},
which centralises data-access operations and offers a uniform interface for
executing queries. This design keeps persistence concerns separated from the
app logic and reduces duplication across components.

% --------------------------------------------------------------------------
% Component View
% --------------------------------------------------------------------------
\section{Component View}
\label{sec:component_view}

This section describes the main software components that constitute the BBP backend
and their responsibilities. As required by the three-tier architecture adopted by the system,
the backend is structured into a set of independent yet cooperating modules, each exposing
well-defined interfaces and encapsulating a cohesive subset of the app logic.

\figure[H]
\centering
\includegraphics[width=\textwidth]{Images/Architectural_Overview/component_view.png}
\caption{Component View Diagram}
\label{fig:component_view}
\endfigure

\subsubsection{API Gateway}
The \textbf{API Gateway} acts as the entry point for all interactions between the BBP mobile
app and the backend. It is responsible for routing incoming requests to the appropriate
internal services, enforcing authentication and authorization requirements, validating inputs,
and translating domain errors into HTTP responses.

The API Gateway exposes the following logical sub-APIs:
\begin{itemize}
	\item \textbf{AuthAPI}: endpoints for token generation and validation.
	\item \textbf{UserAPI}: endpoints for registration, login, token refresh, and profile retrieval.
	\item \textbf{TripAPI}: endpoints for starting, updating, and stopping a trip.
	\item \textbf{PathAPI}: endpoints for route computation and retrieval of path metadata.
	\item \textbf{ReportAPI}: endpoints for creating and confirming obstacle reports.
	\item \textbf{StatisticsAPI}: endpoints for retrieving per-trip and aggregated statistics.
\end{itemize}

\subsubsection{User Module}
The \textbf{User Module} groups two Managers:
\begin{itemize}
	\item \textbf{AuthManager}, responsible for authentication and token issuance.
	\item \textbf{UserManager}, responsible for registration, profile updates,
	      and credential-related operations.
\end{itemize}
Both Managers use the \textbf{QueryManager} for data retrieval and persistence.

\subsubsection{Trip Manager}
The \textbf{TripManager} manages the entire
lifecycle of a cycling session. It receives GPS data from the mobile app, records
the user’s trajectory, and stores trip metadata such as timestamps, duration, distance, and average speed.
When a trip is completed, the component generates its summary and associates it with
relevant environmental data retrieved from the Weather Manager.
The component relies on the \textbf{QueryManager} for storage and communicates
with the Weather Manager to obtain contextual weather information.

\subsubsection{Path Manager}
The \textbf{PathManager} is responsible for retrieving graph
data from the database, computing optimal routes between two locations,
and ranking alternative routes according to their quality and reported conditions.
This service exposes the routing logic to the API Gateway and interacts with the
\textbf{QueryManager} to retrieve and update path and report information needed for route computation.

\subsubsection{Reports Manager}
The \textbf{Reports Manager} handles obstacle reports submitted
by users or automatically detected during trips. It stores and aggregates reports, manages
confirmation and rejection flows, updates path-quality indicators, and exposes relevant data to the mobile app.
This component interacts with the \textbf{QueryManager} for persistence and with the
Routing \& Path Manager when updated conditions affect route evaluation.

\subsubsection{Statistics Manager}
The \textbf{Statistics Manager} computes
and retrieves aggregated cycling statistics. It retrieves historical data through
the \textbf{QueryManager}.

\subsubsection{Weather Manager}
The \textbf{Weather Manager} interacts with the external weather API to obtain
meteorological information. It provides a weather snapshot associated with trip start and end points.
It stores the weather snapshot using the \textbf{QueryManager}.

\subsubsection{QueryManager}
The \textbf{QueryManager} is the data-access component of the backend.
It acts as the single entry point for interacting with the relational DBMS and
provides a set of methods to retrieve, insert, update, and delete domain data.
Centralising data access in one component simplifies consistency checks, reduces duplicated logic, and
keeps the domain layer independent of database details.

% --------------------------------------------------------------------------
% Deployment View
% --------------------------------------------------------------------------

\section{Deployment View}
\label{sec:deployment_view}%

\begin{figure}[H]
    \centering
    \includegraphics[width=0.9\textwidth]{../Images/deployment_view/deployment_view.png}
    \caption{Deployment View of the BBP System}
    \label{fig:deployment_view}
\end{figure}

The deployment view describes the hardware and software infrastructure supporting the \textit{BBP} system. Each tier is 
executed on dedicated hardware nodes and communicates with the others using secure protocols.

\begin{itemize}
    \item \textbf{Presentation Tier:} 
    this tier includes all devices through which end users interact with the BBP system. 
    The primary clients are smartphones or tablets running iOS or Android, where the BBP mobile application is installed. 
    These devices communicate with the backend exclusively via HTTPS, ensuring confidentiality and data integrity. 
    Although the mobile app represents the main access point, any device equipped with a modern web browser and a stable 
    internet connection could technically interact with the system, 
    since the backend exposes standard RESTful endpoints. 
    No application logic is executed at this level, the devices simply capture user input, display results, and forward 
    authenticated HTTP requests toward the Application Tier.

    \item \textbf{Application Tier:}
    this tier is responsible for handling incoming traffic, enforcing security, applying routing and load balancing policies, and 
    executing the core business logic of the system. 
    All external requests first pass through a Linux-based server configured as a Web Application Firewall using \textit{ModSecurity}.  
    ModSecurity blocks malicious traffic such as SQL injection attempts, cross-site scripting payloads, and abnormal request patterns,
     while supporting anomaly detection.  
    Validated HTTPS traffic is then forwarded to the gateway node running \textit{Traefik}, which acts as the system’s 
    reverse proxy and load balancer.  
    Traefik terminates HTTPS connections, exposes a single public entrypoint for clients, and distributes incoming requests 
    across multiple backend replicas using load balancing strategies.  
    Additional middleware functionalities (request logging or rate limiting) can be applied as needed at this level.
    The backend application itself is executed on one or more stateless Application Servers, each running the BBP RESTful backend.  
    Since authentication tokens are included in each request header, no server-side session state is maintained, enabling horizontal 
    scaling and dynamic replica management.  
    Communication between Traefik and the backend replicas is confined to a protected internal network, reducing the system’s 
    exposure to external threats.

    \item \textbf{Data Tier:}
    the Data Tier consists of a dedicated server running a PostgreSQL database instance, which stores all persistent system data 
    such as user information, paths, trips, and analytics.  
    Backend servers interact with the database using standard PostgreSQL drivers over TCP/IP within an isolated internal network 
    segment.  
    Centralizing the database simplifies backup strategies, consistency enforcement, and maintenance operations, while still 
    allowing for potential future extensions such as replication or clustering without altering the upper tiers of the system.

\end{itemize}

% --------------------------------------------------------------------------
% Runtime View
% --------------------------------------------------------------------------

\section{Runtime View}
\label{sec:runtime_view}%

The Runtime View describes how the components of the BBP system collaborate
to realise the behaviour specified in the functional requirements.
While the Component View focuses on the static organisation of the backend
(\textbf{API Gateway}, \textbf{Managers}, \textbf{QueryManager}) and their responsibilities,
the Runtime View illustrates how these components interact dynamically
during the execution of the main use cases.

All diagrams follow the modular structure of the backend:
the mobile client invokes the \textbf{API Gateway}, which routes each request to the
appropriate \textbf{Manager}.
Persistence operations are centralised in the \textbf{QueryManager}, whereas
weather-related data requests involve the \textbf{WeatherManager} and its external API.

The following pages report the sequence diagrams for all core use cases,
from user authentication to trip management, path exploration, reporting,
statistics retrieval, and profile updates.
Together, these diagrams provide a comprehensive understanding of how the BBP
system behaves during execution and how responsibilities are distributed among
its components.

\subsubsection*{[UC1] - User Registration}

A new user wants to register into the BBP system.
The process begins on the mobile app, where the guest user opens the registration page,
and fills in the required details: email, password, and personal information.
A first \textbf{local validation} step checks for malformed inputs before contacting the \textbf{backend}.

If the data is valid, the mobile app sends a registration request to the \textbf{API Gateway},
which forwards it to the \textbf{AuthManager}. This component checks that the email is not
already in use by querying the database through the \textbf{QueryManager}.
If the email already exists, the system returns an error and the mobile app notifies the user accordingly.

If the email does not exist, the \textbf{AuthManager} inserts the new user into the database
and generates an \textbf{authentication token}. The token is returned to the mobile app, which
stores it securely using the device’s \textbf{secure storage} mechanism. The flow concludes with a
success message being shown to the user.

\begin{figure}[H]
	\centering
	\includegraphics[width=\textwidth]{Images/Sequence_Diagrams/user_registration.png}
	\caption{User Registration Sequence Diagram}
	\label{fig:use_case_user_registration}
\end{figure}
\pagebreak

\subsubsection*{[UC2] - User Log In}

A guest user wants to authenticate and obtain access to the system. The process starts
when the user opens the login form and submits credentials through the mobile app. After
a \textbf{local validation}, the app sends an HTTP request to the login endpoint exposed by the \textbf{API Gateway}.

The \textbf{API Gateway} forwards the request to the \textbf{AuthManager}, which first checks whether the provided
email exists by querying the \textbf{DBMS} through the \textbf{QueryManager}. If the email is not found,
the backend returns a \textbf{404 Not Found} error, which the mobile app displays to the user.
If the user exists, the \textbf{AuthManager} verifies the submitted password. Invalid credentials
lead to a \textbf{401 Unauthorized} response and the corresponding error message on the client.

When the credentials are correct, the \textbf{AuthManager} generates an \textbf{authentication token} and
returns it to the mobile app, which stores it securely using the device's
\textbf{secure storage} facility. The user is then successfully logged in and the app
proceeds to show the appropriate authenticated UI.


\begin{figure}[H]
	\centering
	\includegraphics[width=\textwidth]{Images/Sequence_Diagrams/user_login.png}
	\caption{User Log In Sequence Diagram}
	\label{fig:use_case_user_login}
\end{figure}
\pagebreak

\subsubsection*{[UC3] - User Log Out}

When a logged-in user initiates the logout operation from the mobile app, the client first clears
the locally stored authentication token from the \textbf{secure storage}.
Afterward, the mobile app sends an HTTP request to the backend.

The \textbf{API Gateway} forwards the request to the \textbf{AuthManager}, which handles the logout process by deleting
the corresponding \textbf{refresh token} through the \textbf{QueryManager}. The QueryManager executes a \textbf{DELETE}
operation on the database to invalidate the stored refresh token.

If the operation succeeds, the server returns a \textbf{204 NO\_CONTENT} response, and the app
displays a success message to the user. If instead a network or server error occurs,
the system interrupts the flow and the mobile app notifies the user with a “Network unavailable” error message.

\begin{figure}[H]
	\centering
	\includegraphics[width=\textwidth]{Images/Sequence_Diagrams/user_logout.png}
	\caption{User Log Out Sequence Diagram}
	\label{fig:use_case_user_logout}
\end{figure}
\pagebreak

\subsubsection*{[UC4] - Search for a Path}

A user can search for bike paths between two locations. He enters the start and end points
and submits the request. The app performs a preliminary local validation
and, if the data is valid, forwards the request to the backend through the \textbf{API Gateway}.

The \textbf{API Gateway} forwards the request to the \textbf{PathManager}, which loads the relevant
portion of the path graph from the database using the \textbf{QueryManager}. Once the graph
data is retrieved, the \textbf{PathManager} computes the optimal path(s) according to the
requested constraints. If valid routes are found, the API Gateway returns them to
the mobile app, which displays the corresponding suggestions.

If no route satisfies the user’s constraints, the \textbf{PathManager} signals a \textbf{NO\_ROUTE}
condition, which results in a 404 error. In the case of network or server issues,
the app notifies the user with an appropriate error message.

\begin{figure}[H]
	\centering
	\includegraphics[width=\textwidth]{Images/Sequence_Diagrams/search_path.png}
	\caption{Search for a Path Sequence Diagram}
	\label{fig:use_case_search_path}
\end{figure}
\pagebreak

\subsubsection*{[UC5] - Select a Path}

The Select a Path sequence diagram illustrates how the system retrieves the details of
a path selected by a user. The interaction begins when the user chooses a specific path
from the list of suggested routes displayed by the mobile app. The app sends a request
to the backend through the \textbf{API Gateway}, which forwards it to the \textbf{PathManager}.
The \textbf{PathManager} retrieves the corresponding path information by querying the database
through the \textbf{QueryManager}. If a matching record is found, the PathManager returns the
path details to the \textbf{API Gateway}, which sends them back to the mobile app for
presentation to the user.
If the database does not contain a path with the specified identifier, the \textbf{PathManager}
signals a \textbf{NOT\_FOUND} condition, resulting in a \textbf{404} response. In that case, the mobile app
notifies the user that the selected route is unavailable. As in other interactions,
network or server errors trigger an appropriate error message on the client side.

\begin{figure}[H]
	\centering
	\includegraphics[width=0.95\textwidth]{Images/Sequence_Diagrams/select_path.png}
	\caption{Select a Path Sequence Diagram}
	\label{fig:use_case_select_path}
\end{figure}
\pagebreak

\subsubsection*{[UC6] - Create a Path in Manual mode}

When a logged-in user wants to create a new path, he will be asked to choose between
manual and automatic creation modes. If he opts for creating the path in \textbf{manual mode},
he will be presented with a form to fill in the required metadata and segment list.

Before sending the request, the app performs local validation to ensure that all
mandatory fields and segments are correctly specified.

If the input is valid, the mobile app submits the creation request to the
backend through the \textbf{API Gateway}. The gateway forwards the request to the \textbf{PathManager},
which is responsible for handling the creation workflow. The PathManager stores the new
path by delegating the persistence task to the \textbf{QueryManager}, which executes the
corresponding \textbf{INSERT} operation on the \textbf{DBMS}.

Once the database confirms the insertion,
the \textbf{PathManager} sends the result back to the \textbf{API Gateway}.
The gateway responds with a \textbf{201 Created} status and the new pathId. The mobile app then
displays a confirmation message to the user. In the event of network failures or
server-side errors, the app notifies the user accordingly.


\begin{figure}[H]
	\centering
	\includegraphics[width=\textwidth]{Images/Sequence_Diagrams/create_manual.png}
	\caption{Create a Path in Manual Mode Sequence Diagram}
	\label{fig:use_case_create_path_manual}
\end{figure}
\pagebreak

\subsubsection*{[UC7] - Create a Path in Automatic Mode}

When a logged-in user chooses to create a new bike path, he first selects the
\textbf{automatic creation mode} from the mobile app. The app displays a form for entering
the required metadata, such as the path name, description, and other relevant details.
The app performs a first local validation. If the fields are invalid, the user is immediately notified.

Once the input is valid, the app attempts to activate \textbf{GPS tracking}. If activation fails
(e.g., permissions or hardware issues), an error is shown. If tracking succeeds, the \textbf{GPS
module} begins providing continuous location samples (latitude, longitude, speed, timestamp).
The app collects these values during the entire movement loop.

If the GPS signal temporarily fails while moving, the app detects the error and
interrupts the procedure, informing the user. When the user completes the movement
session, GPS tracking is deactivated and the collected samples undergo a final \textbf{local
validation}; if invalid, the user is notified. When both metadata and sensor samples are
valid, the app sends a \textbf{POST request} to the backend. If a network timeout or server error
occurs, an appropriate message is shown to the user.

The \textbf{API} forwards the request to the \textbf{PathManager}, which calls the \textbf{QueryManager} to store
the path in the \textbf{DBMS}. The system inserts metadata and user association into the database,
and upon successful insertion returns a \textbf{201 Created} response containing the new pathId.
The mobile app receives the response and displays a success message,
indicating that the new automatically generated path has been saved.


\begin{figure}[H]
	\centering
	\includegraphics[width=\textwidth]{Images/Sequence_Diagrams/create_automatic.png}
	\caption{Create a Path in Automatic Mode Sequence Diagram}
	\label{fig:use_case_create_path_automatic}
\end{figure}
\pagebreak

\subsubsection*{[UC8] - Delete a Path}

When a logged-in user wants to delete one of his path, he firstly navigates to the list of
his created paths in the mobile app. The app sends a request to the backend to retrieve
all paths associated with the user. The \textbf{API} forwards this request to the \textbf{PathManager},
which retrieves the corresponding records through the \textbf{QueryManager} and the \textbf{DBMS}.
Once the user selects a specific path to delete, the app sends a \textbf{DELETE request} to the backend.

The \textbf{PathManager} first verifies that the path exists and that the requesting user is its
owner. If the ownership check fails, the backend returns a \textbf{403 FORBIDDEN} error.
If the path does not exist, a \textbf{404 NOT\_FOUND} response is generated.

When the user is authorised and the path exists, the \textbf{PathManager} performs the deletion
through the \textbf{QueryManager}, which issues the appropriate \textbf{SQL DELETE} operation to the \textbf{DBMS}.
Successful deletion results in a \textbf{204} response, upon which the mobile app
confirms the removal to the user. Network or server-side failures prompt the mobile app
to display a generic error message.


\begin{figure}[H]
	\centering
	\includegraphics[width=\textwidth]{Images/Sequence_Diagrams/delete_path.png}
	\caption{Delete a Path Sequence Diagram}
	\label{fig:use_case_delete_path}
\end{figure}
\pagebreak

\subsubsection*{[UC9] - Start a Trip as Guest User}

When a guest user wants to start a trip using the BBP mobile app, he first selects a path
from the available options. The app then attempts to activate \textbf{GPS tracking} to monitor
the user's movement along the selected path.

If GPS activation fails, the mobile app immediately notifies the user with an error
message. Otherwise, once tracking is active, the app continuously receives location
updates while the user is moving and refreshes the map accordingly.

If at any point the \textbf{GPS module} reports a loss of signal, the app displays an appropriate
error message to the user. No backend interaction occurs in this use case, as guest
trips are not recorded or stored.


\begin{figure}[H]
	\centering
	\includegraphics[width=\textwidth]{Images/Sequence_Diagrams/start_guest.png}
	\caption{Start a Trip as Guest User Sequence Diagram}
	\label{fig:use_case_start_trip_guest}
\end{figure}
\pagebreak

\subsubsection*{[UC10] - Start a Trip in Manual Mode as a Logged-in User}

In this scenario, a logged-in user starts a trip by selecting a path and enabling \textbf{manual tracking}.
The interaction begins when the user initiates the trip from the mobile app, which
displays the chosen path on the map. The user then selects the \textbf{Manual mode},
and then the app activates the \textbf{GPS module}.
If the GPS fails to activate, the mobile app immediately notifies the user with an error
message. Otherwise, GPS tracking begins, and the device periodically emits position
updates containing latitude, longitude, speed, and time.
The mobile app stores these samples locally and updates the on-screen map in real time
During the trip, if the GPS signal is lost at any point, the GPS module reports an error
and the mobile app displays a corresponding warning to the user, interrupting the normal
flow of position updates.


\begin{figure}[H]
	\centering
	\includegraphics[width=\textwidth]{Images/Sequence_Diagrams/start_manual.png}
	\caption{Start a Trip in Manual Mode as a Logged-in User Sequence Diagram}
	\label{fig:use_case_start_trip_manual_logged}
\end{figure}
\pagebreak

\subsubsection*{[UC11] - Start a Trip in Automatic Mode as a Logged-in user}

When a logged-in user selects a path and chooses to start the trip in \textbf{automatic mode},
the mobile app first displays the map and then attempts to establish a connection with
the \textbf{external sensors} required for automatic detection.

If sensor activation fails, the app immediately informs the user with an error message.
Otherwise, sensors respond successfully and the app proceeds by enabling GPS tracking.
Once tracking is active, the \textbf{GPS Module} periodically sends position updates which the app
stores locally and reflects on the UI map.

During the trip, if the GPS signal is lost, the system interrupts the loop and displays
an appropriate error message. If instead the connection to the external sensors drops
during the session, the app stops the automatic process and notifies the user of the
failure.

\begin{figure}[H]
	\centering
	\includegraphics[width=\textwidth]{Images/Sequence_Diagrams/start_automatic.png}
	\caption{Start a Trip in Automatic Mode as a Logged-in User Sequence Diagram}
	\label{fig:use_case_start_trip_automatic_logged}
\end{figure}
\pagebreak

\subsubsection*{[UC12] - Stop a Trip as Guest User}

This sequence diagram describes how a guest user stops an ongoing trip.
The interaction is entirely local, as guest users do not store trip data on the backend.

The process begins when the user selects the \textbf{Stop Trip} action.
The mobile app then requests the \textbf{GPS module} to deactivate tracking.
Once the GPS confirms that tracking has been successfully stopped,
the app terminates the trip visualisation and returns the user to the map or home screen.

No network communication is involved, and no data is persisted,
making this use case lightweight and fully handled on the device.

\begin{figure}[H]
	\centering
	\includegraphics[width=0.7\textwidth]{Images/Sequence_Diagrams/stop_guest.png}
	\caption{Stop a Trip as a Guest User Sequence Diagram}
	\label{fig:use_case_stop_trip_guest}
\end{figure}
\pagebreak

\subsubsection*{[UC13] - Stop a Trip as a Logged-in User}

When the user wants to stop an ongoing trip, he selects the Stop Trip action
from the mobile app. Upon this action, the mobile app terminates any
active data acquisition (GPS tracking and, if the selected mode is Automatic,
external sensor streams) and sends a stop-trip request to the backend. The
\textbf{TripManager} validates the request and retrieves the corresponding trip
record through the \textbf{QueryManager}. If the request is invalid, for example, if
the trip does not exist or is already closed, the system returns an appropriate error.

If validation succeeds, the \textbf{TripManager} contacts the
\textbf{WeatherManager} to obtain contextual weather information.
This step is best-effort: if the external weather service is reachable and
returns valid data, the weather snapshot is included in the summary; otherwise,
the summary is generated without weather information. The \textbf{TripManager} then
computes the final trip summary, including distance, duration, speed metrics, sensor-derived
data (when available), and the associated weather snapshot. The summary is then
saved through the \textbf{QueryManager}.

The backend responds with a \textbf{201 Created} status and the complete summary.
The mobile app then displays the result to the user.
Network failures, invalid identifiers, or missing trip records trigger the
corresponding alternative flows.

\begin{figure}[H]
	\centering
	\includegraphics[width=\textwidth]{Images/Sequence_Diagrams/stop_loggedin.png}
	\caption{Stop a Trip as a Logged-in User Sequence Diagram}
	\label{fig:use_case_stop_trip_loggedin}
\end{figure}
\pagebreak

\subsubsection*{[UC14] - Make a Report in Manual Mode}

The logged-in user selects a path segment on the map and opens the report-creation form.
The mobile app retrieves the user's current GPS position. If the position cannot be
retrieved, the app immediately shows an error.

After the user submits the form containing the report description and selected
options, the mobile app performs local validation. Invalid inputs cause the app
to show an error without contacting the backend.

If validation succeeds, the app sends the report payload to the backend through
the \textbf{API Gateway}. Network or server-side failures lead to a timeout and an error
message shown to the user.

Once the request is received, the \textbf{API Gateway} forwards the data to the
\textbf{ReportManager}, which creates a report record by storing the user ID,
position, and payload in the database through the \textbf{QueryManager}.
If the insertion succeeds, the \textbf{DBMS} returns the generated report identifier.

Finally, the backend responds with \textbf{201 Created}, and the mobile app displays
a confirmation message to the user.

\begin{figure}[H]
	\centering
	\includegraphics[width=\textwidth]{Images/Sequence_Diagrams/report_manual.png}
	\caption{Make a Report in Manual Mode Sequence Diagram}
	\label{fig:use_case_report_problem_manual}
\end{figure}
\pagebreak

\subsubsection*{[UC15] - Make a Report in Automatic Mode}

When an obstacle is detected by the external sensors during a trip, the mobile app
retrieves the current GPS position from the device. If the retrieval fails, the app
displays an appropriate error message and the flow terminates.

Once the position is available, the app displays a pre-filled report form containing
the detected issue. The user can review and modify the report details before submission.
After the user confirms the report, the app performs local validation on the
generated data. Invalid or incomplete payloads trigger a local error message and no
request is sent to the backend.

If validation succeeds, the mobile app sends the report payload to the backend
via the \textbf{API Gateway}. Network or server errors result in a timeout, causing the app to show a
network-unavailable message.

Upon receiving a valid request, the \textbf{ReportManager} creates a new report
record, storing it through the \textbf{QueryManager}, which inserts the new record
into the database.

After successful insertion, the backend returns a \textbf{201 Created} response.
The mobile app then informs the user that the report has been submitted successfully.

\begin{figure}[H]
	\centering
	\includegraphics[width=\textwidth]{Images/Sequence_Diagrams/report_automatic.png}
	\caption{Make a Report in Automatic Mode Sequence Diagram}
	\label{fig:use_case_report_problem_automatic}
\end{figure}
\pagebreak

\subsubsection*{[UC16] - Confirm a Report}

A logged-in user is on a trip, and a pop-up notification informs him of the presence of
an existing report nearby. The user can choose to confirm or reject it.
After the user submits the decision, the mobile app validates the input 
locally and, if valid, sends a request to the \textbf{backend API}.

The API forwards the request to the \textbf{ReportManager}, which
creates a new confirmation entry associated with the selected report
and the current user. The \textbf{ReportManager} delegates the
persistence of this confirmation to the \textbf{QueryManager}, which
inserts the corresponding record into the database.

If the operation succeeds, the API responds with a \textbf{201
Created} status and returns the confirmation identifier. The mobile
app notifies the user that the confirmation has been
submitted successfully.
The user may also dismiss the form without submitting any confirmation.

In case of client-side validation errors, network failures, or server-side issues,
the mobile app displays the appropriate error message.

\begin{figure}[H]
	\centering
	\includegraphics[width=\textwidth]{Images/Sequence_Diagrams/report_confirm.png}
	\caption{Confirm a Report Sequence Diagram}
	\label{fig:use_case_confirm_other_reports}
\end{figure}
\pagebreak

\subsubsection*{[UC17] - Manage Path Visibility}

The user selects the desired path from the list of previously created paths.
Then, the mobile application sends a request to the backend to retrieve the current 
visibility settings of the selected path. The \textbf{PathManager}, through the \textbf{QueryManager}, 
loads the corresponding path record from the database.

If the path is successfully found, the app displays the existing visibility configuration
and waits for the user to submit the desired changes. Once the user confirms the update,
the mobile application sends a request containing the updated visibility value.
Upon receiving it, the \textbf{PathManager} verifies that the requesting user is indeed the owner
of the path. If the ownership constraint is satisfied, the visibility
attribute is updated in the database. A successful update triggers a confirmation
response, and the mobile application communicates the result to the user.

If the initial fetch request fails due to network or server issues,
an error message is displayed. If the selected path does not exist or no record is returned from the database,
the application notifies the user accordingly. If the user attempts to modify a path they do not own,
the backend returns a \textbf{403 FORBIDDEN} response, and the app signals that the operation is not permitted.

\begin{figure}[H]
	\centering
	\includegraphics[width=\textwidth]{Images/Sequence_Diagrams/manage_visibility.png}
	\caption{Manage Path Visibility Sequence Diagram}
	\label{fig:use_case_manage_visibility}
\end{figure}
\pagebreak

\subsubsection*{[UC18] - View Trip History and Trip Details}

A logged-in user requests to view their trip history from the mobile app. 
The mobile app sends a request to the \textbf{API Gateway}.
If the request succeeds, the \textbf{API} delegates the operation to the \textbf{TripManager}, 
which retrieves the list of the user’s trips through the \textbf{QueryManager}. 
The \textbf{DBMS} returns the corresponding records, and the App displays the resulting history.

When the user selects a specific trip, the app issues a request.
If the trip is missing or does not belong to the user, the \textbf{TripManager}
returns a \textbf{404 Not Found}, which the client displays accordingly.

If the trip exists, the \textbf{TripManager} loads full details from the \textbf{DBMS},
including timestamps, distance, speed metrics, the associated path, and any storedweather snapshots. 
The details are returned to the App, which presents a complete summary of the selected trip.

In case of any network or server failures, the app notifies the user with an appropriate error message.

\begin{figure}[H]
	\centering
	\includegraphics[width=\textwidth]{Images/Sequence_Diagrams/trip_history.png}
	\caption{View Trip History and Trip Details Sequence Diagram}
	\label{fig:use_case_view_trip_history}
\end{figure}
\pagebreak

\subsubsection*{[UC19] - View Overall Statistics}

Upon opening the statistics section, the mobile app requests
the user's overall metrics from the backend. If the request cannot be completed 
due to network or server issues, an error message is shown.  
Otherwise, the \textbf{API} forwards the request to the \textbf{StatsManager},
which loads the user's trip history via the \textbf{QueryManager}. The latter 
retrieves all relevant rows from the \textbf{DBMS}.  

If trip data exists, the \textbf{StatsManager} computes all required aggregates
(e.g., total distance, duration, average speed) and returns the final statistics
to the mobile app, which then displays them. 

If no trip records are found, the system returns a \textbf{404 NOT\_FOUND}
response and the app informs the user accordingly.

\begin{figure}[H]
	\centering
	\includegraphics[width=\textwidth]{Images/Sequence_Diagrams/stats_overall.png}
	\caption{View Overall Statistics Sequence Diagram}
	\label{fig:use_case_view_overall_statistics}
\end{figure}
\pagebreak

\subsubsection*{[UC20] - View Trip Statistics}

When the Logged-in user selects a trip, the mobile app sends a request for detailed metrics.
Network or server failures are handled locally by showing an appropriate error.  

If the request reaches the backend, the \textbf{API} delegates it to the 
\textbf{StatsManager}, which loads the associated trip data through the \textbf{QueryManager}
by querying the \textbf{DBMS}. If the requested trip is found, the \textbf{StatsManager}
computes the aggregated statistics and returns them to the mobile app, 
which presents them to the user. 

If the trip does not exist or does not belong to the user, the backend replies with
a \textbf{404 NOT\_FOUND} response and the app notifies the user.

\begin{figure}[H]
	\centering
	\includegraphics[width=\textwidth]{Images/Sequence_Diagrams/stats_trip.png}
	\caption{View Trip Statistics Sequence Diagram}
	\label{fig:use_case_view_trip_statistics}
\end{figure}
\pagebreak

\subsubsection*{[UC21] - Edit Personal Profile}

After the logged-in user opens the edit form, the mobile app
locally validates the submitted fields. If the data is incomplete or invalid,
the app immediately notifies the user. If the input is valid, the updated
payload is sent to the \textbf{API}, which delegates the request to the \textbf{UserManager}.

The update is forwarded to the \textbf{QueryManager}, which issues the corresponding \textbf{UPDATE} operation on the database.  
If the update succeeds, the modified user profile is returned to the app and displayed to the user. 

In case of network or server errors during the request, the mobile app shows an appropriate error message.

\begin{figure}[H]
	\centering
	\includegraphics[width=\textwidth]{Images/Sequence_Diagrams/edit_profile.png}
	\caption{Edit Personal Profile Sequence Diagram}
	\label{fig:use_case_edit_profile}
\end{figure}
\pagebreak

% --------------------------------------------------------------------------
% Component Interfaces
% --------------------------------------------------------------------------

\section{Component Interfaces}
\label{sec:component_interfaces}%

% --------------------------------------------------------------------------
% Architectural Styles and Patterns
% --------------------------------------------------------------------------

\section{Selected architectural styles and patterns}
\label{sec:arch_patterns}%

% --------------------------------------------------------------------------
% Other Design Decisions
% --------------------------------------------------------------------------

\section{Other Design Decisions}
\label{sec:other_design_decisions}%