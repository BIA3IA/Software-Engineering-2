% --------------------------------------------------------------------------
% Reference Documents
% --------------------------------------------------------------------------

\section{Reference Documents}
\label{sec:refdocs}%

The preparation of this document was supported by the following reference materials:
\begin{itemize}
	\item IEEE Standard for Software Requirement Specifications \cite{ieee};
	\item Assignment specification for the RASD and DD of the Software Engineering II course,
	      held by professors Matteo Rossi, Elisabetta Di Nitto, and Matteo Camilli at the
	      Politecnico di Milano, A.Y. 2025/2026 \cite{dd};
	\item Slides of the Software Engineering 2 course available on WeBeep \cite{slides};
\end{itemize}

% --------------------------------------------------------------------------
% Software Used
% --------------------------------------------------------------------------

\section{Software Used}
\label{sec:software}

The following software tools have been used during the development of this project:

\begin{itemize}
	\item \textbf{Visual Studio Code} : main IDE
	\item \textbf{LaTeX} : typesetting system for documentation
	\item \textbf{Git} : version control system
	\item \textbf{GitHub} : remote repository hosting service
	\item \textbf{Lucidchart} : diagram creation tool
\end{itemize}

% --------------------------------------------------------------------------
% AI Tools Usage
% --------------------------------------------------------------------------

\section{Use of AI Tools}
\label{sec:ai}

AI tools were part of our workflow in a way similar to other software we used during the
project. Rather than producing content autonomously, they helped us reason about how to
present certain sections, reorganize ideas, and make the document more coherent.

We used AI mainly while drafting the text, for example to compare alternative ways of
explaining scenarios, reorganize long paragraphs, or check whether the wording of some
requirements could lead to misinterpretations. In several cases, discussing a passage
with an AI assistant helped us clarify our own understanding of the underlying concept
before writing the final version.

\begin{longtable}{|p{0.32\textwidth}|p{0.63\textwidth}|}
	\hline
	\textbf{Tools Used}          &
	Gemini, ChatGPT
	\\ \hline

	\textbf{Examples of Prompts} &
	\begin{itemize}
		\item “Suggest a clearer way to express this requirement without changing its meaning.”
		\item “Can this sentence be misinterpreted? If yes, propose an alternative phrasing.”
		\item “Provide a more concise version of this paragraph while keeping the content intact.”
		\item “Format the following text in LaTeX.”
		\item “Write the basic layout of a LaTeX table with three columns and a header row.”
		\item “Assist with debugging or formatting issues related to VS Code or LaTeX.”
		\item “Apply the Alloy LaTeX style to this code snippet.”
	\end{itemize}

	\\ \hline

	\textbf{Input Provided}      &
	\begin{itemize}
		\item Early drafts of paragraphs.
		\item Short text fragments requiring clarity checks.
		\item Sections with repeated structure where consistent wording was needed.
	\end{itemize}
	\\ \hline

	\textbf{Constraints Applied} &
	\begin{itemize}
		\item Preserve the intended meaning of the original text.
		\item Avoid introducing new requirements or assumptions.
		\item Maintain terminology aligned with the definitions in Section 1.
	\end{itemize}
	\\ \hline

	\textbf{Outputs Obtained}    &
	\begin{itemize}
		\item Clearer or more concise formulations of existing statements.
		\item Identification of potentially ambiguous sentences.
		\item Terminology suggestions to maintain coherence.
		\item LaTeX formatting assistance for tables and code excerpts.
	\end{itemize}
	\\ \hline

	\textbf{Refinement Process}  &
	\begin{itemize}
		\item Critical review of all AI-generated suggestions.
		\item Verification against the original intent to avoid unintended changes.
		\item Manual integration to ensure consistency with our writing style.
		\item Alignment check with established terminology and definitions.
	\end{itemize}
	\\ \hline
\end{longtable}
