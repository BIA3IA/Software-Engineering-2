% --------------------------------------------------------------------------
% User Interfaces
% --------------------------------------------------------------------------

\newcommand{\phoneimage}[2][]{%
	\begin{tikzpicture}[baseline=(img.base)]
		% --- SCHERMO: la tua immagine, con clip ad angoli arrotondati ---
		\node[
			inner sep=0pt,
			outer sep=0pt,
			rounded corners=14pt, % regola se vuoi più o meno tondo
			clip
		] (img) {\includegraphics[#1]{#2}};

		% --- BORDO ATTORNO ALLO SCHERMO (ben visibile) ---
		\draw[
			rounded corners=14pt,
			line width=0.9pt,
			color=black!65
		] (img.north west) rectangle (img.south east);

		% --- NOTCH SEMPLICE ---
		\path (img.north west) -- (img.north east)
		coordinate[pos=0.5] (topcenter);

		\draw[
			rounded corners=3pt,
			fill=black,
			draw=black
		]
		($(topcenter)+(-0.60cm,-0.16cm)$)
		rectangle
		($(topcenter)+(0.60cm,-0.30cm)$);

		% --- HOME BAR IN BASSO ---
		\path (img.south west) -- (img.south east)
		coordinate[pos=0.5] (bottomcenter);

		\draw[
			rounded corners=2pt,
			fill=white!92,
			draw=none
		]
		($(bottomcenter)+(-0.38cm,0.20cm)$)
		rectangle
		($(bottomcenter)+(0.38cm,0.25cm)$);
	\end{tikzpicture}%
}

\section{User Interfaces}
\label{sec:user_interfaces}%

All mockups included here are conceptual prototypes that illustrate the expected interaction flow,
the structure of each screen, and the behaviour of both logged-in and guest users. While the visual
style is representative of the final layout, the graphical design may undergo refinements during
implementation.

This section provides a high-level visualization of how users interact with the system across its main
features, including authentication, map exploration, path search and selection, trip navigation,
obstacle reporting, custom path creation and management, and profile settings.
The screens also highlight how certain functionalities change depending on the user state
(guest or logged-in), ensuring a clear understanding of the overall user experience.

\section{Authentication and Guest Access}
\label{sec:authentication_and_guest_access}%

\begin{figure}[H]
	\centering
	\begin{minipage}[t]{0.31\textwidth}
		\centering
		\vspace{0pt}
		\phoneimage[width=\textwidth]{Design/1_welcome.png}
		\caption{Welcome Screen}
		\label{fig:welcome_screen_mockup}
	\end{minipage}
	\hfill
	\begin{minipage}[t]{0.31\textwidth}
		\centering
		\vspace{0pt}
		\phoneimage[width=\textwidth]{Design/2_login.png}
		\caption{Login Screen}
		\label{fig:login_screen_mockup}
	\end{minipage}
	\hfill
	\begin{minipage}[t]{0.31\textwidth}
		\centering
		\vspace{0pt}
		\phoneimage[width=\textwidth]{Design/3_signup.png}
		\caption{Signup Screen}
		\label{fig:signup_screen_mockup}
	\end{minipage}
\end{figure}

\subsection*{Welcome Screen}
\label{sec:welcome_screen}%

The welcome screen displays the app logo, the app's name, and a short tagline summarizing
its purpose: discovering bike paths, tracking rides, and joining a cycling community.
Three navigation options are provided: Sign In, Log In, and Continue as Guest.
The Sign In button directs new users to the registration screen to create an account, while
the Log In button allows returning users to access their existing accounts.
The Continue as Guest option enables users to explore the app without registering,
offering limited functionality.
This screen serves as the entry point to all subsequent interactions and does not
require any backend communication. Its goal is to guide the user toward authentication
or guest access in a clear, minimal, and intuitive way.

\subsection*{Login Screen}
\label{sec:login_screen}%

After choosing the Log In option from the Welcome Screen, the user is taken to the Login Screen.
This interface allows returning users to authenticate by entering their email address and password.
Below the login form, users can navigate back to the Welcome Screen.
This screen ensures secure access to all BBP features that require authentication, including trip
recording with data storage, path creation, reporting, and access to statistics.

\subsection*{Signup Screen}
\label{sec:signup_screen}%

The Signup Screen, reachable through the welcome page, enables new users to create an account
by providing a username, email address, password, and password confirmation.
This interface is designed to keep the registration process simple, requiring only
the essential information needed to activate a personal profile.
Users can switch to the Welcome Screen by pressing the dedicated button.
Creating an account unlocks all core functionalities of BBP, such as recording trips,
submitting reports, managing custom paths, and accessing personal statistics.

\section{Home for Logged-in and Guest Users}
\label{sec:home_for_logged_in_and_guest_users}%

\begin{figure}[H]
	\centering
	\begin{minipage}[t]{0.31\textwidth}
		\centering
		\vspace{0pt}
		\phoneimage[width=\textwidth]{Design/4_home.png}
		\caption{Home Screen}
		\label{fig:home_screen_mockup}
	\end{minipage}
	\hfill
	\begin{minipage}[t]{0.31\textwidth}
		\centering
		\vspace{0pt}
		\phoneimage[width=\textwidth]{Design/4_home_guest.png}
		\caption{Home Screen for Guests}
		\label{fig:home_screen_mockup_guest}
	\end{minipage}
	\hfill
	\begin{minipage}[t]{0.31\textwidth}
		\centering
		\vspace{0pt}
		\phoneimage[width=\textwidth]{Design/5_unlock.png}
		\caption{Authentication Pop-up}
		\label{fig:authentication_pop_up_guest_users_mockup}
	\end{minipage}
\end{figure}

\subsection*{Home Screen}
\label{sec:home_screen}%

The Home Screen is the central hub of the BBP application, allowing users to search for bike paths,
view the interactive map, and navigate to all major sections of the app.

At the top of the interface, users can enter a Starting point and a Destination using two text fields.
By default, the starting point is set to the user’s current GPS location, but it can be manually edited
if needed. Below the input fields, the Find Paths button initiates the path search process,
retrieving all available bike routes that connect the selected points.
The map occupies most of the screen, displaying the user's current position and any computed paths.
In the lower-right corner, a floating action button with a plus icon is available for logged-in users.
Pressing it opens the path creation flow, allowing them to create a custom bike path.

At the bottom of the screen, a persistent navigation bar provides access to the main sections of the app:
Home, Trip History, My Paths, and User Profile.

Guest users can fully interact with the map and use the path search form.
However, features that require authentication, such as creating new paths, viewing personal trip history,
or accessing the user profile, remain disabled.
Icons in the bottom navigation bar appear visually greyed out, and a lock symbol indicates restricted access.
Tapping a disabled icon triggers a pop-up inviting the user to sign in to unlock all the app's features.
The creation floating action button is also greyed out, as path creation is restricted to logged-in users.
This interface clearly conveys the distinction between freely accessible features and those requiring authentication.

\subsection*{Authentication Pop-up for Guest Users}
\label{sec:authentication_pop_up_guest_users}%

When a guest user attempts to access a restricted feature, the app displays a modal pop-up overlay
informing them that the selected functionality is available only to authenticated users.
The pop-up contains a short explanation and two action buttons: one to sign in, and one to
continue as a guest without enabling the restricted action.

Behind the pop-up, the rest of the interface is dimmed and temporarily disabled, ensuring that
the user’s attention remains on the modal and preventing interaction with the underlying elements.
The modal can be closed either by selecting one of the available actions or by tapping outside
the pop-up area, returning the guest user to the current screen and allowing them to continue
browsing the map or searching for bike paths uninterrupted.

\section{Path Search}
\label{sec:path_search}%


\begin{figure}[H]
	\centering
	\begin{minipage}[t]{0.31\textwidth}
		\centering
		\vspace{0pt}
		\phoneimage[width=\textwidth]{Design/6_search.png}
		\caption{Search Results}
		\label{fig:search_results_mockup}
	\end{minipage}
	\hspace{0.05\textwidth}
	\begin{minipage}[t]{0.31\textwidth}
		\centering
		\vspace{0pt}
		\phoneimage[width=\textwidth]{Design/6_search_guest.png}
		\caption{Search Results for Guests}
		\label{fig:search_results_guests_mockup}
	\end{minipage}
\end{figure}

\begin{figure}[H]
	\centering
	\begin{minipage}[t]{0.31\textwidth}
		\centering
		\vspace{0pt}
		\phoneimage[width=\textwidth]{Design/22_biking_prompt.png}
		\caption{Automatic Ride Detection}
		\label{fig:automatic_ride_detection_popup_mockup}
	\end{minipage}
	\hspace{0.05\textwidth}
	\begin{minipage}[t]{0.31\textwidth}
		\centering
		\vspace{0pt}
		\phoneimage[width=\textwidth]{Design/22_biking_prompt_guest.png}
		\caption{Automatic Ride Detection for Guests}
		\label{fig:automatic_ride_detection_popup_guest_mockup}
	\end{minipage}
\end{figure}


\subsection*{Search Results}
\label{sec:search_results}%

After the user submits a path search from the home screen, the app displays an “Available Paths”
panel anchored to the top of the map.
The panel lists all suggested paths matching the search criteria.
Each item shows the path name, a short description if available, estimated distance and travel time,
and its current condition (e.g., Optimal, Maintenance) together with the number of aggregated reports.
The underlying map remains visible as a background reference centered on the searched area,
while the user scrolls through the list to compare the alternatives.
Tapping the close icon in the top-right corner of the panel dismisses the results and returns
the interface to the previous state, where the user can enter a new search and run it again.

The behavior is identical for logged-in and guest users, except for the general restrictions
applied to guests.

\subsection*{Automatic Ride Detection}
\label{sec:automatic_ride_detection}%

When the app is open, BBP continuously monitors the user's movement pattern.
If the system detects that the user is likely biking without having started a trip, it displays
a modal prompt at the top of the map asking: “Are you biking?”. The dialog reminds the user not
to forget to start a trip so that the ride can be tracked.

The pop-up dims the underlying interface and presents a single primary action.
Tapping this button initiates the standard trip-start flow. If the user taps
outside the modal, the dialog closes and the app returns to the normal view without
starting a trip.

The behaviour is the same for logged-in and guest users in terms of detection and navigation.
However, as outlined in the other screens, only logged-in users will have their trip stored and
included in their personal statistics.

\section{Path Selection and Trip Start}
\label{sec:path_selection_and_trip_start}%

\begin{figure}[H]
	\centering
	\begin{minipage}[t]{0.31\textwidth}
		\centering
		\vspace{0pt}
		\phoneimage[width=\textwidth]{Design/7_select.png}
		\caption{Path Selection}
		\label{fig:path_selection_mockup}
	\end{minipage}
	\hfill
	\begin{minipage}[t]{0.31\textwidth}
		\centering
		\vspace{0pt}
		\phoneimage[width=\textwidth]{Design/7_select_guest.png}
		\caption{Path Selection for Guests}
		\label{fig:path_selection_mockup_guest}
	\end{minipage}
	\hfill
	\begin{minipage}[t]{0.31\textwidth}
		\centering
		\vspace{0pt}
		\phoneimage[width=\textwidth]{Design/8_mode.png}
		\caption{Automatic Mode Activation}
		\label{fig:automatic_mode_activation_popup_mockup}
	\end{minipage}
\end{figure}

\subsection*{Path Selection}
\label{sec:path_selection}%

When the user taps on one of the suggested paths in the results list, BBP highlights the selected
option by applying a blue outline around the corresponding card. This visual cue clearly indicates
which path is currently active while keeping the rest of the list unchanged. Tapping a different
card updates the selection, tapping on the same card again deselects it, while tapping outside does
not alter the current state.

Once a path is selected, the app displays the corresponding route on the map. The route is shown
using a bold blue polyline, along with the estimated distance and travel time, giving the user an
immediate visual understanding of the full trip.
If the selected path includes confirmed reports, their corresponding markers are displayed on the
map, allowing the user to identify potentially problematic segments.

A “Start Trip” button appears inside the selected card only when the path origin matches the user’s
current GPS position.
If the origin does not correspond to the current position, the app displays the path preview and its
route on the map, but the user can't start a trip from that location.

When the Start Trip button is available and pressed, for logged-in users, the system opens a
pop-up requesting whether to enable the Automatic Report Mode. Once the mode is selected, the
trip begins.

For guest users, the trip can still be started, but the activity is not stored in the system
and no automatic detection features are available.

Pressing the close icon at the top of the results panel hides the list and returns the interface
to the standard map view without an active selection.

The behaviour is identical for logged-in and guest users regarding map visualization, path selection,
and result browsing. However, only logged-in users can access advanced features from the bottom
navigation bar, which appear disabled in guest mode.

\subsection*{Automatic Mode Activation}
\label{sec:automatic_mode_activation}%

After tapping the Start Trip button, logged-in users are shown a pop-up prompting them to enable
the Automatic Report Mode. This mode activates the device’s sensors (gyroscope and accelerometer)
to automatically detect potential obstacles such as potholes or rough surfaces. Whenever an anomaly
is detected, BBP prepares a pre-filled report and displays a confirmation prompt for the user to
review, edit, or submit.

If the user continues, the trip starts immediately and the system begins tracking the movement
along the selected path. Automatic detection does not prevent manual reporting: users may still
send manual reports at any time during the trip through the dedicated interface.

\section{Navigation and Trip Completion}
\label{sec:navigation_and_trip_completion}%

\begin{figure}[H]
	\centering
	\begin{minipage}[t]{0.31\textwidth}
		\centering
		\vspace{0pt}
		\phoneimage[width=\textwidth]{Design/9_in_trip.png}
		\caption{Navigation View }
		\label{fig:navigation_view_mockup}
	\end{minipage}
	\hspace{0.05\textwidth}
	\begin{minipage}[t]{0.31\textwidth}
		\centering
		\vspace{0pt}
		\phoneimage[width=\textwidth]{Design/9_in_trip_guest.png}
		\caption{Navigation View for Guest Users}
		\label{fig:navigation_view_guest_mockup}
	\end{minipage}
\end{figure}
\begin{figure}[H]
	\centering
	\begin{minipage}[t]{0.31\textwidth}
		\centering
		\vspace{0pt}
		\phoneimage[width=\textwidth]{Design/10_trip_completed.png}
		\caption{Trip Completion}
		\label{fig:trip_completion_mockup}
	\end{minipage}
	\hspace{0.05\textwidth}
	\begin{minipage}[t]{0.31\textwidth}
		\centering
		\vspace{0pt}
		\phoneimage[width=\textwidth]{Design/10_trip_completed_guest.png}
		\caption{Trip Completion for Guest Users}
		\label{fig:trip_completion_guest_mockup}
	\end{minipage}
\end{figure}

\subsection*{Navigation View}
\label{sec:navigation_view}%

During an active trip, the interface switches to a dedicated full-screen navigation mode.
The map occupies the entire screen, displaying the user’s real-time position and the selected
path highlighted with a thick blue route polyline. No bottom navigation elements are accessible
during a trip, ensuring that the user remains fully focused on navigation.

At the bottom of the screen, a large Complete Trip button allows the user to manually end the
session at any moment. On the lower-right side, the interface shows a floating report button
red and active for logged-in users, enabling manual obstacle reporting.
The button is grey and disabled for guest users, who can't submit reports, and if clicked
it triggers a pop-up inviting them to sign in.
The button’s floating position and color clearly distinguish whether reporting is available.

A trip session ends when the user presses the Complete Trip button, the destination is reached or
when the system detects that the user has significantly deviated from the planned path.
When the trip ends, BBP stops tracking and displays a confirmation message.

For logged-in users, the trip data is stored in their personal history.
For guest users, the navigation experience is identical, but no information is saved once the trip
is completed.

This view visually reinforces the functional difference between user types: although both can ride
along a selected path, only logged-in users can contribute reports and preserve their trip data.

\subsection*{Trip Completion}
\label{sec:trip_completion}%

At the end of a trip, the app displays a completion pop-up that summarizes the outcome
of the session.
This pop-up appears in all termination scenarios.

For logged-in users, the pop-up shows the message “Great Ride!” and includes a View Stats button.
Tapping this button redirects the user to the Trip History screen and automatically opens the detailed
summary of the trip that has just been completed.
If the user taps outside the dialog, the pop-up simply closes and the map becomes interactive again.

For guest users, the pop-up displays “Trip Complete” along with a reminder that trip statistics are
not saved in guest mode.
Only one button is shown, which closes the dialog.
Since guest trips are never stored, no option to view statistics is provided.

In both cases, the background map is dimmed while the modal is visible, preventing interaction
until the user acknowledges the pop-up.

\section{Reports}
\label{sec:reports}%

\begin{figure}[H]
	\centering
	\begin{minipage}[t]{0.31\textwidth}
		\centering
		\vspace{0pt}
		\phoneimage[width=\textwidth]{Design/11_report.png}
		\caption{Report Submission }
		\label{fig:report_submission_mockup}
	\end{minipage}
	\hfill
	\begin{minipage}[t]{0.31\textwidth}
		\centering
		\vspace{0pt}
		\phoneimage[width=\textwidth]{Design/11_report_success.png}
		\caption{Report Submission Success}
		\label{fig:report_submission_success_popup_mockup}
	\end{minipage}
	\hfill
	\begin{minipage}[t]{0.31\textwidth}
		\centering
		\vspace{0pt}
		\phoneimage[width=\textwidth]{Design/12_confirm_report.png}
		\caption{Report Confirmation }
		\label{fig:report_confirmation_mockup}
	\end{minipage}
\end{figure}

\subsection*{Report Submission}
\label{sec:report_submission}%

During an active trip, logged-in users can submit a report through a dedicated pop-up interface.
The report dialog appears when the user taps the red report button or when an automatic detection
triggers a report prompt. The pop-up overlays the current navigation view and pauses the map
interaction by dimming the background.

The pop-up contains two fields:
Path Condition, allowing the user to assign the current status of the affected segment
(e.g., Optimal, Requires Maintenance, Closed), and Obstacle Type, specifying the encountered issue
(e.g., Obstacle, Path Damage, Work in Progress).

For manual submissions, both fields are initially empty and must be filled in by the user before
submitting. For automatic detections, the fields are pre-filled with the values inferred from sensor
data, but the user may freely edit the information, confirm it, or discard the report entirely.

Once the report is successfully submitted, the app displays a confirmation pop-up titled
“Report Submitted”, thanking the user for contributing to the community. The dialog includes
a Continue Trip button, which closes the pop-up and returns the user to the navigation view so
they can resume riding without interruption.
If the user does not interact with the dialog, it will automatically close after a short timeout.

Only logged-in users can access this feature. In guest mode, the report button is visible but appears
greyed out and can't be pressed.

\subsection*{Report Confirmation}
\label{sec:report_confirmation}%

While following a path during an active trip, BBP notifies logged-in users when they approach
a location where another cyclist previously submitted a report.
In this situation, a confirmation pop-up appears at the top of the screen with the message
“Obstacle Detected”, informing the user that an obstacle was previously reported at that spot.

The dialog asks the user to confirm whether the issue is still present, offering two options,
one to confirm that the report is still valid and another to mark it as no longer present.

The background map is dimmed to draw attention to the dialog while keeping the navigation visible.
If the user does not respond within a short timeout, the pop-up automatically closes so as not to
interrupt the ride.

After the user chooses either option, a second pop-up briefly appears to confirm that their response
has been recorded, allowing them to immediately resume their trip.
This confirmation dialog automatically closes after a short delay if the user does not interact with it

This feature is available only for logged-in users, guest users do not receive confirmation prompts
and can't validate reports.

\section{Path Creation for Logged-in Users}
\label{sec:path_creation_for_logged_in_users}%

\begin{figure}[H]
	\centering
	\begin{minipage}[t]{0.31\textwidth}
		\centering
		\vspace{0pt}
		\phoneimage[width=\textwidth]{Design/13_create.png}
		\caption{Path Creation }
		\label{fig:path_creation_mockup}
	\end{minipage}
	\hfill
	\begin{minipage}[t]{0.31\textwidth}
		\centering
		\vspace{0pt}
		\phoneimage[width=\textwidth]{Design/14_during_creation.png}
		\caption{Creation View }
		\label{fig:finish_creation_mockup}
	\end{minipage}
	\hfill
	\begin{minipage}[t]{0.31\textwidth}
		\centering
		\vspace{0pt}
		\phoneimage[width=\textwidth]{Design/15_creation_completed.png}
		\caption{Creation Completion }
		\label{fig:creation_completion_mockup}
	\end{minipage}
\end{figure}

\subsection*{Path Creation}
\label{sec:path_creation}%

When the user selects the path creation feature, a pop-up form appears at the top of the screen,
requesting the basic information needed to define a new bike path.
The form includes Path Name, where the user enters the title of the new path, Description,
an optional text field for adding details about the route, Visibility, allowing the user to
choose whether the new path will be public or private. The user can also select
the Creation Mode, which determines how the path will be constructed: either manually by drawing
it on the map, or automatically by recording the GPS track during a ride.

Once the required fields are completed, the user may proceed by tapping Start Creating,
which closes the dialog and starts the selected creation flow.
If the user chooses not to continue, the pop-up can be dismissed at any time using the button
in the upper-right corner.

Only logged-in users can create new paths, this feature is not available in guest mode.

\subsection*{Creation View}
\label{sec:finish_creation}%

When the user begins creating a new path, the app transitions into a dedicated creation view.
The interface expands to a full-screen map and hides all navigation elements, allowing the user
to focus entirely on constructing the path.

The content of this screen depends on the selected creation mode.
In Automatic Mode, the map shows the user’s real-time position as they cycle.
As the user moves, the app draws a blue polyline representing the segment of the path already
recorded. The line continuously updates to reflect the ongoing GPS trace.
This view allows the user to visually monitor the path being generated as they ride.

In Manual Mode, the user can manually define the route by interacting with the map.
Each waypoint placed by the user extends the blue polyline, progressively shaping the
final path. This mode allows full manual control over the geometry of the route.

At any moment, the user may cancel the creation by tapping the button in the upper-left corner,
which closes the creation view and discards the current progress.

When the user is satisfied with the path, they can save it by tapping the Save Path button
at the bottom of the screen.
This action stores the new path together with the metadata previously entered in the creation form.

Only logged-in users can access the creation view and save newly created paths.

\subsection*{Creation Completion}
\label{sec:creation_completion}%

After the user completes the path creation process and saves the new route, the app displays
a confirmation pop-up titled “Path Created!”.
The dialog informs the user that the new path has been successfully stored and is now ready to use.

The pop-up includes a single action button, “Go to My Paths”, which redirects the user to the
section containing all the paths they have created. From there, they can view the newly saved
path in detail or manage their collection.

If the user taps outside the dialog, the pop-up simply closes and the app returns to the map view,
allowing them to continue exploring the interface.

This confirmation appears only for logged-in users, as guests can't create or save custom paths.

\section{Trip Management for Logged-in Users}
\label{sec:trip_management_for_logged_in_users}%

\begin{figure}[H]
	\centering
	\begin{minipage}[t]{0.31\textwidth}
		\centering
		\vspace{0pt}
		\phoneimage[width=\textwidth]{Design/16_trip_history.png}
		\caption{Trip History Screen }
		\label{fig:trip_history_screen_mockup}
	\end{minipage}
	\hfill
	\begin{minipage}[t]{0.31\textwidth}
		\centering
		\vspace{0pt}
		\phoneimage[width=\textwidth]{Design/16_trip_order.png}
		\caption{Trip Sorting }
		\label{fig:trip_sorting_options_mockup}
	\end{minipage}
	\hfill
	\begin{minipage}[t]{0.31\textwidth}
		\centering
		\vspace{0pt}
		\phoneimage[width=\textwidth]{Design/16_trip_expanded.png}
		\caption{Trip Details }
		\label{fig:trip_history_details_mockup}
	\end{minipage}
\end{figure}

\begin{figure}[H]
	\centering
	\begin{minipage}[t]{0.31\textwidth}
		\centering
		\vspace{0pt}
		\phoneimage[width=\textwidth]{Design/16_trip_stats.png}
		\caption{Trip Statistics }
		\label{fig:trip_statistics_mockup}
	\end{minipage}
	\hfill
	\begin{minipage}[t]{0.31\textwidth}
		\centering
		\vspace{0pt}
		\phoneimage[width=\textwidth]{Design/16_trip_weather.png}
		\caption{Trip Weather }
		\label{fig:trip_weather_details_mockup}
	\end{minipage}
	\hfill
	\begin{minipage}[t]{0.31\textwidth}
		\centering
		\vspace{0pt}
		\phoneimage[width=\textwidth]{Design/16_delete_trip.png}
		\caption{Trip Deletion }
		\label{fig:trip_deletion_confirmation_mockup}
	\end{minipage}
\end{figure}

The Trip History screen allows logged-in users to review all their previously recorded cycling
activities and is accessible through the second icon in the bottom navigation bar. The top
section includes a sorting control placed beside the screen title. Tapping it opens a compact
dropdown menu that lets the user reorder the list of trips by date, distance, duration,
or alphabetical order. Once a sorting option is selected, the list immediately updates to
reflect the new ordering.

All trips are presented as compact cards showing essential information about each session,
including the trip name, the total distance, the duration, and the date of completion. A delete
icon also appears on each card, allowing users to remove a trip from their history. When the user
presses the delete icon, a confirmation pop-up appears to prevent accidental deletions. When the user
taps a trip card, it expands smoothly into a detailed view. In this expanded state, the interface
displays a map preview showing the route followed during the session, enriched with a weather badge
positioned in the corner of the map. Beneath the map, a summary section reports the date, duration,
and the name of the selected path, followed by a performance panel showing additional metrics such
as distance, average speed, maximum speed, and elevation.

If the user taps the weather badge, the app reveals an additional panel containing detailed
meteorological information, such as temperature, humidity, wind speed, visibility, pressure, and
overall weather conditions at the time of the trip. When the recorded activity includes confirmed
reports, the corresponding markers appear directly on the map. Selecting one of them opens a small
dialog containing the details of the associated issue, allowing the user to inspect what was
encountered along the route.

Tapping the trip header again collapses the view, returning the interface to the scrollable list of
compact cards. This screen provides a rich and well-structured overview of past cycling activities,
enabling users to explore their performance and recall the conditions of each session.

The Trip History is available exclusively to logged-in users, as guest users can't store or view past trips.



\section{Paths Management for Logged-in Users}
\label{sec:paths_management_for_logged_in_users}%

\begin{figure}[H]
	\centering
	\begin{minipage}[t]{0.31\textwidth}
		\centering
		\vspace{0pt}
		\phoneimage[width=\textwidth]{Design/17_paths.png}
		\caption{My Paths Screen }
		\label{fig:my_paths_screen_mockup}
	\end{minipage}
	\hspace{0.05\textwidth}
	\begin{minipage}[t]{0.31\textwidth}
		\centering
		\vspace{0pt}
		\phoneimage[width=\textwidth]{Design/17_path_expanded.png}
		\caption{Path Details }
		\label{fig:my_path_details_mockup}
	\end{minipage}
\end{figure}

\begin{figure}[H]
	\centering
	\begin{minipage}[t]{0.31\textwidth}
		\centering
		\vspace{0pt}
		\phoneimage[width=\textwidth]{Design/17_path_visibility.png}
		\caption{Path Visibility }
		\label{fig:path_visibility_toggle_mockup}
	\end{minipage}
	\hspace{0.05\textwidth}
	\begin{minipage}[t]{0.31\textwidth}
		\centering
		\vspace{0pt}
		\phoneimage[width=\textwidth]{Design/17_delete_path.png}
		\caption{Path Deletion }
		\label{fig:path_deletion_confirmation_mockup}
	\end{minipage}
\end{figure}

The My Paths screen displays all custom bike paths created by the logged-in user and is accessible
through the third icon in the bottom navigation bar. At the top of the page, a small arrow icon next
to the screen title opens a compact sorting menu, allowing the user to reorder their paths by date,
distance, alphabetical order, or visibility. Once an option is selected, the list is immediately
re-sorted.

Each item in the list shows the path title, a short description, the total distance, the creation
date, and two management icons: one for toggling visibility and one for deletion. Tapping a path
expands it and reveals a map preview showing the full route. When expanded, the entry also provides
a Start This Path button, which redirects the user to the home screen with the selected path already
loaded on the map. If the user’s current position matches the path’s starting point, the trip can be
started immediately.

Managing a path triggers dedicated confirmation dialogs. Changing visibility opens a pop-up asking
the user to confirm whether they want to switch the path between public and private. The change is
applied only after confirmation. Deleting a path opens a separate confirmation dialog warning that
the action is permanent, and the path is removed only if the user explicitly confirms.

Collapsing an expanded path restores the compact list layout, allowing users to browse their
collection efficiently.

All interactions inside this screen are available exclusively to logged-in users. Guests can't view,
create, start, or manage custom paths.

\section{Profile and Settings for Logged-in Users}

\begin{figure}[H]
	\centering
	\begin{minipage}[t]{0.31\textwidth}
		\centering
		\vspace{0pt}
		\phoneimage[width=\textwidth]{Design/18_profile.png}
		\caption{Profile Screen }
		\label{fig:profile_screen_mockup}
	\end{minipage}
	\hfill
	\begin{minipage}[t]{0.31\textwidth}
		\centering
		\vspace{0pt}
		\phoneimage[width=\textwidth]{Design/20_settings.png}
		\caption{Settings Screen }
		\label{fig:settings_screen_mockup}
	\end{minipage}
	\hfill
	\begin{minipage}[t]{0.31\textwidth}
		\centering
		\vspace{0pt}
		\phoneimage[width=\textwidth]{Design/19_edit_profile.png}
		\caption{Edit Profile }
		\label{fig:edit_profile_mockup}
	\end{minipage}
\end{figure}

\subsection*{Profile Screen}
\label{sec:profile_screen}%

The Profile screen provides logged-in users with a complete overview of their cycling activity
and personal account details. It is accessible through the fourth icon in the bottom navigation bar.

At the top of the page, the user’s profile card displays their avatar, name, and email address.
A small edit icon next to the name allows the user to update their personal information, such as name
or email. A settings icon is available in the top-right corner and redirects to the Settings screen,
where account preferences and configuration options can be adjusted.

Below the profile header, the screen shows a set of Overall Stats, summarising the user’s total
distance ridden, total number of recorded trips, and the total number of custom paths they have created.
These metrics provide a quick snapshot of the user’s long-term activity.

Further down, the Activity Stats section offers a more detailed breakdown. A compact dropdown
menu allows the user to filter the statistics by time period, such as monthly activity.
The metrics displayed include the number of paths and trips completed during the selected
period, the total distance travelled, total riding time, average and maximum speed, elevation gain,
calories burned, and other performance indicators. Each metric is represented with its own colourful
icon for quick visual identification.

All information in this section is available exclusively to logged-in users. Guests cannot access
the Profile screen or view personal statistics.

\subsection*{Edit Profile}
\label{sec:edit_profile}%

The Edit Profile screen allows logged-in users to update their account details.
It includes editable fields for the Username and Email Address, followed by a dedicated section
for changing the password.

To update the password, the user must enter their Current Password, then provide a New Password,
and finally confirm it in the Confirm Password field. This ensures that the new credentials are
correctly entered and match before proceeding.

Once all desired changes have been made, the user can tap Save Changes to submit the updated
information. The app validates the input and, if successful, stores the new data and returns the
user to the previous screen. A Cancel button is also available, allowing the user to dismiss the
form without applying any modifications.

This screen is accessible exclusively to logged-in users and is reached from the Profile screen by
tapping the pencil icon next to the user’s name. The user can return to the Profile view
at any time by tapping the back arrow in the top-left corner.


\subsection*{Settings Screen}
\label{sec:settings_screen}

The Settings screen allows logged-in users to customize their app experience and manage a small set
of personal preferences. It is accessible from the Profile screen through the settings icon
in the top-right corner.

At the top, the Appearance option lets the user choose the visual theme of the application.
The current theme is displayed on the right side of the row, and tapping the selector opens a small
menu where the user can switch between the available modes (e.g. Light or Dark).

Below it, the Default Privacy setting controls the visibility assigned to newly created paths.
The user can choose whether new paths should be public or private by default. This preference
applies automatically during path creation but can still be overridden manually for each individual path.

The Get Help section provides quick access to the support channel, allowing the user to contact the
team if assistance is needed.

At the bottom of the screen, the Sign Out button logs the user out of their account and returns the
application to guest mode.

This screen is available exclusively to logged-in users, as guest users do not manage appearance
preferences, default visibility settings, or account actions.

\section{Error Messages}
\label{sec:error_messages}%

\begin{figure}[H]
	\centering
	\begin{minipage}[t]{0.65\textwidth}
		The app includes a global error-handling mechanism that provides consistent feedback whenever
		an unexpected issue occurs. In these cases, a dedicated Error pop-up appears at the top of the screen,
		displaying a clear title and a short description of what went wrong. The message shown in the dialog
		is dynamically replaced with the specific error text relevant to the situation, such as network
		failures, permission issues, or invalid operations.

		The pop-up dims the underlying interface to draw attention to the alert and includes a single
		Close button that dismisses the dialog and returns the user to the previous screen. Tapping outside
		the pop-up also closes the notification.

		This error dialog can appear at any point in the app, including during path searches, trip management,
		report submission, navigation, or when modifying custom paths.
	\end{minipage}
	\hfill
	\begin{minipage}[t]{0.31\textwidth}
		\centering
		\vspace{0pt}
		\phoneimage[width=\textwidth]{Design/21_error.png}
		\caption{Error Pop-up }
		\label{fig:error_popups_mockup}
	\end{minipage}
\end{figure}

% --------------------------------------------------------------------------
% Navigation Flow
% --------------------------------------------------------------------------

\section{Navigation Flow}
\label{sec:navigation_flow}%

At a global level, the application is organised around a small set of core areas:
the authentication entry point (Welcome, Login, Signup), the map-based Home Screen, the trip-related
features (navigation, reporting, history), the path management features (creation and created paths), and
the personal area (Profile, Edit Profile, Settings).
The bottom navigation bar connects the Home Screen, Trip History, My Paths, and Profile, while
contextual pop-ups and full-screen views are used to support focused tasks such as starting a trip,
creating a path, or handling errors.

The typical entry flow starts from the Welcome Screen, where the user can either authenticate
or continue as a guest. In both cases, the user is redirected to the Home Screen, which acts as the
central hub for the rest of the application. From here, users can search for paths, inspect search
results, start a trip along a selected route, or, if authenticated, initiate the creation of a new
path. During navigation, the interface temporarily switches to a dedicated full-screen view for the
active trip, and returns to the standard map view once the trip is completed.

Trip-related exploration is handled through the Trip History Screen, which the user can access at
any time via the navigation bar. From this section, users can review past activities and inspect
the associated details, while the My Paths Screen provides a similar list-and-detail pattern for
custom paths, including controls for starting a trip from a saved path and for managing visibility
and deletion. The Profile Screen and its related views (Edit Profile and Settings) group all
account-related functionalities and preferences, and can be reached through the Profile entry in
the navigation bar and the settings shortcut within the profile.

Throughout these flows, the application distinguishes between guest and logged-in users: guests have
access to browsing and basic navigation, whereas authenticated users can record trips, submit and
validate reports, create and manage custom paths, and view statistics. In addition, several
cross-cutting behaviours, such as automatic ride detection, authentication prompts for restricted
actions, and global error pop-ups, may appear on top of the current screen without altering the
underlying navigation structure, ensuring that the overall flow remains consistent across different
usage scenarios.
