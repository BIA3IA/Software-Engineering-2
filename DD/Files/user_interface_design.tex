% --------------------------------------------------------------------------
% User Interface Design
% --------------------------------------------------------------------------

% --------------------------------------------------------------------------
% User Interfaces
% --------------------------------------------------------------------------
\section{User Interfaces}
\label{sec:user_interfaces}%

The following section presents an overview of the user interfaces
designed for the Best Bike Paths (BBP) application.
All mockups included here are conceptual prototypes aimed at illustrating the expected
interaction flow and the general layout of the main pages of the system. They are not intended to represent
the final graphical design, which will be refined during implementation.

The purpose of this section is to provide a clear, high-level visualization of how users will access
core functionalities, such as registration, login, path exploration, trip recording, reporting,
and profile management.

\subsection{Welcome Page}
\label{sec:welcome_page}%

When opening the application for the first time, users are greeted with a welcome page
that introduces the Best Bike Paths (BBP) system.
This screen presents the app logo, a short tagline summarizing its main purpose, and
a “Get Started” button that allows the user to proceed to the authentication flow.
The page serves as a simple entry point, offering a clean overview of the app before
accessing any functionality.
At this stage, no interaction with backend services is required. The page only guides
users toward signing in or creating an account.

\begin{figure}[H]
	\centering
	\includegraphics[width=0.5\textwidth]{Design/1_welcome.png}
	\caption{Welcome Page Mockup}
	\label{fig:welcome_page_mockup}
\end{figure}


\subsection{Login Page}
\label{sec:login_page}%

After selecting Get Started from the welcome screen, users are taken to the login page.
This interface allows returning users to authenticate by entering their username and
password. The page also provides shortcuts to the Sign Up screen for new users and
an option to continue as a Guest, enabling limited access without registration.
The purpose of this page is to act as the main entry point into the application,
ensuring that authenticated users can access all features such as trip recording
path creation, and reporting.

\begin{figure}[H]
	\centering
	\includegraphics[width=0.5\textwidth]{Design/3_login.png}
	\caption{Login Page Mockup}
	\label{fig:login_page_mockup}
\end{figure}


\subsection{Signup Page}
\label{sec:signup_page}%

The sign-up page enables new users to create an account by providing an email,
username, and password.
This interface is designed to keep the registration process simple, requiring only
the essential information needed to activate a personal profile.
From this page, users can switch to the Login page if they already have an account,
or continue as a Guest if they prefer to explore the system without registering.
Creating an account unlocks all core functionalities of BBP, such as recording trips,
submitting reports, managing custom paths, and accessing personal statistics.

\begin{figure}[H]
	\centering
	\includegraphics[width=0.5\textwidth]{Design/2_signup.png}
	\caption{Signup Page Mockup}
	\label{fig:signup_page_mockup}
\end{figure}

\subsection{Home Page}
\label{sec:home_page}%

The home page is the main entry point for authenticated users.
It provides access to the core features of the application through an interactive map
and a bottom navigation bar.
At the top of the screen, users can specify an origin and a destination to initiate a
path search. The Search Path button triggers the route computation process, and results
are later displayed.
The interactive map occupies most of the screen and shows the user’s current position.
A floating button allows authenticated users to create a new path, leading them to the
manual or automatic creation flow.
The bottom navigation bar grants access to all main sections of the application
(home, trip history, path history, and user profile). All functionalities are fully
available for logged-in users.

\begin{figure}[H]
	\centering
	\includegraphics[width=0.5\textwidth]{Design/4_home.png}
	\caption{Home Page Mockup}
	\label{fig:home_page_mockup}
\end{figure}

\subsection{Home Page for Guest Users}
\label{sec:home_page_guest}%

Guest users access a simplified version of the home page.
While the map and the search form (origin, destination, and Search Path button) remain
available, all other features require authentication and are therefore visually disabled.
Items in the bottom navigation bar appear greyed out, indicating restricted access.
Guest users can only perform path searches and browse suggested bike routes, but cannot
create paths, record trips, or access personal sections of the application.
If the user attempts to tap any disabled icon, the application displays a pop-up
prompting them to sign up or log in in order to unlock the full feature set.
This layout clearly communicates the distinction between browsing capabilities and the
additional features unlocked through registration.

\begin{figure}[H]
	\centering
	\includegraphics[width=0.5\textwidth]{Design/4_home_guest.png}
	\caption{Home Page Mockup for Guest Users}
	\label{fig:home_page_mockup_guest}
\end{figure}

\subsection{Guest User Prompt}
\label{sec:guest_user_prompt}%

When a guest user attempts to interact with any restricted feature, such as the items
in the bottom navigation bar or other elements requiring authentication, the application
displays a modal pop-up overlay.
The pop-up clearly communicates that advanced functionalities
are available only to registered or logged-in users. It provides a single action button
that redirects the user to the authentication flow.
The rest of the interface is dimmed to highlight the modal and prevent interaction with the
disabled elements. If the user taps outside the pop-up, the modal closes and the guest user
is returned to the current screen, allowing them to continue exploring the map or searching
for bike paths without interruption.

\begin{figure}[H]
	\centering
	\includegraphics[width=0.5\textwidth]{Design/18_unlock.png}
	\caption{Guest User Prompt Mockup}
	\label{fig:guest_user_prompt_mockup}
\end{figure}

\subsection{Search Page}
\label{sec:search_page}%
After submitting an origin and a destination, the application displays a search results page showing
all suggested routes between the two points.
Each result includes the route name, its current condition (e.g., Optimal, Requires Maintenance),
and the number of aggregated reports supporting that evaluation.
The selected route is also highlighted on the map, which provides an estimate of distance and travel time.
The overall structure of the page is identical for both authenticated and guest users. However, only 
logged-in users can interact with advanced features such as opening the detailed route view, saving paths,
or starting a trip.
In guest mode, these elements, as well as the corresponding icons in the bottom navigation bar, appear
disabled. 

\begin{figure}[H]
	\centering
	\includegraphics[width=0.5\textwidth]{Design/5_search.png}
	\caption{Search Page Mockup}
	\label{fig:search_page_mockup}
\end{figure}

\begin{figure}[H]
	\centering
	\includegraphics[width=0.5\textwidth]{Design/5_search_guest.png}
	\caption{Search Page Mockup for Guest Users}
	\label{fig:search_page_mockup_guest}
\end{figure}

\subsection{Route Selection}
\label{sec:route_selection}%

When the user taps on one of the suggested routes in the results list, the application
highlights the selected route on the map. The selected route is highlighted in blue,
making it clearly distinguishable from the other suggestions. Tapping the same route
again deselects it, while tapping a different route updates the selection accordingly.

The map visualization includes the complete path layout, estimated travel time and
distance, and any relevant annotations such as obstacle markers. If the selected route
contains confirmed reports, these are shown directly on the map using warning icons,
allowing the user to immediately identify critical or deteriorated segments. Tapping one
of these markers opens a small dialog displaying the corresponding report details, giving
the user additional insight into the nature and status of the obstacle.

Editing either the origin or destination input fields automatically clears the current
selection and the existing results, triggering a new search once the user submits updated
values.

The behaviour is identical for both authenticated and guest users. The only difference
is that logged-in users can proceed with restricted actions, while in guest mode those
controls appear disabled.

Multiple mockups are provided to illustrate different scenarios, including routes with
no reports and routes containing obstacle indicators.


\begin{figure}[H]
    \centering
    \includegraphics[width=0.5\textwidth]{Design/6_select.png}
    \caption{Route Selection Mockup without Reports}
    \label{fig:route_selection_no_reports_mockup}
\end{figure}

\begin{figure}[H]
    \centering
    \includegraphics[width=0.5\textwidth]{Design/7_select_report.png}
    \caption{Route Selection Mockup with Reports}
    \label{fig:route_selection_single_report_mockup}
\end{figure}

\begin{figure}[H]
    \centering
    \includegraphics[width=0.5\textwidth]{Design/6_select_guest.png}
    \caption{Route Selection Mockup for Guest Users}
    \label{fig:route_selection_guest_mockup}
\end{figure}

\begin{figure}[H]
    \centering
    \includegraphics[width=0.5\textwidth]{Design/7_select_report_guest.png}
    \caption{Route Selection Mockup with Reports for Guest Users}
    \label{fig:route_selection_single_report_guest_mockup}
\end{figure}

\subsection{Different Origin}
\label{sec:different_origin}%

If the origin specified in the search does not match the user's current physical location,
the system allows the user to view the results but prevents trip initiation. In this case,
the \textit{Start Trip} button is disabled for authenticated users, and no start option is
available for guests. The user can explore the route but cannot begin navigation until
the actual position coincides with the search origin.

\begin{figure}[H]
    \centering
    \includegraphics[width=0.5\textwidth]{Design/23_search_no_origin.png}
    \caption{Detailed Route View Mockup}
    \label{fig:detailed_route_view_mockup}
\end{figure}
 
\begin{figure}[H]
    \centering
    \includegraphics[width=0.5\textwidth]{Design/23_search_no_origin_guest.png}
    \caption{Detailed Route View Mockup for Guest Users}
    \label{fig:detailed_route_view_guest_mockup}
\end{figure}

\subsection{Mode Selection Prompt}
\label{sec:mode_selection_prompt}%

After tapping the \textit{Start Trip} button, authenticated users are prompted to choose
whether to enable the Automatic Mode. This mode activates automatic obstacle detection
and data collection using the external available sensors. Even when Automatic Mode is
enabled, the user may still submit manual reports at any time by tapping the dedicated
report icon during the trip.
Once the user makes a selection, the application immediately starts the
trip session and begins tracking the user's movement along the selected route.

\begin{figure}[H]
    \centering
    \includegraphics[width=0.5\textwidth]{Design/9_mode.png}
    \caption{Trip Start Prompt Mockup}
    \label{fig:trip_start_prompt_mockup}
\end{figure}

\subsection{During the Trip}
\label{sec:during_the_trip}%

During an active trip, the application switches to a dedicated navigation view showing
the user's real-time position on the selected route. The route remains highlighted in blue,
and the interface provides two main controls: a \textit{Stop Trip} button at the top of the
screen and a report button for submitting manual obstacle reports.

When the user taps \textit{Stop Trip}, the trip immediately ends and is saved to the
user's history. Trip sessions are also automatically stopped and saved when the user
reaches the destination or when the system detects that the user has moved significantly
away from the planned route. In all cases, the application notifies the user that the trip
has been successfully stored.

The report button allows authenticated users to submit manual reports at any moment
during the trip. This feature is disabled when the user is in guest mode. In guest mode,
all trip-related buttons appear greyed out, no reports can be submitted, and no trip data
is stored. When a guest ends a trip no information is saved.

\begin{figure}[H]
    \centering
    \includegraphics[width=0.5\textwidth]{Design/8_trip_started.png}
    \caption{During the Trip Mockup}
    \label{fig:during_the_trip_mockup}
\end{figure}

\begin{figure}[H]
    \centering
    \includegraphics[width=0.5\textwidth]{Design/8_trip_started_guest.png}
    \caption{During the Trip Mockup for Guest Users}
    \label{fig:during_the_trip_guest_mockup}
\end{figure}

\subsection{Finish Trip}
\label{sec:finish_trip}%

At the end of a trip, the application displays a pop-up informing the user that the
navigation session has been completed. The pop-up appears both when the user manually
stops the trip and when the system automatically ends it upon reaching the destination
or detecting a significant deviation from the planned route.

For authenticated users, the pop-up offers two options: closing the dialog or opening the
trip summary page. Selecting \textit{Check Summary} redirects the user to the list of
recorded trips, where a detailed overview of the completed session can be reviewed.

In guest mode, the pop-up only displays a \textit{Close} button. Since guest trips are not
stored, no summary is available, and the user is returned directly to the map after closing
the dialog.

\begin{figure}[H]
    \centering
    \includegraphics[width=0.5\textwidth]{Design/20_finished_trip.png}
    \caption{Finish Trip Mockup}
    \label{fig:finish_trip_mockup}
\end{figure}

\begin{figure}[H]
    \centering
    \includegraphics[width=0.5\textwidth]{Design/20_finished_trip_guest.png}
    \caption{Finish Trip Mockup for Guest Users}
    \label{fig:finish_trip_guest_mockup}
\end{figure}

\subsection{Reports}
\label{sec:reports}%

During an active trip, users may submit reports through a dedicated pop-up interface.
This dialog appears in two situations: when the user manually taps the report button, or
when the Automatic Mode detects a potential obstacle. In the manual case, the fields
are initially empty and must be filled in by the user. When the pop-up is triggered by an
automatic detection, the form is pre-compiled with the estimated path status and obstacle
type inferred from sensor data, although the user remains free to modify or discard the
report.

The pop-up allows authenticated users to either confirm the report or close the dialog
without submitting it. Only logged-in users can complete this action. In guest mode, the
report function is disabled and the corresponding button appears greyed out.

\begin{figure}[H]
    \centering
    \includegraphics[width=0.5\textwidth]{Design/10_report.png}
    \caption{Report Submission Mockup}
    \label{fig:report_submission_mockup}
\end{figure}

\subsection{Confirm or Reject Report}
\label{sec:confirm_reject_report}%

While following a route during an active trip, the system notifies authenticated users
when they pass near a location where a previous report was submitted by another user.
In these cases, a confirmation pop-up appears, asking whether the reported obstacle is
still present. The dialog offers two options: \textit{Confirm}, to validate the existing
report, or \textit{Reject}, to indicate that the obstacle is no longer observed.

If the user does not interact with the pop-up within a short timeout period, the dialog
automatically disappears to avoid interrupting the navigation experience. This feature is
available only for authenticated users. Guest users are not prompted for report
confirmations and cannot participate in the validation process.

\begin{figure}[H]
    \centering
    \includegraphics[width=0.5\textwidth]{Design/19_confirm_report.png}
    \caption{Confirm or Reject Report Mockup}
    \label{fig:confirm_reject_report_mockup}
\end{figure}

\subsection{Creating a New Path}
\label{sec:creating_new_path}%

When the user selects the path creation feature, a pop-up form appears requesting the
basic metadata required to define a new bike path. The form includes fields for the path
title, an optional description, the desired visibility level (e.g., public or private), and the
creation mode.

The \textit{Creation Mode} determines how the new path will be generated. If the user
selects the manual mode, the path will be drawn directly on the map by placing and
adjusting waypoints. If the automatic mode is selected, the user will start cycling and the
application will record the GPS trace of the trip, using it as the final path geometry once
the recording is completed.

After filling in the required fields, the user may proceed by tapping \textit{Start Creation}
or dismiss the dialog with the \textit{Close} button. Only authenticated users can create
new paths.

\begin{figure}[H]
    \centering
    \includegraphics[width=0.5\textwidth]{Design/11_creation.png}
    \caption{Creating a New Path Mockup}
    \label{fig:creating_new_path_mockup}
\end{figure}

\subsection{Creation Finish}
\label{sec:creation_finish}%

Once the user completes the path creation process, the interface displays a
\textit{Finish Creation} button at the top of the screen. This behaviour is identical for both
creation modes. 

Tapping \textit{Finish Creation} saves the newly created path together with the metadata
provided in the initial form. Only authenticated users can perform this operation.
After confirmation, the application stores the
custom path and returns the user receives a notification indicating that the new route has been
successfully saved.

\begin{figure}[H]
    \centering
    \includegraphics[width=0.5\textwidth]{Design/13_finish_automatic.png}
    \caption{Creation Finish Mockup}
    \label{fig:creation_finish_mockup}
\end{figure}

\subsection{Creation Succeeds}
\label{sec:creation_succeeds}%

After completing the creation workflow, the system confirms the successful creation of
the new path by displaying a pop-up notification. The dialog informs the user that the
path has been saved and provides two options: closing the pop-up or navigating directly
to the list of previously created paths through the \textit{Check My Paths} button.

Only authenticated users can access this confirmation dialog, as guests cannot create or
store custom paths. Selecting \textit{Check My Paths} redirects the user to the dedicated
section where all their saved paths are listed, while choosing \textit{Close} simply returns
the user to the map view.

\begin{figure}[H]
    \centering
    \includegraphics[width=0.5\textwidth]{Design/22_creation_success.png}
    \caption{Creation Succeeds Mockup}
    \label{fig:creation_succeeds_mockup}
\end{figure}

\subsection{Trip History Page}
\label{sec:trip_history_page}%

The \textit{My Trips} page provides authenticated users with an overview of all previously
recorded trips. Each entry displays the destination name, date, total distance, and trip
duration. Tapping on a trip record expands it to show a detailed summary, including a
map visualization of the followed route, the weather conditions at the time of the trip,
and additional metrics such as average speed and elevation level. Tapping the same
record again collapses the details, returning the list to its compact form.

If the displayed weather badge is selected, the application reveals additional
weather-related information, providing a more complete snapshot of the environmental
conditions encountered during the trip.

If the route contained confirmed reports, these are displayed directly on the map using
warning markers, allowing the user to identify critical segments encountered during the
trip. Tapping on a marker opens a small dialog showing the details of the corresponding
report, enabling the user to inspect the nature and severity of the issue encountered on
that segment.

This page presents all trip-related information in a unified and structured format,
making it easy for users to review past sessions and compare their performance over
time.

This functionality is available only for authenticated users.

\begin{figure}[H]
    \centering
    \includegraphics[width=0.5\textwidth]{Design/14_trips.png}
    \caption{Trip History Page Mockup}
    \label{fig:trip_history_page_mockup}
\end{figure}

\subsection{Path History Page}
\label{sec:path_history_page}%

The \textit{My Paths} page displays all custom bike paths created by the authenticated
user. Each entry shows the path title, date of creation, total distance, and estimated
duration. Tapping on a path expands the entry and reveals a preview of the route on the
map, including any obstacle reports associated with that path. Tapping the same entry
again collapses the view and returns the list to its compact layout.

Each path provides a set of management actions. The visibility icon allows the user to
toggle the path's visibility between public and private, while the delete icon permanently
removes the path from the user's collection. Selecting the preview map or the dedicated
view icon opens a detailed view of the path, showing its full geometry and all associated
reports. Tapping a report marker within the preview reveals the corresponding report
details.

When the user selects the \textit{Start} button inside a path entry, the application redirects
them to the home page with the origin and destination fields automatically filled with the
path's endpoints, and the corresponding route already selected. The user may then start
the trip directly, provided that their current physical position matches the starting point
of the path.

This section is accessible only to authenticated users. Guest users cannot create, view,
or manage custom paths.


\begin{figure}[H]
    \centering
    \includegraphics[width=0.5\textwidth]{Design/15_paths.png}
    \caption{Path History Page Mockup}
    \label{fig:path_history_page_mockup}
\end{figure}

\subsection{Profile Page}
\label{sec:profile_page}%

The \textit{Profile} page provides authenticated users with an overview of their personal
statistics and riding activity. At the top of the page, the user can view their profile
information together with aggregated metrics such as the total distance travelled and
the total number of recorded trips. A shortcut is also provided to start a new trip
directly from this section.

The page includes a summary of the user's most recent trip, followed by a weekly
analysis showing the average ride duration for each day of the week. Additional
statistics, such as total weekly distance, average speed, and elevation level, are presented
in a dedicated section to give the user a comprehensive view of their performance and
progress over time.

A settings icon is available in the top-right corner of the page. Tapping this icon
redirects the user to the \textit{Settings} page, where personal preferences and privacy
options can be configured.

This page is available exclusively to authenticated users.

\begin{figure}[H]
    \centering
    \includegraphics[width=0.5\textwidth]{Design/16_profile.png}
    \caption{Profile Page Mockup}
    \label{fig:profile_page_mockup}
\end{figure}

\subsection{Settings Page}
\label{sec:settings_page}

The \textit{Settings} page, accessible from the Profile section, allows authenticated users
to customize their application preferences and manage their privacy settings. The page
includes options such as enabling or disabling Dark Mode and selecting the preferred
data sharing level (e.g., public or private).

The \textit{Contact Us} option redirects the user to the application's store page (Google
Play Store or Apple App Store, depending on the platform), where they can access the
official support channel or submit feedback.
 
At the bottom of the page, the user can log out from their account through the
\textit{Log Out} button, which immediately terminates the current session and returns the
application to guest mode.

This page is available only to authenticated users, as guest users do not manage personal
preferences or sharing settings.

\begin{figure}[H]
    \centering
    \includegraphics[width=0.5\textwidth]{Design/17_settings.png}
    \caption{Settings Page Mockup}
    \label{fig:settings_page_mockup}
\end{figure}

\subsection{Error Pop-ups}
\label{sec:error_popups}%

The application includes a global error-handling mechanism to guarantee consistent
feedback across all user interactions. Whenever an unexpected issue occurs — such as a
network failure, invalid input, or an internal processing error — the system displays a
dedicated error pop-up. The dialog presents a clear error message describing the problem
and provides a single \textit{Close} button to dismiss the notification.

This error pop-up may appear during any operation, including searching for routes,
starting or saving trips, submitting reports, or managing custom paths. Dismissing the
dialog returns the user to the previous screen without applying any partial changes,
ensuring a stable and predictable user experience.

\begin{figure}[H]
    \centering
    \includegraphics[width=0.5\textwidth]{Design/21_error.png}
    \caption{Error Pop-up Mockup}
    \label{fig:error_popups_mockup}
\end{figure}

% TODO rivedere le settings. Non puoi cambiare ne username ne email ora.
% TODO rivedere il flusso e rivedere cosa c'é scritto sui vari bottoni.