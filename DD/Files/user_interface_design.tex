% --------------------------------------------------------------------------
% User Interface Design
% --------------------------------------------------------------------------

% --------------------------------------------------------------------------
% User Interfaces
% --------------------------------------------------------------------------
\section{User Interfaces}
\label{sec:user_interfaces}%

The following section presents an overview of the user interfaces
designed for the Best Bike Paths (BBP) app.
All mockups included here are conceptual prototypes aimed at illustrating the expected
interaction flow and the general layout of the main pages of the system. They are not intended to represent
the final graphical design, which will be refined during implementation.

The purpose of this section is to provide a clear, high-level visualization of how users will access
core functionalities, such as registration, login, path exploration, trip recording, reporting,
and profile management.

\subsection{Welcome Screen}
\label{sec:welcome_screen}%

When opening the app for the first time, users are greeted with a welcome screen
that introduces the Best Bike Paths (BBP) system.
This screen presents the app logo, a short tagline summarizing its main purpose, and
a “Get Started” button that allows the user to proceed to the authentication flow.
The screen serves as a simple entry point, offering a clean overview of the app before
accessing any functionality.
At this stage, no interaction with backend services is required. The screen only guides
users toward signing in or creating an account.

\begin{figure}[H]
	\centering
	\includegraphics[width=0.5\textwidth]{Design/1_welcome.png}
	\caption{Welcome Screen Mockup}
	\label{fig:welcome_screen_mockup}
\end{figure}

\subsection{Login Screen}
\label{sec:login_screen}%

After selecting Get Started from the welcome screen, users are taken to the login screen.
This interface allows returning users to authenticate by entering their username and
password. The screen also provides shortcuts to the Sign Up screen for new users and
an option to continue as a Guest, enabling limited access without registration.
The purpose of this screen is to act as the main entry point into the app,
ensuring that only logged-in users can access all features such as trip recording
path creation, and reporting.

\begin{figure}[H]
	\centering
	\includegraphics[width=0.5\textwidth]{Design/2_login.png}
	\caption{Login Screen Mockup}
	\label{fig:login_screen_mockup}
\end{figure}

\subsection{Signup Screen}
\label{sec:signup_screen}%

The sign-up screen enables new users to create an account by providing an email,
username, and password.
This interface is designed to keep the registration process simple, requiring only
the essential information needed to activate a personal profile.
From this screen, users can switch to the Login screen if they already have an account,
or continue as a Guest if they prefer to explore the system without registering.
Creating an account unlocks all core functionalities of BBP, such as recording trips,
submitting reports, managing custom paths, and accessing personal statistics.

\begin{figure}[H]
	\centering
	\includegraphics[width=0.5\textwidth]{Design/3_signup.png}
	\caption{Signup Screen Mockup}
	\label{fig:signup_screen_mockup}
\end{figure}

\subsection{Home Screen}
\label{sec:home_screen}%

The home screen provides access to the core features of the app through an interactive map
and a bottom navigation bar.
At the top of the screen, users can specify an origin and a destination to initiate a
path search. The origin will be automatically set to the user's current location by default,
but can be modified as needed. The Search Path button triggers the path computation process,
and results are later displayed.
The interactive map occupies most of the screen and shows the user’s current position.
A floating button, rendered as a marker icon with a plus symbol in the lower-right corner,
allows logged-in users to create a new path and enter the creation flow with one tap.
The bottom navigation bar grants access to all main sections of the app
(home, trip history, path history, and user profile). All functionalities are fully
available for logged-in users.

Guest users access a simplified version of the home screen.
While the map and the search form (origin, destination, and Search Path button) remain
available, all other features require authentication and are therefore visually disabled.
Items in the bottom navigation bar appear greyed out, indicating restricted access.
If the user attempts to tap any disabled icon, the app displays a pop-up
prompting them to sign up or log in in order to unlock the full feature set.
This layout clearly communicates the distinction between browsing capabilities and the
additional features unlocked through registration.

\begin{figure}[H]
	\centering
	\begin{minipage}{0.48\textwidth}
		\centering
		\includegraphics[width=\textwidth]{Design/4_home.png}
		\caption{Home Screen Mockup}
		\label{fig:home_screen_mockup}
	\end{minipage}
	\hfill
	\begin{minipage}{0.48\textwidth}
		\centering
		\includegraphics[width=\textwidth]{Design/4_home_guest.png}
		\caption{Home Screen Mockup for Guest Users}
		\label{fig:home_screen_mockup_guest}
	\end{minipage}
\end{figure}

\subsection{Authentication Pop-up for Guest Users}
\label{sec:authentication_pop_up_guest_users}%

When a guest user attempts to interact with any restricted feature, the app
displays a modal pop-up overlay.
The pop-up clearly communicates that advanced functionalities
are available only to registered or logged-in users. It provides two action buttons
that redirect the user to the authentication flow.
The rest of the interface is dimmed to highlight the modal and prevent interaction with the
disabled elements. If the user taps outside the pop-up, the modal closes and the guest user
is returned to the current screen, allowing them to continue exploring the map or searching
for bike paths without interruption.

\begin{figure}[H]
	\centering
	\includegraphics[width=0.5\textwidth]{Design/5_unlock.png}
	\caption{Authentication Pop-up Mockup for Guest Users}
	\label{fig:authentication_pop_up_guest_users_mockup}
\end{figure}

\subsection{Search Results}
\label{sec:search_results}%

After submitting an origin and a destination, the app displays the search results showing
all suggested paths between the two points. Each result includes the path name, its current condition
(e.g., \textbf{Optimal}, \textbf{Requires Maintenance}), and the number of aggregated reports supporting
that evaluation. The optimal path found is also displayed on the map, which provides an estimate of
distance and travel time.

If the user taps the search bar and enters a new origin or destination, the current results are cleared.
The interface returns to the initial state by showing the \textbf{Search Paths} button again, allowing
the user to perform a new search with the updated locations.

The overall structure of the screen is identical for both logged-in and guest users. However,
only logged-in users can interact with advanced features. In guest mode, these elements, as well as
their corresponding icons in the bottom navigation bar, appear disabled.

\begin{figure}[H]
	\centering
	\begin{minipage}{0.48\textwidth}
		\centering
		\includegraphics[width=\textwidth]{Design/6_search.png}
		\caption{Search Results Mockup}
		\label{fig:search_results_mockup}
	\end{minipage}
	\hfill
	\begin{minipage}{0.48\textwidth}
		\centering
		\includegraphics[width=\textwidth]{Design/6_search_guest.png}
		\caption{Search Results Mockup for Guest Users}
		\label{fig:search_results_mockup_guest}
	\end{minipage}
\end{figure}

\subsection{Path Selection}
\label{sec:path_selection}%

When the user taps on one of the suggested paths in the results list, BBP selects that path
and displays it on the map. The selection is visually indicated by turning the textual
information (name, condition, aggregated reports) blue while the card background remains
white, making the active suggestion stand out from the others. Tapping the same path again
deselects it, while tapping a different path updates the selection accordingly.

The map visualization shows the complete path, the estimated travel time and distance.
If the selected path contains confirmed reports, these are displayed directly on the map
using warning icons, allowing the user to immediately identify critical or deteriorated segments.
Tapping one of these markers opens a small dialog containing the associated report details,
providing additional insight into the nature and current status of the issue.

Editing either the origin or destination input fields automatically clears the current selection
and removes the existing results. Once the user submits the updated values, the interface enables
a new search.

The behaviour is identical for both logged-in and guest users, except that restricted actions
are disabled for guests and remain accessible only to logged-in users.

Multiple mockups are provided to illustrate different scenarios, including paths with no reports
and paths containing obstacle indicators.

\begin{figure}[H]
	\centering
	\begin{minipage}{0.48\textwidth}
        \centering
        \includegraphics[width=\textwidth]{Design/7_select.png}
        \caption{Path Selection Mockup}
        \label{fig:path_selection_no_reports_mockup}
    \end{minipage}
    \hfill
    \begin{minipage}{0.48\textwidth}
        \centering
        \includegraphics[width=\textwidth]{Design/7_select_guest.png}
        \caption{Path Selection Mockup for Guest Users}
        \label{fig:path_selection_no_reports_guest_mockup}
    \end{minipage}
\end{figure}

\begin{figure}[H]
    \centering
    \begin{minipage}{0.48\textwidth}
        \centering
        \includegraphics[width=\textwidth]{Design/7_select_report.png}
        \caption{Path Selection Mockup with Report Markers}
        \label{fig:path_selection_with_reports_mockup}
    \end{minipage}
    \hfill
    \begin{minipage}{0.48\textwidth}
        \centering
        \includegraphics[width=\textwidth]{Design/7_select_report_guest.png}
        \caption{Path Selection Mockup with Report Markers for Guest Users}
        \label{fig:path_selection_with_reports_guest_mockup}
    \end{minipage}
\end{figure}

\subsection{Path Preview}
\label{sec:path_preview}%

If the origin specified in the search does not match the user's current physical location,
the system allows the user to view the results but prevents trip initiation. In this case,
the \textbf{Start Trip} button will not appear. The user can explore the path but cannot
begin navigation until the actual position coincides with the search origin.

This behaviour is consistent for both logged-in and guest users, with the latter
having limited access to restricted features.

\begin{figure}[H]
    \centering
    \begin{minipage}{0.48\textwidth}
        \centering
        \includegraphics[width=\textwidth]{Design/8_search_no_origin.png}
        \caption{Detailed Path Preview Mockup}
        \label{fig:detailed_path_preview_mockup}
    \end{minipage}
    \hfill
    \begin{minipage}{0.48\textwidth}
        \centering
        \includegraphics[width=\textwidth]{Design/8_search_no_origin_guest.png}
        \caption{Detailed Path Preview Mockup for Guest Users}
        \label{fig:detailed_path_preview_guest_mockup}
    \end{minipage}
\end{figure}

\subsection{Automatic Mode Activation}
\label{sec:automatic_mode_activation}%

After tapping the \textbf{Start Trip} button, logged-in users are prompted to choose
whether to enable the Automatic Mode. This mode activates automatic obstacle detection
and data collection using the available external sensors. Even when Automatic Mode is
enabled, the user may still submit manual reports at any time by tapping the dedicated
report icon during the trip.
Once the user makes a selection, the app immediately starts the
trip session and begins tracking the user's movement along the selected path.

\begin{figure}[H]
	\centering
	\includegraphics[width=0.5\textwidth]{Design/9_mode.png}
	\caption{Automatic Mode Activation Mockup}
    \label{fig:automatic_mode_activation_mockup}
\end{figure}

\subsection{Navigation View}
\label{sec:navigation_view}%

During an active trip, the app switches to a dedicated navigation view showing the user's
real-time position on the selected path. The path remains highlighted in blue, and the 
interface provides two main controls: a \textbf{Stop Trip} button at the top of the screen 
and a \textbf{report button} for submitting manual obstacle reports.
Both logged-in and guest users can enter this view once they press \textbf{Start Trip}, the
difference lies in what data is retained and which controls remain available.

A trip can end in three situations: when the user presses \textbf{Stop Trip}, when the destination
is reached, or when the system detects that the user has moved significantly away from the planned 
path. In all cases, the app stops the trip and notifies the user that the session has ended.

If the user is logged-in, the completed trip is then saved to their activity history. If the user is
a guest, no information is saved, even though the navigation experience is identical.

The \textbf{report button} allows logged-in users to submit manual reports at any moment during
the trip. This feature is disabled in guest mode, where only the \textbf{Stop Trip} button remains
active. All other trip-related controls, including the bottom navigation bar and the \textbf{report button},
appear greyed out to stress that the trip can be ridden but not stored or enriched with reports.

\begin{figure}[H]
    \centering
    \begin{minipage}{0.48\textwidth}
        \centering
        \includegraphics[width=\textwidth]{Design/10_trip_started.png}
        \caption{Navigation View Mockup}
        \label{fig:navigation_view_mockup}
    \end{minipage}
    \hfill
    \begin{minipage}{0.48\textwidth}
        \centering
        \includegraphics[width=\textwidth]{Design/10_trip_started_guest.png}
        \caption{Navigation View Mockup for Guest Users}
        \label{fig:navigation_view_guest_mockup}
    \end{minipage}
\end{figure}

\subsection{Trip Completion}
\label{sec:trip_completion}%

At the end of a trip, the app displays a pop-up informing the user that the navigation 
session has been completed. The pop-up appears both when the user manually stops the trip and 
when the system ends it automatically upon reaching the destination or detecting a significant
deviation from the planned path.

For logged-in users, the pop-up includes a \textbf{Check Summary} button that redirects them
directly to the summary screen of the completed trip. The pop-up can also be dismissed by tapping
anywhere outside the dialog.

In guest mode, no summary is available because guest trips are not stored. The pop-up therefore
contains no action buttons and simply closes when the user taps outside the dialog, returning
them to the map.

\begin{figure}[H]
    \centering
    \begin{minipage}{0.48\textwidth}
        \centering
        \includegraphics[width=\textwidth]{Design/11_finished_trip.png}
        \caption{Trip Completion Mockup}
        \label{fig:trip_completion_mockup}
    \end{minipage}
    \hfill
    \begin{minipage}{0.48\textwidth}
        \centering
        \includegraphics[width=\textwidth]{Design/11_finished_trip_guest.png}
        \caption{Trip Completion Mockup for Guest Users}
        \label{fig:trip_completion_guest_mockup}
    \end{minipage}
\end{figure}

\subsection{Report Submission}
\label{sec:report_submission}%

During an active trip, users may submit reports through a dedicated pop-up interface.
This dialog appears in two situations: when the user manually taps the \textbf{report button}, or
when the Automatic Mode detects a potential obstacle. In the manual case, the fields
are initially empty and must be filled in by the user. When the pop-up is triggered by an
automatic detection, the form is pre-compiled with the estimated path status and obstacle
type inferred from sensor data, although the user remains free to modify or discard the
report.

Only logged-in users can complete this action. In guest mode, the
report function is disabled and the corresponding button appears greyed out.

\begin{figure}[H]
	\centering
	\includegraphics[width=0.5\textwidth]{Design/12_report.png}
	\caption{Report Submission Mockup}
	\label{fig:report_submission_mockup}
\end{figure}

\subsection{Report Confirmation}
\label{sec:report_confirmation}%

While following a path during an active trip, the system notifies logged-in users
when they pass near a location where a previous report was submitted by another user.
In these cases, a confirmation pop-up appears, asking whether the reported obstacle is
still present. The dialog offers two options: \textbf{Confirm}, to validate the existing
report, or \textbf{Reject}, to indicate that the obstacle is no longer observed.

If the user does not interact with the pop-up within a short timeout period, the dialog
automatically disappears to avoid interrupting the navigation experience. This feature is
available only for logged-in users. Guest users are not prompted for report
confirmations and cannot participate in the validation process.

\begin{figure}[H]
	\centering
	\includegraphics[width=0.5\textwidth]{Design/13_confirm_report.png}
	\caption{Report Confirmation Mockup}
	\label{fig:report_confirmation_mockup}
\end{figure}

\subsection{Path Creation}
\label{sec:path_creation}%

When the user selects the path creation feature, a pop-up form appears requesting the
basic metadata required to define a new bike path. The form includes fields for the path
title, an optional description, the desired visibility level (e.g., public or private), and the
creation mode. The visibility setting determines whether the path will be accessible to all users
or only to the creator. It will be pre-set according to the user's preferences in the Settings screen,
but can be modified for each new path.

The \textbf{Creation Mode} determines how the new path will be generated. If the user
selects the manual mode, the path will be drawn directly on the map by placing and
adjusting waypoints. If the automatic mode is selected, the user will start cycling and the
app will record the GPS trace of the trip, using it as the final path geometry once
the recording is completed.

After filling in the required fields, the user may proceed by tapping \textbf{Start Creation}
or dismiss the dialog with the \textbf{Close} button.

Only logged-in users can create new paths.

\begin{figure}[H]
	\centering
	\includegraphics[width=0.5\textwidth]{Design/14_creation.png}
	\caption{Path Creation Mockup}
	\label{fig:path_creation_mockup}
\end{figure}

\subsection{Creation View}
\label{sec:finish_creation}%

During the path creation, the app switches to a dedicated creation view showing the user's
real-time position on the selected path if the automatic mode is chosen, or allowing the user to
draw the path by placing waypoints on the map in manual mode.

The interface provides a \textbf{Finish Creation} button at the top of the screen, which 
the user can tap to complete the process and save the newly created path together with the 
metadata provided in the initial form.
After the path is successfully stored, the app displays a confirmation message 
informing the user that the new path has been saved.

If the user wishes to cancel the creation, they can
tap the home icon in the bottom navigation bar, which will discard the current path and return
them to the home screen.

These actions are available only to logged-in users.

\begin{figure}[H]
	\centering
	\includegraphics[width=0.5\textwidth]{Design/15_finish_creation.png}
	\caption{Creation View Mockup}
	\label{fig:finish_creation_mockup}
\end{figure}

\subsection{Creation Completion}
\label{sec:creation_completion}%

After completing the creation workflow, the system confirms the successful creation of the new path
by displaying a pop-up notification. The dialog informs the user that the path has been saved and
provides a single action button, \textbf{Check Path}, which redirects directly to the detail screen
of the newly created path.

The pop-up can also be dismissed by tapping anywhere outside the dialog, returning the user to the
map view. Only logged-in users can access this confirmation dialog, as guests cannot create or store
custom paths.

\begin{figure}[H]
	\centering
	\includegraphics[width=0.5\textwidth]{Design/16_creation_success.png}
	\caption{Creation Completion Mockup}
	\label{fig:creation_completion_mockup}
\end{figure}

\subsection{Trip History Screen}
\label{sec:trip_history_page}%

The \textbf{My Trips} screen provides logged-in users with an overview of all previously
recorded trips. It is accessible through the second icon in the bottom
navigation bar.

At the top of the screen, a small arrow icon appears next to the \textbf{My Trips} title.
Tapping this icon opens a compact overlay menu that allows the user to choose how the trips
should be sorted, for example by date, distance, or alphabetical order. 
Selecting an option immediately updates the ordering of the list.

Each entry displays the destination name, date, total distance, and trip
duration. Tapping on a trip record expands it to show a detailed summary, including a
map visualization of the followed path, the weather conditions at the time of the trip,
and additional metrics such as average speed and elevation level. Tapping the same
record again collapses the details, returning the list to its compact form.

If the displayed weather badge is selected, the app reveals additional
weather-related information, providing a more complete snapshot of the environmental
conditions encountered during the trip.

If the path contained confirmed reports, these are displayed directly on the map using
warning markers, allowing the user to identify critical segments encountered during the
trip. Tapping on a marker opens a small dialog showing the details of the corresponding
report, enabling the user to inspect the nature and severity of the issue encountered on
that segment.

This screen presents all trip-related information in a unified and structured format,
making it easy for users to review past sessions and compare their performance over
time.

This functionality is available only for logged-in users.

\begin{figure}[H]
	\centering
	\includegraphics[width=0.5\textwidth]{Design/17_trips.png}
	\caption{Trip History Screen Mockup}
	\label{fig:trip_history_screen_mockup}
\end{figure}

\subsection{Path History Screen}
\label{sec:path_history_screen}%

The \textbf{My Paths} screen displays all custom bike paths created by the logged-in
user.  It is accessible through the third icon in the bottom
navigation bar, represented by the bike icon.

At the top of the screen, a small arrow icon appears next to the \textbf{My Paths} title.
Tapping this icon opens a compact overlay menu that allows the user to choose how the paths
should be sorted, for example by date, distance, or alphabetical order. 
Selecting an option immediately updates the ordering of the list.

Each entry shows the path title, date of creation, total distance, and estimated
duration. Tapping on a path expands the entry and reveals a preview of the path on the
map. Tapping the same entry again collapses the view and returns the list to its compact layout.

Each path provides a set of management actions. The visibility icon allows the user to
toggle the path's visibility between public and private, while the delete icon permanently
removes the path from the user's collection. 

When the user selects the \textbf{Start} button inside a path entry, the app redirects
them to the home screen with the origin and destination fields automatically filled with the
path's endpoints, and the corresponding path already selected. The user may then start
the trip directly, provided that their current physical position matches the starting point
of the path.

This section is accessible only to logged-in users. Guest users cannot create, view,
or manage custom paths.

\begin{figure}[H]
	\centering
	\includegraphics[width=0.5\textwidth]{Design/18_paths.png}
	\caption{Path History Screen Mockup}
	\label{fig:path_history_screen_mockup}
\end{figure}

\subsection{Profile Screen}
\label{sec:profile_screen}%

The \textbf{Profile} screen provides logged-in users with an overview of their personal
statistics and riding activity. It is accessible through the fourth icon in the bottom
navigation bar, represented by the avatar icon.

At the top of the screen, users can view their profile information together with aggregated
metrics such as total distance travelled and total number of recorded trips. A small pencil
icon appears next to the user's name, tapping it opens a dedicated screen where the user can
update their personal details, including name, email, and password. A shortcut is also provided
to start a new trip directly from this section, that will redirect the user to the home screen
when selected.

The screen includes a summary of the user's most recent trip, followed by a weekly analysis
showing the average ride duration for each day of the week. Additional statistics, such as
total weekly distance, average speed, and elevation level—are presented in a dedicated
section to offer a comprehensive overview of the user's performance and progress over time.

A calendar icon is available within the statistics area. Tapping it opens a pop-up that allows
the user to filter the displayed metrics, selecting, for example, monthly statistics or data
related to a specific day.

A settings icon is also available in the top-right corner of the screen. Selecting it redirects
the user to the \textbf{Settings} screen, where personal preferences and privacy options can be
configured.

This screen is available exclusively to logged-in users.

\begin{figure}[H]
	\centering
	\includegraphics[width=0.5\textwidth]{Design/19_profile.png}
	\caption{Profile Screen Mockup}
	\label{fig:profile_screen_mockup}
\end{figure}

\subsection{Settings Screen}
\label{sec:settings_screen}

The \textbf{Settings} screen, accessible from the Profile section, allows logged-in users
to customize their app preferences and manage their privacy settings. The screen
includes options such as enabling or disabling Dark Mode and selecting the preferred
data sharing level (e.g., public or private). 

The \textbf{Contact Us} option redirects the user to the app's store screen (Google
Play Store or Apple App Store, depending on the platform), where they can access the
official support channel or submit feedback.

At the bottom of the screen, the user can log out from their account through the
\textbf{Log Out} button, which immediately terminates the current session and returns the
app to guest mode.

This screen is available only to logged-in users, as guest users do not manage personal
preferences or sharing settings.

\begin{figure}[H]
	\centering
	\includegraphics[width=0.5\textwidth]{Design/20_settings.png}
	\caption{Settings Screen Mockup}
	\label{fig:settings_screen _mockup}
\end{figure}


\subsection{Personal Information}
\label{sec:personal_informations}%

The \textbf{Personal Information} screen allows logged-in users to update their account details.
The screen displays editable fields for the username, email address, and password.
To change the password, the user must enter the new password twice, once in the \textbf{Password}
field and again in the \textbf{Repeat Password} field, to ensure consistency.

Once the desired changes have been made, the user can tap the \textbf{Save Changes} button to submit
the updated information. The app validates the input and, if the update is successful,
stores the new account details and returns the user to the previous screen.

This screen is accessible exclusively to logged-in users and can be reached from the \textbf{Profile}
screen by tapping the pencil icon next to the user's name. To return to the profile view, the user can
simply tap the profile icon again in the bottom navigation bar.

\begin{figure}[H]
    \centering
    \includegraphics[width=0.5\textwidth]{Design/21_personal_info.png}
    \caption{Personal Information Mockup}
    \label{fig:personal_information_mockup}
\end{figure}

\subsection{Error Pop-ups}
\label{sec:error_popups}%

The app includes a global error-handling mechanism to ensure consistent feedback
across all user interactions. Whenever an unexpected issue occurs, such as a network failure,
invalid input, or an internal processing error, the system displays a dedicated error pop-up.
The dialog replaces the message with the specific error description relevant to the situation.

The pop-up is dismissed when the user taps anywhere outside the dialog. This behaviour is
consistent across the entire app.

This error pop-up may appear during any operation, including searching for paths, 
starting or saving trips, submitting reports, or managing custom paths.

\begin{figure}[H]
	\centering
	\includegraphics[width=0.5\textwidth]{Design/22_error.png}
	\caption{Error Pop-up Mockup}
	\label{fig:error_popups_mockup}
\end{figure}
