\newcommand{\phoneimage}[2][]{%
	\begin{tikzpicture}[baseline=(img.base)]
		\node[
			inner sep=0pt,
			outer sep=0pt,
			rounded corners=14pt,
			clip
		] (img) {\includegraphics[#1]{#2}};

		\draw[
			rounded corners=14pt,
			line width=0.9pt,
			color=black!65
		] (img.north west) rectangle (img.south east);

		\path (img.north west) -- (img.north east)
		coordinate[pos=0.5] (topcenter);

		\draw[
			rounded corners=3pt,
			fill=black,
			draw=black
		]
		($(topcenter)+(-0.60cm,-0.16cm)$)
		rectangle
		($(topcenter)+(0.60cm,-0.30cm)$);

		\path (img.south west) -- (img.south east)
		coordinate[pos=0.5] (bottomcenter);

		\draw[
			rounded corners=2pt,
			fill=white!92,
			draw=none
		]
		($(bottomcenter)+(-0.38cm,0.20cm)$)
		rectangle
		($(bottomcenter)+(0.38cm,0.25cm)$);
	\end{tikzpicture}%
}

All mockups presented in this chapter are conceptual prototypes designed to illustrate the expected
interaction flow, the structural layout of each screen, and the different behaviours available to both
guest and logged-in users. While the visual style is representative of the intended interface,
refinements may occur during implementation.

This chapter provides a high-level overview of the system’s user experience. Its purpose is not to
specify implementation details, but to visually and conceptually describe how users navigate the application
across its main functionalities: authentication, map exploration, path search and selection, trip navigation,
obstacle reporting, custom path creation and management, and profile configuration. The mockups also
highlight how certain capabilities vary depending on the user state, guest or logged-in,
ensuring a clear understanding of the overall interaction model.

\section{Navigation Flow}
\label{sec:navigation_flow}%

From a global perspective, the application is organised around a small set of core areas:
the authentication entry point (Welcome, Login, Signup), the map-based Home Screen, trip-related features
(navigation, reporting, history), path management features (creation and created paths), and the personal
area (Profile, Edit Profile, Settings). The bottom navigation bar connects Home, Trip History, My Paths,
and Profile, and adopts a modern auto-hiding behaviour: it becomes hidden while scrolling downward to
maximise vertical space and reappears when the user scrolls upward, reinforcing spatial orientation
without obstructing content.

Navigation begins at the Welcome Screen, where users may authenticate or proceed as guests. In either case,
they are redirected to the Home Screen, which functions as the central hub for all subsequent interactions.
From here, users can search for paths, explore search results, begin trip navigation along a selected route,
or, if logged-in, initiate the creation of a custom path. During active navigation, the interface
transitions to a dedicated full-screen Trip Navigation View, and automatically returns to the map once the
trip is completed.

Trip exploration is accessible at any moment through the Trip History Screen, reachable via the navigation
bar. Here, users can review previous activities and inspect detailed trip information. A similar
list-and-detail pattern governs the My Paths Screen, which enables users to manage their custom paths,
start navigation from a saved route, or adjust visibility and deletion settings. Account-related
functionality is grouped within the Profile Screen and its associated views (Edit Profile and Settings),
accessible directly from the navigation bar or through the profile’s settings shortcut.

Across all flows, the application clearly differentiates between guest and logged-in users: guests may
browse the map and perform basic navigation, while logged-in users gain access to trip recording, obstacle
reporting and validation, custom path creation and management, and trip statistics. Several cross-cutting
behaviours, such as automatic ride detection, authentication prompts for restricted actions, and global
error pop-ups, may appear on top of the current interface without altering the underlying navigation
structure, ensuring that interaction remains consistent and predictable across different usage scenarios.

\section{Authentication and Guest Access}
\label{sec:authentication_and_guest_access}%

\begin{figure}[H]
	\centering
	\begin{minipage}[t]{0.31\textwidth}
		\centering
		\vspace{0pt}
		\phoneimage[width=\textwidth]{Design/1_welcome.png}
		\caption{Welcome Screen}
		\label{fig:welcome_screen_mockup}
	\end{minipage}
	\hfill
	\begin{minipage}[t]{0.31\textwidth}
		\centering
		\vspace{0pt}
		\phoneimage[width=\textwidth]{Design/2_login.png}
		\caption{Login Screen}
		\label{fig:login_screen_mockup}
	\end{minipage}
	\hfill
	\begin{minipage}[t]{0.31\textwidth}
		\centering
		\vspace{0pt}
		\phoneimage[width=\textwidth]{Design/3_signup.png}
		\caption{Signup Screen}
		\label{fig:signup_screen_mockup}
	\end{minipage}
\end{figure}

\subsection*{Welcome Screen}
\label{sec:welcome_screen}%

The Welcome Screen introduces the user to the application through a minimal layout featuring the app
logo, name, and a short tagline summarising its purpose: discovering bike paths, tracking rides, and
engaging with the cycling community. Three navigation options are provided: \textbf{Sign Up},
\textbf{Log In}, and \textbf{Continue as Guest}. The first leads new users to the registration flow,
the second allows returning users to access their existing account, and the third offers immediate
exploration with limited functionalities.

This screen acts as the entry point to all subsequent interactions. It does not require backend
communication and is intentionally designed to be simple, guiding users toward authentication or guest
access in an intuitive way.

\subsection*{Login Screen}
\label{sec:login_screen}%

After selecting the \textbf{Log In} option from the Welcome Screen, users are taken to the Login
Screen, where they can authenticate by entering their email address and password. A secondary action
allows navigation back to the Welcome Screen.

Successful authentication unlocks all features requiring a validated identity, including trip recording
with persistent storage, obstacle reporting, custom path creation, and access to personal statistics.
The screen focuses on clarity and ease of use, maintaining a clean layout that prioritises the login
form.

\subsection*{Signup Screen}
\label{sec:signup_screen}%

The Signup Screen enables new users to create an account by entering a username, email address,
password, and password confirmation. The interface intentionally keeps the registration process
lightweight, requesting only the essential information required for profile creation. Users may return
to the Welcome Screen through the dedicated navigation control.

Creating an account grants access to all core functionalities of the system: trip recording, obstacle
reporting and validation, custom path management, and statistics, providing a complete and personalised
experience.

\section{Home for Logged-in and Guest Users}
\label{sec:home_for_logged_in_and_guest_users}%

\begin{figure}[H]
	\centering
	\begin{minipage}[t]{0.31\textwidth}
		\centering
		\vspace{0pt}
		\phoneimage[width=\textwidth]{Design/4_home.png}
		\caption{Home Screen}
		\label{fig:home_screen_mockup}
	\end{minipage}
	\hfill
	\begin{minipage}[t]{0.31\textwidth}
		\centering
		\vspace{0pt}
		\phoneimage[width=\textwidth]{Design/4_home_guest.png}
		\caption{Home Screen for Guests}
		\label{fig:home_screen_mockup_guest}
	\end{minipage}
	\hfill
	\begin{minipage}[t]{0.31\textwidth}
		\centering
		\vspace{0pt}
		\phoneimage[width=\textwidth]{Design/5_unlock.png}
		\caption{Authentication Pop-up}
		\label{fig:authentication_pop_up_guest_users_mockup}
	\end{minipage}
\end{figure}

\subsection*{Home Screen}
\label{sec:home_screen}%

The Home Screen serves as the central hub of the BBP application, providing access to map exploration,
path search, and all major navigation routes. At the top of the interface, users can specify a
Starting Point and a Destination through two input fields. By default, the Starting Point is set to
the user's current GPS location, but it may be manually edited when needed. Once both fields are
filled, the \textbf{Find Paths} button retrieves all available cycling routes connecting the selected
points.

The interactive map occupies most of the screen, displaying the user's position along with any computed
paths. In the lower-right corner, authenticated users see a floating action button with a plus
icon, which opens the custom path creation flow. For guest users, this appears visually disabled.

A persistent bottom navigation bar grants quick access to the core sections of the application:
\textbf{Home}, \textbf{Trip History}, \textbf{My Paths}, and \textbf{Profile}. The bar adopts an
auto-hiding behaviour: it hides while scrolling downward to maximise map visibility and reappears when
scrolling upward, ensuring fluid navigation without obstructing content.

Guest users may freely interact with the map and perform path searches. However, features requiring
authentication,such as path creation, viewing personal trip history, or accessing the profile, are
greyed out. A lock icon visually indicates restricted access. Attempting to select a disabled icon
triggers an authentication pop-up inviting the user to sign in. This design clearly communicates
the distinction between publicly accessible features and those reserved for authenticated users.

\subsection*{Authentication Pop-up for Guest Users}
\label{sec:authentication_pop_up_guest_users}%

When a guest user attempts to access a restricted capability, the system displays a modal pop-up
explaining that the selected functionality is available only to logged-in users. The modal presents
two actions: one to proceed to the Sign In flow, and one to continue as a guest without enabling
the requested feature.

While the pop-up is visible, the underlying interface is dimmed and temporarily disabled to focus
the user's attention on the modal and prevent accidental interaction with the map or navigation
controls. The pop-up may be closed either by selecting one of the available actions or by tapping
outside the modal area, returning the guest user to their current context and allowing them to
continue exploring the application uninterrupted.

\section{Path Search}
\label{sec:path_search}%

\begin{figure}[H]
	\centering
	\begin{minipage}[t]{0.24\textwidth}
		\centering
		\vspace{0pt}
		\phoneimage[width=\textwidth]{Design/6_search.png}
		\caption{Search Results}
		\label{fig:search_results_mockup}
	\end{minipage}
	\hfill
	\begin{minipage}[t]{0.24\textwidth}
		\centering
		\vspace{0pt}
		\phoneimage[width=\textwidth]{Design/6_search_guest.png}
		\caption{Search Results for Guests}
		\label{fig:search_results_guests_mockup}
	\end{minipage}
	\hfill
	\begin{minipage}[t]{0.24\textwidth}
		\centering
		\vspace{0pt}
		\phoneimage[width=\textwidth]{Design/22_biking_prompt.png}
		\caption{Ride Detection}
		\label{fig:automatic_ride_detection_popup_mockup}
	\end{minipage}
	\hfill
	\begin{minipage}[t]{0.24\textwidth}
		\centering
		\vspace{0pt}
		\phoneimage[width=\textwidth]{Design/22_biking_prompt_guest.png}
		\caption{Ride Detection for Guests}
		\label{fig:automatic_ride_detection_popup_guest_mockup}
	\end{minipage}
\end{figure}

\subsection*{Search Results}
\label{sec:search_results}%

After submitting a path search from the Home Screen, the application displays an
\textit{Available Paths} panel anchored to the top of the map. The panel lists all suggested
routes that match the selected criteria. Each entry shows the path name, a short description
when available, the estimated distance and travel time, and the current condition of the path
(e.g., Optimal, Maintenance), accompanied by the number of aggregated reports.

The underlying map remains visible as contextual background, centered on the queried area,
allowing users to visually understand where each suggested route is located while scrolling
through the list. A close icon in the upper-right corner dismisses the panel and restores the
standard map view, enabling the user to adjust the search parameters or run a new query.

The behaviour is the same for logged-in and guest users, aside from the general limitations
applicable to guests throughout the application.

\subsection*{Automatic Ride Detection}
\label{sec:automatic_ride_detection}%

While the application is open, BBP continuously monitors movement patterns to detect whether
the user is biking without having manually started a trip. When such behaviour is identified,
the system displays a modal prompt at the top of the map asking: “Are you biking?”. This prompt
serves as a reminder to begin tracking so that the ride can be recorded properly.

The pop-up dims the rest of the interface and presents a single primary action. Selecting it
initiates the standard trip-start flow. Tapping outside the modal closes the prompt and returns
the user to the active screen without starting a trip.

Detection and interaction behave identically for guests and authenticated users, however, only
logged-in users will have their recorded trips saved and included in their personal statistics,
consistent with the restrictions outlined in other parts of the interface.

\section{Path Selection and Trip Start}
\label{sec:path_selection_and_trip_start}%

\begin{figure}[H]
	\centering
	\begin{minipage}[t]{0.31\textwidth}
		\centering
		\vspace{0pt}
		\phoneimage[width=\textwidth]{Design/7_select.png}
		\caption{Path Selection}
		\label{fig:path_selection_mockup}
	\end{minipage}
	\hfill
	\begin{minipage}[t]{0.31\textwidth}
		\centering
		\vspace{0pt}
		\phoneimage[width=\textwidth]{Design/7_select_guest.png}
		\caption{Path Selection for Guests}
		\label{fig:path_selection_mockup_guest}
	\end{minipage}
	\hfill
	\begin{minipage}[t]{0.31\textwidth}
		\centering
		\vspace{0pt}
		\phoneimage[width=\textwidth]{Design/8_mode.png}
		\caption{Automatic Mode Activation}
		\label{fig:automatic_mode_activation_popup_mockup}
	\end{minipage}
\end{figure}

\subsection*{Path Selection}
\label{sec:path_selection}%

When the user taps on one of the suggested paths in the results list, BBP highlights the selected
option by applying a blue outline around the corresponding card. This visual cue identifies the
currently active path while keeping the rest of the list unchanged. Tapping a different card updates
the selection, tapping the same card again deselects it, and tapping outside the panel leaves the
current state unchanged.

Once a path is selected, the system displays the corresponding route on the map using a bold blue
polyline, together with its estimated distance and travel time. This provides an immediate visual
representation of the full route. If the path includes confirmed reports, the associated markers
are also rendered on the map, enabling users to identify potentially problematic areas along the
way.

A \textbf{Start Trip} button appears within the selected card only when the path origin matches the
user's current GPS position. If the origin differs, the user may still inspect the route preview,
but the trip cannot be started from the current location.

When the Start Trip button is available, selecting it opens a pop-up for logged-in users asking
whether they wish to enable Automatic Report Mode. After choosing a mode, the trip begins.

Guest users may start navigation from a selected path, but their activity is not stored and no
automatic detection mechanisms are available.

Pressing the close icon at the top of the results panel dismisses the list and restores the
standard map view without an active selection.

Map visualisation, path highlighting, and browsing behaviour are consistent across guest and
authenticated users. However, only logged-in users may access advanced features from the bottom
navigation bar, which remains visually disabled for guests.

\subsection*{Automatic Mode Activation}
\label{sec:automatic_mode_activation}%

After tapping the Start Trip button, logged-in users are presented with a pop-up asking whether
they wish to enable Automatic Report Mode. When activated, this mode leverages the device's
sensors (gyroscope and accelerometer) to detect anomalies such as potholes or rough surfaces.
Upon detecting an irregularity, BBP generates a pre-filled report and displays a confirmation
prompt for the user to review, edit, or submit.

Once the user proceeds, the trip begins and the system starts tracking movement along the selected
route. Automatic detection does not replace manual reporting: users may submit manual reports at
any time during the trip through the dedicated interface.

\section{Navigation and Trip Completion}
\label{sec:navigation_and_trip_completion}%

\begin{figure}[H]
	\centering
	\begin{minipage}[t]{0.24\textwidth}
		\centering
		\vspace{0pt}
		\phoneimage[width=\textwidth]{Design/9_in_trip.png}
		\caption{Navigation View}
		\label{fig:navigation_view_mockup}
	\end{minipage}
	\hfill
	\begin{minipage}[t]{0.24\textwidth}
		\centering
		\vspace{0pt}
		\phoneimage[width=\textwidth]{Design/9_in_trip_guest.png}
		\caption{Navigation View for Guests}
		\label{fig:navigation_view_guest_mockup}
	\end{minipage}
	\hfill
	\centering
	\begin{minipage}[t]{0.24\textwidth}
		\centering
		\vspace{0pt}
		\phoneimage[width=\textwidth]{Design/10_trip_completed.png}
		\caption{Trip Completion}
		\label{fig:trip_completion_mockup}
	\end{minipage}
	\hfill
	\begin{minipage}[t]{0.24\textwidth}
		\centering
		\vspace{0pt}
		\phoneimage[width=\textwidth]{Design/10_trip_completed_guest.png}
		\caption{Trip Completion for Guests}
		\label{fig:trip_completion_guest_mockup}
	\end{minipage}
\end{figure}

\subsection*{Navigation View}
\label{sec:navigation_view}%

During an active trip, the interface transitions to a dedicated full-screen Navigation View.
The map occupies the entire display, showing the user’s real-time position and the selected path
highlighted with a thick blue polyline. Bottom navigation elements are hidden in this mode to
ensure that the user remains fully focused on the ongoing ride.

A prominent \textbf{Complete Trip} button is anchored at the bottom of the screen, allowing the user
to manually end the session at any time. On the lower-right side, a floating report button is shown:
red and active for logged-in users, enabling manual obstacle reporting, greyed out for guest users,
who cannot submit reports. Attempting to press the disabled button triggers an authentication pop-up,
inviting the guest to sign in. The floating placement and colour clearly communicate whether reporting
is available.

A trip ends when the user presses the Complete Trip button, upon reaching the destination, or when
the system detects a significant deviation from the planned route. Once the trip terminates, BBP
stops tracking and displays a confirmation message.

For authenticated users, the recorded trip is stored in their personal history. Guest users experience
the same navigation interface, but no information is saved after the session concludes.

This view visually reinforces the functional distinction between user types: while both may follow a
selected route, only logged-in users can submit reports and preserve their trip data.

\subsection*{Trip Completion}
\label{sec:trip_completion}%

When a trip ends, regardless of whether it was completed manually, automatically, or due to route
deviation, the system displays a completion pop-up summarising the outcome of the session.

For logged-in users, the pop-up shows the message “Great Ride!” along with a \textbf{View Stats}
button. Selecting this option redirects the user to the Trip History Screen and automatically opens
the detailed summary of the newly completed trip. Tapping outside the modal simply closes it and
restores full map interaction.

For guest users, the pop-up displays “Trip Complete” together with a reminder that trip statistics
are not saved in guest mode. A single button closes the dialog, and no option to view statistics is
offered since guest sessions are never stored.

In all cases, the underlying map is dimmed while the modal is visible, preventing interaction until
the user acknowledges the completion message.

\section{Reports}
\label{sec:reports}%

\begin{figure}[H]
	\centering
	\begin{minipage}[t]{0.31\textwidth}
		\centering
		\vspace{0pt}
		\phoneimage[width=\textwidth]{Design/11_report.png}
		\caption{Report Submission}
		\label{fig:report_submission_mockup}
	\end{minipage}
	\hfill
	\begin{minipage}[t]{0.31\textwidth}
		\centering
		\vspace{0pt}
		\phoneimage[width=\textwidth]{Design/11_report_success.png}
		\caption{Successful Report}
		\label{fig:report_submission_success_popup_mockup}
	\end{minipage}
	\hfill
	\begin{minipage}[t]{0.31\textwidth}
		\centering
		\vspace{0pt}
		\phoneimage[width=\textwidth]{Design/12_confirm_report.png}
		\caption{Report Confirmation}
		\label{fig:report_confirmation_mockup}
	\end{minipage}
\end{figure}

\subsection*{Report Submission}
\label{sec:report_submission}%

During an active trip, logged-in users can submit a report through a dedicated pop-up interface.
The report dialog appears when the user taps the red report button or when an automatic detection
triggers a prompt. The pop-up overlays the current Navigation View and temporarily pauses map
interaction by dimming the background.

The dialog contains two fields: \textbf{Path Condition}, allowing the user to indicate the status
of the affected segment (e.g., Optimal, Requires Maintenance, Closed), and \textbf{Obstacle Type},
specifying the encountered issue (e.g., Obstacle, Path Damage, Work in Progress).

For manual submissions, both fields are initially empty and must be completed before sending the
report. For automatically detected events, the fields are pre-filled with values inferred from
sensor data, but the user may freely adjust the information, confirm it, or discard the prompt
entirely.

After the report is submitted, the system displays a confirmation pop-up titled \textit{“Report
	Submitted”}, thanking the user for contributing to the community. A \textbf{Continue Trip} button
closes the dialog and returns the user to the Navigation View. If the user does not interact with
the confirmation pop-up, it automatically dismisses after a short timeout.

Report submission is available only to logged-in users. In guest mode, the report button remains
visible for consistency but appears greyed out and cannot be pressed.

\subsection*{Report Confirmation}
\label{sec:report_confirmation}%

While following a path during an active trip, BBP notifies logged-in users when they approach a
location where another cyclist previously submitted a report. In such cases, a confirmation pop-up
appears at the top of the screen with the message \textit{“Obstacle Detected”}, indicating that
an issue was reported at that point.

The dialog asks the user to confirm whether the obstacle is still present, offering two options:
one to validate that the issue persists and one to indicate that it is no longer observed.
The underlying map is dimmed to draw attention to the dialog while keeping the navigation context
visible. If the user does not respond within a short timeout, the pop-up automatically closes to
avoid interrupting the ride.

After the user selects an option, a brief confirmation pop-up appears to acknowledge that their
response has been recorded. This secondary dialog automatically closes after a short delay if the
user does not interact with it.

This feature is available exclusively to logged-in users. Guest users do not receive confirmation
prompts and cannot validate existing reports.

\section{Path Creation for Logged-in Users}
\label{sec:path_creation_for_logged_in_users}%

\begin{figure}[H]
	\centering
	\begin{minipage}[t]{0.31\textwidth}
		\centering
		\vspace{0pt}
		\phoneimage[width=\textwidth]{Design/13_create.png}
		\caption{Path Creation }
		\label{fig:path_creation_mockup}
	\end{minipage}
	\hfill
	\begin{minipage}[t]{0.31\textwidth}
		\centering
		\vspace{0pt}
		\phoneimage[width=\textwidth]{Design/14_during_creation.png}
		\caption{Creation View }
		\label{fig:finish_creation_mockup}
	\end{minipage}
	\hfill
	\begin{minipage}[t]{0.31\textwidth}
		\centering
		\vspace{0pt}
		\phoneimage[width=\textwidth]{Design/15_creation_completed.png}
		\caption{Succesful Creation}
		\label{fig:creation_completion_mockup}
	\end{minipage}
\end{figure}

\subsection*{Path Creation}
\label{sec:path_creation}%

When the user selects the path creation feature, the system displays a pop-up form at the top of the
screen requesting the basic information required to define a new bike path. The form includes:
\textbf{Path Name}, where the user provides the title of the route, \textbf{Description}, an optional
field for adding additional details, \textbf{Visibility}, allowing the user to choose whether the
path is public or private, and \textbf{Creation Mode}, which determines how the path will be
constructed, either manually by drawing it on the map or automatically by recording the GPS trace
during a ride.

Once the required fields have been completed, the user may proceed by tapping \textbf{Start Creating},
which closes the dialog and initiates the selected creation flow. If the user chooses not to continue,
the pop-up can be dismissed at any time through the close button in the upper-right corner.

Path creation is available exclusively to logged-in users, this feature is disabled in guest mode.

\subsection*{Creation View}
\label{sec:finish_creation}%

After the user begins constructing a new path, the interface transitions into a dedicated full-screen
Creation View. The map expands to occupy the entire display and all navigation elements are hidden to
ensure an uninterrupted creation experience.

The behaviour of this view depends on the selected creation mode. In \textbf{Automatic Mode}, the map
shows the user's real-time position while cycling. As the user moves, BBP draws a blue polyline that
represents the recorded segment of the path, continuously updating to reflect the evolving GPS trace.
This provides an immediate visual representation of the route as it is being generated.

In \textbf{Manual Mode}, the user defines the route by interacting directly with the map. Each waypoint
placed on the map extends the blue polyline, allowing the user to progressively shape the geometry of
the final path. This mode offers full manual control over the route layout.

At any time, the user may cancel the creation process by tapping the button in the upper-left corner,
which closes the Creation View and discards all progress. When satisfied with the constructed route,
the user can save it by tapping the \textbf{Save Path} button at the bottom of the screen. This action
stores the path along with the metadata previously entered in the creation form.

Access to the Creation View and the ability to save paths are restricted to logged-in users.

\subsection*{Creation Completion}
\label{sec:creation_completion}%

After the new path has been saved, the system displays a confirmation pop-up titled
\textit{“Path Created!”}. The dialog informs the user that the route has been successfully stored and
is now available for use.

The pop-up includes a single action button, \textbf{Go to My Paths}, which redirects the user to the
section containing all the paths they have created. Here, they may inspect the newly stored path in
detail or manage their collection. If the user taps outside the modal, the pop-up simply closes and
the application returns to the map view.

This confirmation appears only for logged-in users, as guests cannot create or save custom paths.

\section{Trip Management for Logged-in Users}
\label{sec:trip_management_for_logged_in_users}%

\vspace{-1em}

\begin{figure}[H]
	\centering
	
	\begin{minipage}[t]{0.29\textwidth}
		\centering
		\vspace{0pt}
		\phoneimage[width=\textwidth]{Design/16_trip_history.png}
		\caption{Trip History}
	\end{minipage}
	\hfill
	\begin{minipage}[t]{0.29\textwidth}
		\centering
		\vspace{0pt}
		\phoneimage[width=\textwidth]{Design/16_trip_order.png}
		\caption{Trip Sorting}
	\end{minipage}
	\hfill
	\begin{minipage}[t]{0.29\textwidth}
		\centering
		\vspace{0pt}
		\phoneimage[width=\textwidth]{Design/16_trip_expanded.png}
		\caption{Trip Details}
	\end{minipage}

	\vspace{0.5em}

	\begin{minipage}[t]{0.29\textwidth}
		\centering
		\vspace{0pt}
		\phoneimage[width=\textwidth]{Design/16_trip_stats.png}
		\caption{Trip Stats}
	\end{minipage}
	\hfill
	\begin{minipage}[t]{0.29\textwidth}
		\centering
		\vspace{0pt}
		\phoneimage[width=\textwidth]{Design/16_trip_weather.png}
		\caption{Trip Weather}
	\end{minipage}
	\hfill
	\begin{minipage}[t]{0.29\textwidth}
		\centering
		\vspace{0pt}
		\phoneimage[width=\textwidth]{Design/16_delete_trip.png}
		\caption{Trip Deletion}
	\end{minipage}
\end{figure}

The Trip History Screen is accessible through the second icon in the bottom navigation bar and
allows logged-in users to review all previously recorded cycling activities. At the top of the
interface, a sorting control is placed beside the screen title. When tapped, it opens a compact
dropdown menu that lets the user reorder the list of trips by date, distance, duration, or
alphabetical order. Selecting an option immediately updates the list to reflect the chosen order.

Trips are presented as compact cards displaying the essential information for each session:
trip name, total distance, duration, and completion date. A delete icon appears on each card, giving
users the ability to remove a trip from their history. Tapping the icon triggers a confirmation
pop-up to prevent accidental deletions.

Tapping a trip card expands it smoothly into a detailed view. In this state, the interface displays
a map preview showing the route taken during the session, enriched with a weather badge placed in
the corner of the map. Beneath the map, a summary section reports the date, duration, and name of
the selected path, followed by a performance panel that includes additional metrics such as distance,
average speed, maximum speed, and elevation.

Tapping the weather badge reveals a detailed meteorological panel containing information such as
temperature, humidity, wind speed, visibility, pressure, and overall weather conditions at the time
of the trip. If the recorded activity includes confirmed reports, their corresponding markers appear
along the displayed route. Selecting a marker opens a small dialog with details about the associated
issue, allowing the user to inspect what was encountered during the ride.

Tapping the trip header collapses the view and returns the interface to the scrollable list of
compact cards. This screen provides a rich and well-structured overview of past cycling activities,
enabling users to explore their performance and recall the conditions of each session.

Trip History is available exclusively to logged-in users, as guest users cannot store or view past
trips.

\section{Paths Management for Logged-in Users}
\label{sec:paths_management_for_logged_in_users}%

\begin{figure}[H]
	\centering
	\begin{minipage}[t]{0.24\textwidth}
		\centering
		\vspace{0pt}
		\phoneimage[width=\textwidth]{Design/17_paths.png}
		\caption{Paths Screen }
		\label{fig:my_paths_screen_mockup}
	\end{minipage}
	\hfill
	\begin{minipage}[t]{0.24\textwidth}
		\centering
		\vspace{0pt}
		\phoneimage[width=\textwidth]{Design/17_path_expanded.png}
		\caption{Path Details}
		\label{fig:my_path_details_mockup}
	\end{minipage}
	\hfill
	\centering
	\begin{minipage}[t]{0.24\textwidth}
		\centering
		\vspace{0pt}
		\phoneimage[width=\textwidth]{Design/17_path_visibility.png}
		\caption{Path Visibility}
		\label{fig:path_visibility_toggle_mockup}
	\end{minipage}
	\hfill
	\begin{minipage}[t]{0.24\textwidth}
		\centering
		\vspace{0pt}
		\phoneimage[width=\textwidth]{Design/17_delete_path.png}
		\caption{Path Deletion}
		\label{fig:path_deletion_confirmation_mockup}
	\end{minipage}
\end{figure}

The My Paths Screen displays all custom bike paths created by the logged-in user and is accessible
through the third icon in the bottom navigation bar. At the top of the interface, a small arrow icon
next to the screen title opens a compact sorting menu, allowing the user to reorder their paths by
date, distance, alphabetical order, or visibility. Once an option is selected, the list immediately
updates to reflect the chosen ordering.

Each path is presented as a compact card containing the title, a short description, the total
distance, the creation date, and two management icons: one for toggling visibility and one for
deletion. Tapping a card expands it into a detailed view that includes a map preview displaying the
full route. When expanded, the entry also provides a \textbf{Start This Path} button, which redirects
the user to the Home Screen with the selected path already loaded on the map. If the user’s current
position matches the starting point of the path, the trip can be started immediately.

Management actions trigger dedicated confirmation dialogs. Changing visibility opens a pop-up asking
the user to confirm whether they wish to switch the path between public and private. The update is
applied only after explicit confirmation. Deleting a path displays a separate confirmation dialog
warning that the action is permanent, the path is removed only if the user confirms the request.

Collapsing an expanded entry restores the compact list layout, allowing users to browse their
collection efficiently.

All interactions within this screen are available exclusively to logged-in users. Guest users cannot
view, create, start, or manage custom paths.

\section{Profile and Settings for Logged-in Users}

\begin{figure}[H]
	\centering
	\begin{minipage}[t]{0.31\textwidth}
		\centering
		\vspace{0pt}
		\phoneimage[width=\textwidth]{Design/18_profile.png}
		\caption{Profile Screen}
		\label{fig:profile_screen_mockup}
	\end{minipage}
	\hfill
	\begin{minipage}[t]{0.31\textwidth}
		\centering
		\vspace{0pt}
		\phoneimage[width=\textwidth]{Design/20_settings.png}
		\caption{Settings}
		\label{fig:settings_screen_mockup}
	\end{minipage}
	\hfill
	\begin{minipage}[t]{0.31\textwidth}
		\centering
		\vspace{0pt}
		\phoneimage[width=\textwidth]{Design/19_edit_profile.png}
		\caption{Edit Profile}
		\label{fig:edit_profile_mockup}
	\end{minipage}
\end{figure}

\subsection*{Profile Screen}
\label{sec:profile_screen}%

The Profile Screen provides logged-in users with an overview of their cycling activity and personal
account details. It is accessible through the fourth icon in the bottom navigation bar.

At the top of the page, a profile header displays the user’s avatar, name, and email address. A
small edit icon next to the name allows the user to update their personal information, while a
settings icon in the upper-right corner redirects to the Settings Screen, where account preferences
and configuration options can be adjusted.

Below the header, the \textbf{Overall Stats} section summarises long-term activity metrics, including
total distance ridden, total recorded trips, and the number of custom paths created. These values
provide a compact snapshot of the user’s cumulative progress.

Further down, the \textbf{Activity Stats} section offers a more granular breakdown. A compact
dropdown menu allows users to filter statistics by time period, such as monthly activity.
Displayed metrics may include paths and trips completed within the selected period, total distance,
riding time, average and maximum speed, elevation gain, calories burned, and other performance
indicators. Each metric is paired with a colourful icon to support quick visual identification.

All information on this screen is available exclusively to logged-in users. Guest users cannot
access the Profile Screen or view personal statistics.

\subsection*{Edit Profile}
\label{sec:edit_profile}%

The Edit Profile Screen enables logged-in users to update their account details. It includes editable
fields for the \textbf{Username} and \textbf{Email Address}, followed by a dedicated section for
changing the password.

To update the password, the user must enter their \textbf{Current Password}, then specify a
\textbf{New Password}, and confirm it in the \textbf{Confirm Password} field. This ensures the new
credentials are entered correctly before submission.

After making the desired changes, the user can tap \textbf{Save Changes} to apply them. The app
validates the input and, if successful, stores the updated information and returns the user to the
Profile Screen. A \textbf{Cancel} button allows the user to dismiss the form without saving.

This screen is accessible exclusively to logged-in users and is reached from the Profile Screen by
tapping the pencil icon next to the user's name. The user may return to the Profile view at any time
by tapping the back arrow in the upper-left corner.

\subsection*{Settings}
\label{sec:settings_screen}%

The Settings Screen allows logged-in users to customise their app experience and manage personal
preferences. It is accessible from the Profile Screen through the settings icon in the top-right
corner.

At the top of the page, the \textbf{Appearance} option lets users choose the visual theme of the
application. The current theme is displayed on the right side of the row, and tapping it opens a
selector where users may switch between available modes (e.g., Light or Dark).

Below this, the \textbf{Default Privacy} setting specifies the visibility applied to newly created
paths. Users may choose whether new paths should be public or private by default. This preference
is automatically applied during path creation but can still be overridden for individual paths.

The \textbf{Get Help} section provides direct access to the support channel, allowing the user to
contact the team if assistance is required.

Finally, a \textbf{Sign Out} button appears at the bottom of the screen. Selecting it logs the user
out and returns the application to guest mode.

This screen is available exclusively to logged-in users, as guest users do not manage appearance
settings, default visibility, or account-related actions.

\section{Error Messages}
\label{sec:error_messages}%

\begin{figure}[H]
	\centering
	\begin{minipage}[t]{0.65\textwidth}

		The application includes a global error-handling mechanism that provides consistent feedback whenever
		an unexpected issue occurs. In such cases, a dedicated \textbf{Error} pop-up appears at the top of the
		screen, displaying a clear title and a short description of the problem. The message shown in the
		dialog is dynamically replaced with context-specific information, such as network failures,
		permission issues, or invalid operations.

		When the error dialog is visible, the underlying interface is dimmed to draw the user’s attention to
		the alert. The pop-up includes a single \textbf{Close} button that dismisses the message and returns
		the user to the previous screen. Tapping outside the pop-up also closes the notification.

		This error dialog may appear at any point in the application, including during path searches,
		trip management, report submission, navigation, or when modifying custom paths, ensuring that errors
		are communicated in a consistent and predictable manner.

	\end{minipage}
	\hfill
	\begin{minipage}[t]{0.31\textwidth}
		\centering
		\vspace{-1em}
		\phoneimage[width=\textwidth]{Design/21_error.png}
		\caption{Error Pop-up}
		\label{fig:error_popups_mockup}
	\end{minipage}
\end{figure}

