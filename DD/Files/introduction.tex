% --------------------------------------------------------------------------
% Introduction
% --------------------------------------------------------------------------

% --------------------------------------------------------------------------
% Purpose 
% --------------------------------------------------------------------------
\section{Purpose}
\label{sec:purpose}%
	The purpose of this document is to provide a comprehensive design framework for the development, integration, and testing of the system. This document serves as a guide for developers and testers, offering a structured approach to building, integrating, and validating the system components while maintaining alignment with project goals and user needs.

% --------------------------------------------------------------------------
% Scope
% --------------------------------------------------------------------------

\section{Scope}
\label{sec:scope}%
	The scope of the Best Bike Paths (BBP) platform is to create and maintain a comprehensive inventory of bike paths by supporting users in recording, analyzing, and sharing cycling activities and path conditions. It allows logged-in users to record their trips, automatically collecting GPS and sensor data or manually inputting information, which is then analyzed to provide statistics and identify obstacles like potholes. Crucially, BBP incorporates a mechanism for validating and merging this user-reported data, ensuring the quality and freshness of path information before making it publishable. Finally, the system enables all users, logged in or not, to search for the best-scored path between any two points, leveraging the validated, community-contributed inventory to optimize their cycling routes. \newline
	This document describes the architecture, components, their detailed functionalities, interactions, and interfaces, ensuring a clear understanding of the system’s structure and behavior. For a comprehensive overview of system features, refer to the RASD.


\section{Definitions, Acronyms, Abbreviations}
\label{sec:definitions}%

\subsection{Definitions}
\label{subsec:definitions}%

\begin{itemize}
	\item \textbf{HTTPS:} Secures all communication between the app and the server using encryption.
	\item \textbf{HTTP/REST:} Defines the API structure, using simple request-response calls for all application actions (GET, POST).
	\item \textbf{TCP/IP:} The fundamental protocols ensuring reliable and ordered delivery of data packets across the network.
	\item \textbf{ORM/ODM:} Maps application classes directly to database tables or documents, simplifying data manipulation.
	\item \textbf{DAO Pattern:} Isolates database access logic from the main business logic for modularity.
	\item \textbf{MVC Pattern:} Separates the application into three interconnected components: Model (data), View (UI), and Controller (business logic).
	\item 	\textbf{Connection Pooling:} Maintains a set of open database connections to improve performance and scalability by reducing connection overhead.
\end{itemize}

\subsection{Acronyms}
\label{subsec:acronyms}%

\begin{itemize}
	\item \textbf{BBP:} Best Bike Paths.
	\item \textbf{DD:} Design Document.
	\item \textbf{CRUD:} Create, Read, Update, Delete.
	\item \textbf{REST:} Representational State Transfer.
	\item \textbf{HTTP:} HyperText Transfer Protocol.
	\item \textbf{JSON:} JavaScript Object Notation.
	\item \textbf{DB:} Database.
	\item \textbf{DBMS:} DataBase Management System.
	\item \textbf{RASD:} Requirement Analysis and Specification Document.
	\item \textbf{GPS:} Global Positioning System.
	\item \textbf{API:} Application Programming Interface.
	\item \textbf{UI:} User Interface.
	\item \textbf{ORM:} Object-Relational Mapping.
	\item \textbf{ODM:} Object-Document Mapping.
	\item \textbf{TCP/IP:} Transmission Control Protocol/Internet Protocol.
\end{itemize}

% --------------------------------------------------------------------------
% Revision history
% --------------------------------------------------------------------------

\section{Revision history}
\label{sec:revhistory}%

\begin{itemize}
	\item Version 1.0 (23 December 2025);
\end{itemize}

% --------------------------------------------------------------------------
% Reference Documents
% --------------------------------------------------------------------------
\section{Reference Documents}
\label{sec:refdocs}%

The preparation of this document was supported by the following reference materials:
\begin{itemize}
	\item IEEE Standard for Software Requirement Specifications \cite{ieee};
	\item Assignment specification for the RASD and DD of the Software Engineering II course,
	      held by professors Matteo Rossi, Elisabetta Di Nitto, and Matteo Camilli at the
	      Politecnico di Milano, A.Y. 2025/2026 \cite{dd};
	\item Slides of the Software Engineering 2 course available on WeBeep \cite{slides};
\end{itemize}

% --------------------------------------------------------------------------
% Document Structure
% --------------------------------------------------------------------------

\section{Document Structure}
\label{sec:docstructure}%

Mainly the current document is divided into 7 chapters, which are:
\begin{enumerate}
	\item \textbf{Introduction:} Provides an overview of the document, outlining its purpose, scope, and relevance to the project.
	\item \textbf{Architectural Design:} Explains the technical architecture of the system, including the components, layers, and their interactions.
	\item \textbf{User Interface Design:} Details the visual and functional design of the user interfaces, emphasizing usability and user experience.
	\item \textbf{Requirements Traceability:} Maps the requirements to their implementation.
	\item \textbf{Implementation, Integration and Test Plan:} Describes the approach to system development, integration strategies, and the planned testing methods to ensure quality and reliability.
	\item \textbf{Effort Spent:} Summarizes the distribution of time and effort invested by each team member in completing the document and its sections.
	\item \textbf{References:} A list of all sources (including Software) and documents used in the preparation of this document.

\end{enumerate}