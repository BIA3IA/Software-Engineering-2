% --------------------------------------------------------------------------
% Introduction
% --------------------------------------------------------------------------

% --------------------------------------------------------------------------
% Purpose 
% --------------------------------------------------------------------------
\section{Purpose}
\label{sec:purpose}%

The purpose of this document is to refine the system requirements defined in the \textbf{Requirement Analysis
	and Specification Document (RASD)} and translate them into architectural, component-level,
and interaction design choices. This document provides guidance for developers and testers by
defining a structured approach to the design, integration, and validation of system components,
ensuring alignment with the project goals and user requirements.

% --------------------------------------------------------------------------
% Scope
% --------------------------------------------------------------------------

\section{Scope}
\label{sec:scope}%

The scope of the \textbf{Best Bike Paths (BBP)} platform is to create and maintain a comprehensive inventory of bike paths by supporting users in recording,
analyzing, and sharing cycling activities and path conditions. It allows logged-in users to record their trips, automatically collecting GPS and sensor
data or manually inputting information, which is then analyzed to provide statistics and identify obstacles. Crucially, BBP incorporates
a mechanism for validating and merging this user-reported data, ensuring the quality and freshness of path information before making it publishable. Finally,
the system enables all users, logged in or not, to search for the best-scored path between two points, leveraging the validated, community-contributed
inventory to optimize their cycling routes.

This document focuses on the architectural and component-level design of the system, describing how its elements interact and cooperate to satisfy the
requirements defined for the platform. The complete and formal definition of system requirements is provided in the \textbf{RASD}.


\section{Definitions, Acronyms, Abbreviations}
\label{sec:definitions}%

\subsection{Definitions}
\label{subsec:definitions}%

\begin{itemize}
	\item \textbf{HTTPS:} Secures all communication between the app and the server using encryption.
	\item \textbf{REST API:} The set of HTTP-based interfaces exposed by the backend service, used by the mobile client to exchange data in JSON format.
	\item \textbf{DAO Pattern:}  A design pattern that isolates data access logic from the core business logic, improving modularity and maintainability.
\end{itemize}

\subsection{Acronyms}
\label{subsec:acronyms}%

\begin{itemize}
	\item \textbf{BBP:} Best Bike Paths.
	\item \textbf{DD:} Design Document.
	\item \textbf{CRUD:} Create, Read, Update, Delete.
	\item \textbf{REST:} Representational State Transfer.
	\item \textbf{HTTP:} HyperText Transfer Protocol.
	\item \textbf{JSON:} JavaScript Object Notation.
	\item \textbf{DBMS:} DataBase Management System.
	\item \textbf{RASD:} Requirement Analysis and Specification Document.
	\item \textbf{GPS:} Global Positioning System.
	\item \textbf{API:} Application Programming Interface.
	\item \textbf{UI:} User Interface.
\end{itemize}

% --------------------------------------------------------------------------
% Revision history
% --------------------------------------------------------------------------

\section{Revision history}
\label{sec:revhistory}%

\begin{itemize}
	\item Version 1.0 (07 January 2026);
\end{itemize}

% --------------------------------------------------------------------------
% Document Structure
% --------------------------------------------------------------------------

\section{Document Structure}
\label{sec:docstructure}%

Mainly the current document is divided into 7 chapters, which are:
\begin{enumerate}
	\item \textbf{Introduction:} provides an overview of the document, outlining its purpose, scope, and relevance to the project.
	\item \textbf{Architectural Design:} explains the technical architecture of the system, including the components, layers, and their interactions.
	\item \textbf{User Interface Design:} details the visual and functional design of the user interfaces, emphasizing usability and user experience.
	\item \textbf{Requirements Traceability:} maps the requirements defined in the RASD to the architectural components and design decisions presented in this document.
	\item \textbf{Implementation, Integration and Test Plan:} describes the approach to system development, integration strategies, and the planned testing methods to ensure quality and reliability.
	\item \textbf{References:} lists all sources (including Software) and documents used in the preparation of this document.
	\item \textbf{Effort Spent:} summarizes the distribution of time and effort invested by each team member in completing the document and its sections.
\end{enumerate}